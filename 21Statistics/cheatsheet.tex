\documentclass{article}
\usepackage{cmap}
\usepackage[utf8]{inputenc}
\usepackage[english,ukrainian]{babel}
\usepackage{graphicx}
\usepackage{geometry}
\usepackage{listings}
\usepackage{indentfirst}
\usepackage{caption}
\usepackage{amsmath}
\geometry{
	a4paper,
	left=20mm,
	right=20mm,
	top=20mm,
	bottom=20mm
}
\lstset{
	extendedchars=\true,
	tabsize=4,
	language=python,
	showstringspaces=false,
	showtabs=false,
	frame=lrtb,
	columns=fixed,
	keepspaces,
	breaklines=true
}
\graphicspath{ {pictures} }
\setlength{\parindent}{4em}

\begin{document}
\begin{Large}
\section*{Комбінації, перестановки, розміщення}
\begin{list}{-}{}
	\item Комбінації $C_n^k=\frac{n!}{k!(n-k)!}$ - \textit{Порядок не важливий}
	\item Комбінації з повтореннями $\overline{C_n^k}=C_{n+k-1}^k$
	\item Розміщення $A_n^k=k!C_n^k=\frac{n!}{(n-k)!}$ - \textit{Порядок важливий}
	\item Розміщення з повтореннями $\overline{A_n^k}=n^k$
	\item Перестановки $P_n=n!$
	\item Перестановки з повтореннями $\overline{P_n}=\frac{n!}{n_1!n_2!...n_k!}$
\end{list}

\section*{Математичне сподівання}
Середнє значення випадкової величини

$M(x)=np)$

$M(x)=\sum_{i=1}^{i}x_ip_i$, $M(x)=\int_{-\infty}^{+\infty}xf(x)dx$
\section*{Дисперсія}
Розсіювання випадкової величини від математичного сподівання

$D(x)=npq$

$D(x)=M(x)^2-M(x^2)$

\section*{Мода}
Те значення випадкової величини ймовірність якоїі найбільша

\section*{Медіана}
таке значення випадкової величини, відносно якого рівноймовірно одержання більшого або меншого значення випадкової величини

Площа під кривою розподілу ділиться навпіл $F(x)=0.5$

\section*{Коефіцієнт варіації}
показує, наскільки велике розсіювання порівняно із середнім значенням випадкової величини

$V=\frac{\sigma_x}{M(x)}\cdot100\%$

\section*{Функція розподілу}
\begin{enumerate}
	\item Неспадна функція
	\item Значення лежить в межах [0;1]
	\item Неперервна зліва
	\item \{0 .. 1
\end{enumerate}

\section*{Щільність розподілу}
Похідна від функції розподілу

\begin{enumerate}
	\item $> 0$
	\item $\int_{-\infty}^{+\infty}f(x)dx = 1$
\end{enumerate}

\section*{Найімовірніше число появи випадкової події}
$np-q\le m_0 \le np+p$

\section*{Схема Бернуллі}
Ймовірність того, що незалежна подія настане рівно $m$ разів з $n$ випробувань
$P_n(m)=C_n^m\cdot p^m\cdot q^{n-m}$

\section*{Локальна теорема Мавра-Лапласа}
Яка ймовірність настання незалежної події рівно $m$ разів з $n$ випробувань з ймовірністю успіху $p$ < ймовірності невдачі $q$

$P_n(m)=\frac{1}{\sqrt{npq}}\phi(x)$, $x=\frac{m - np}{\sqrt{npq}}$

$\phi$ - функція Гаусса

\section*{Інтегральна теорема Мавра-Лапласа}
Яка ймовірність настання незалежної події від $m_1$ до $m_2$ разів з $n$ випробувань з ймовірністю успіху $p$ < ймовірності невдачі $q$

$P_n(m)=\Phi(x_2)-\Phi(x_1)$, $x_1=\frac{m_1 - np}{\sqrt{npq}}$, $x_2=\frac{m_2 - np}{\sqrt{npq}}$

$\Phi(-x) = -\Phi(x)$, $\Phi$ - функція Лапласа

\section*{Теорема Пуассона}
n$\rightarrow+\infty$, $p\rightarrow 0$

$\lambda=np$

Точна імовірність (табл)
$P_n(m)=\frac{\lambda^m}{m!}e^{-\lambda}$

Імовірність в межах
$P_n(m_1,m_2)=\sum_{m=m_1}^{m_2}P_n(m)=\sum_{m=m_1}^{m_2}\frac{\lambda^m}{m!}e^{-\lambda}$

\section*{Розподіли}
\subsection*{Дискретні розподіли}
Дано $n$ - кількість випробувань
\begin{list}{-}{}
	\item Пуассона ($p$ $\downarrow\downarrow$ $n$ $\uparrow\uparrow$, $np < 10$)
	\item Геометричний (до першого успіху)
	\item Біномний - незалежні спроби
\end{list}
\subsection*{Неперервні розподіли}
\begin{list}{-}{}
	\item Рівномірний - $\text{щільність розподілу} = const$
	\item Показниковий - $\lambda$
	\item Нормальний - $\sigma$ $a$
\end{list}

\section*{Коваріація}
$k_{xy}=M(xy)-M(x)M(y)$

\section*{Коефіцієнт кореляції}
$1\le r_{xy}=\frac{k_{xy}}{\sigma_x\sigma_y}\le 1$

$r_{xy}=0$ - некорельовані, інакше корельовані

\section*{Незалежність}
$F(x,y)=F(x)\cdot F(y)$

Потік подій — послідовність однотипних подій. Середня к-ть подій за одиницю часу — інтенсивність подій (лямбда).
Стаціонарний, якщо лямбда стала. Ординарний, якщо змінна
\end{Large}
\end{document}