\documentclass[oneside,14pt]{extarticle}
\usepackage[utf8]{inputenc}
\usepackage[english,ukrainian]{babel}
\usepackage{amssymb,amsfonts,amsmath,amsthm,mathtext,textcomp}

\usepackage[includehead, headsep=0pt, footskip=0pt, top=2cm, bottom=2cm, left=2cm, right=15mm]{geometry}
\usepackage{indentfirst}
\usepackage[onehalfspacing]{setspace}
\usepackage[headings]{fancyhdr}
\usepackage{etoolbox}
\usepackage{flafter}
\usepackage{hyperref}
\usepackage{listings}
\usepackage{graphicx}
\usepackage{float}
\usepackage[center]{titlesec}
\titlelabel{\thetitle.\quad}
\usepackage{array}
\fancyhf{}
\renewcommand{\headrulewidth}{0pt}
\renewcommand{\baselinestretch}{1.5}
\pagestyle{fancy}
\fancyfoot[R]{\thepage}
\lstset{breaklines=true,}
\graphicspath{ {./pictures} }
\counterwithin{figure}{section}

\begin{document}
\begin{titlepage}
	\begin{center}
		\textbf{МІНІСТЕРСТВО ОСВІТИ І НАУКИ УКРАЇНИ\\
		НАЦІОНАЛЬНИЙ УНІВЕРСИТУТ <<ЛЬВІВСЬКА ПОЛІТЕХНІКА>>\\
		НАВЧАЛЬНО-НАУКОВИЙ ІНСТИТУТ АДМІНІСТРУВАННЯ\\
		ТА ПІСЛЯДИПЛОМНОЇ ОСВІТИ\\}
		Кафедра адміністративного та фінансового менеджменту

		
		\vspace{130pt}
		\textbf{КОНТРОЛЬНА РОБОТА}\\
		На тему:
		<<Історія успіху компанії Reddit>>\\
		з дисципліни\\
		\textbf{<<ВБ1.2. ТЕХНОЛОГІЇ РОЗРОБКИ СТАРТАПІВ>>}
		\vspace*{40pt}
		
		\begin{flushright}
			\textbf{Виконав}:\\
			
			студент групи ТРС-АФМ-209\\
			Коваленко Д.М.\\
			\vspace{10pt}
			\textbf{Прийняла}:\\
			асист. Н.О. Іванкова
			
			\vspace{28pt}
			«\rule{1cm}{0.15mm}» \rule{1.5cm}{0.15mm} 2023 р.\\
			$\sum$ = \rule{1cm}{0.15mm}……………\\
			
		\end{flushright}
		\vspace{\fill}
		Львів — 2023
	\end{center}
\end{titlepage}
\setcounter{page}{2}
\tableofcontents
\newpage

\section*{ВСТУП}
\addcontentsline{toc}{section}{ВСТУП}

Зростання стартапів за останні роки спричинило революцію у світі бізнесу. Ці молоді інноваційні компанії порушили традиційні бізнес-моделі та створили нові продукти та послуги, які змінили спосіб нашого життя та роботи. Одним із відомих стартапів останного часу є Reddit, онлайн-платформа для обміну новинами та контентом, яка перетворилася на один із найбільших веб-сайтів у світі.

Reddit був заснований у 2005 році Стівом Хаффманом і Алексісом Оганяном, двома друзями по коледжу, які хотіли створити онлайн-спільноту, де користувачі могли б ділитися посиланнями та обговорювати цікаві теми. Відтоді Reddit виріс і став платформою з понад 430 мільйонами активних користувачів щомісяця та оцінкою в 10 мільярдів доларів. Його успіх пояснюється низкою факторів, у тому числі вмістом, орієнтованим на користувачів, модерацією спільноти та розробкою з відкритим кодом.

Актуальність дослідження цієї теми полягає в тому, що успіх Reddit — це не лише історія підприємництва, а й відображення мінливого ландшафту Інтернету та сили онлайн-спільнот. Розвиток соціальних медіа та контенту, створеного користувачами, порушив традиційні медіа-канали та породив нові форми взаємодії та співпраці. Вивчаючи історію Reddit, ми можемо отримати уявлення про динаміку онлайн-спільнот і фактори, які сприяють їх успіху.

Історія Reddit — це захоплююча історія підприємництва, розвитку спільноти та інновацій. Будучи одним із найуспішніших стартапів останнього часу, Reddit продемонстрував силу контенту, керованого користувачами, модерації спільноти та розробки з відкритим кодом. Вивчаючи фактори, які сприяли його успіху, ми можемо отримати уявлення про динаміку онлайн-спільнот і мінливий ландшафт Інтернету.

\section*{ОСНОВНА ЧАСТИНА}
\addcontentsline{toc}{section}{ОСНОВНА ЧАСТИНА}

Засновники Reddit Алексіс Оганян і Стів Хаффман з самого початку мали чітке бачення свого стартапу. Вони хотіли створити онлайн-спільноту, де люди могли б ділитися посиланнями, новинами та ідеями один з одним у простий, привабливий і демократичний спосіб. Їхня ідея полягала в тому, щоб створити платформу, яка дозволяла б користувачам голосувати за вміст, щоб найпопулярніші публікації піднімалися вгору та їх бачили більше людей.

Оганян і Хаффман також вірили в силу розбудови спільноти, і вони наполегливо працювали, щоб культивувати почуття причетності та спільної мети серед користувачів Reddit. Вони заохочували користувачів створювати та модерувати власні субредити, які були схожі на міні-спільноти в рамках більшої спільноти Reddit, зосереджені на конкретних темах чи інтересах. Вони також звернули увагу на пряму взаємодію зі своїми користувачами, відповідаючи на відгуки та беручи участь в обговореннях на сайті.

Іншим ключовим фактором успіху Reddit була готовність засновників експериментувати та адаптуватися. Вони постійно налаштовували та вдосконалювали функції сайту, ґрунтуючись на відгуках користувачів і власних спостереженнях про те, що працює, а що ні. Вони також залишалися відкритими для нових ідей і партнерства, що допомогло їм розширити охоплення та збільшити базу користувачів.

Загалом, успіх Reddit можна пояснити поєднанням факторів, зокрема чітким і переконливим баченням, сильним почуттям спільності та бажанням експериментувати та адаптуватися. Залишаючись вірними своїм основним цінностям і чуйно реагуючи на потреби користувачів, Оганян і Хаффман змогли створити стартап, який став однією з найвпливовіших і найпоширеніших онлайн-платформ у світі.

Перший етап розробки платформи Reddit був вирішальним для успіху стартапу. На цьому етапі засновники, Стів Хаффман і Алексіс Оганян, зосередилися на створенні мінімально життєздатного продукту (MVP), щоб перевірити ринок і отримати відгуки від перших користувачів.

MVP був простим форумом, де користувачі могли надсилати посилання та голосувати за них. Ця базова функція дозволила засновникам випробувати концепцію контенту, створеного користувачами, та обговорення, керованого спільнотою. Початкова версія Reddit була запущена в 2005 році і залучила невелику, але пристрасну базу користувачів.

Перший етап розвитку також передбачав забезпечення фінансування стартапу. Засновники представили свою ідею інвесторам і залучили 12 000 доларів від Y Combinator, акселератора стартапів. Це фінансування дозволило їм зосередитися на розвитку платформи та вдосконаленні її функцій.

Перший етап розвитку Reddit був вирішальним для створення основної функціональності платформи та створення бази лояльних користувачів. Уроки, засвоєні на цьому етапі, вплинули на майбутній розвиток платформи та заклали основу її успіху як веб-сайту соціальних новин і дискусій.

Оскільки база користувачів Reddit зростала, засновники зосередилися на покращенні функцій платформи та взаємодії з користувачем. Вони представили нові функції, такі як subreddits, які дозволили користувачам створювати нішеві спільноти навколо певних тем, що цікавлять. Ця інновація кардинально змінила правила, оскільки дозволила Reddit перетворитися з простого форуму для обміну посиланнями на різноманітну платформу, керовану спільнотою.

На цьому етапі розвитку Reddit також почав монетизувати свою платформу. Засновники експериментували з різними моделями доходу, включаючи рекламу та спонсорство, але врешті-решт зупинилися на безкоштовній моделі, де користувачі могли придбати преміум-членство для доступу до додаткових функцій.

Успіх Reddit на ранніх стадіях можна пояснити кількома факторами. По-перше, засновники виявили прогалину на ринку для платформи, яка сприяла обговоренню спільнотою та контенту, створеному користувачами. Вони також визнали важливість відгуків користувачів у формуванні розвитку платформи.

Крім того, засновники використали можливості соціальних мереж для просування Reddit і залучення нових користувачів. Вони використовували такі платформи, як Twitter і Facebook, щоб спілкуватися з потенційними користувачами та поширювати інформацію про свій стартап.

Підсумовуючи, перший етап розвитку Reddit був вирішальним для успіху платформи. Зосередженість засновників на створенні мінімально життєздатного продукту, забезпеченні фінансування та зборі відгуків від перших користувачів дозволила їм створити основну функціональність платформи та створити базу лояльних користувачів. Уроки, засвоєні на цьому етапі, вплинули на майбутній розвиток платформи та заклали основу для її подальшого успіху як провідного веб-сайту соціальних новин і дискусій.

Впровадження нових функцій і можливостей є одним із ключових факторів, які сприяли успіху Reddit. Ця популярна платформа соціальних новин і дискусій постійно вдосконалювалася та розвивалася, щоб задовольнити мінливі потреби своїх користувачів.

З моменту заснування в 2005 році Reddit додав цілий ряд нових функцій і можливостей, у тому числі subreddits (тематичні спільноти), живі події, власний відеохостинг і мобільний додаток. Ці оновлення полегшили користувачам пошук і обмін вмістом, взаємодію з іншими та персоналізацію свого досвіду на платформі.

Крім того, Reddit також використовує новітні технології, такі як штучний інтелект (AI) і машинне навчання (ML), щоб покращити свою функціональність. Наприклад, платформа використовує штучний інтелект, щоб рекомендувати контент користувачам на основі їхніх інтересів і поведінки, і ML для виявлення та видалення спаму та невідповідного вмісту.

Таким чином, впровадження нових функцій і можливостей було вирішальним для успіху Reddit як стартапу. Залишаючись на випередженні та постійно вдосконалюючи свою платформу, Reddit зберіг свою позицію як провідний сайт соціальних новин і дискусій.

Одним із ключових способів, за допомогою яких Reddit виділяється серед інших платформ соціальних медіа, є його субредити. Це спільноти, створені користувачами, зосереджені на певних темах, від технологій до спорту та мемів. Можливість приєднуватися та брати участь у субредітах дозволила користувачам адаптувати свій досвід на Reddit відповідно до своїх інтересів і створила відчуття спільноти на платформі.

Reddit також досяг успіху у впровадженні нових функцій, які розширили його охоплення та привабливість. Наприклад, у 2019 році платформа запровадила події в прямому ефірі, дозволяючи користувачам спілкуватися один з одним у режимі реального часу щодо певної теми чи події. Цю функцію використовували для всього: від живих сесій із запитаннями та відповідями зі знаменитостями до обговорень екстрених новин.

Ще однією важливою подією для Reddit став запуск його мобільного додатку в 2016 році. Це зробило платформу більш доступною та зручною для користувачів, які пересуваються, і допомогло збільшити кількість мобільних користувачів сайту. Крім того, програма представила нові функції, такі як чат і push-повідомлення, що ще більше покращує взаємодію з користувачем.

Окрім реклами, Reddit також реалізував кілька функцій монетизації для своїх користувачів, зокрема Reddit Gold і Reddit Premium. Ці функції надають користувачам різноманітні переваги та переваги, такі як перегляд без реклами, ексклюзивний доступ до певних субредітів і можливість налаштовувати свої профілі.

Однак, незважаючи на ці зусилля з монетизації, успіх Reddit можна значною мірою пояснити його сильною залученістю спільноти та вмістом, орієнтованим на користувачів. Унікальна система субредітів Reddit дозволяє користувачам створювати та керувати вмістом на основі їхніх інтересів, сприяючи почуттю спільності та зв’язку.

Рекламні стратегії Reddit створені для того, щоб надати рекламодавцям можливість орієнтуватися на конкретну аудиторію та охоплювати її за допомогою різних форматів, включаючи медійну рекламу, спонсорований вміст і рекламні публікації. Медійна реклама – це традиційна банерна реклама, яка з’являється в різних частинах веб-сайту, тоді як спонсорований вміст – це вміст, створений рекламодавцями, який виглядає та виглядає як звичайні публікації Reddit. Рекламовані публікації схожі на спонсорований вміст, але їх створюють користувачі Reddit і просувають рекламодавці, щоб охопити ширшу аудиторію.

Зусилля Reddit щодо монетизації також були успішними в отриманні прибутку та надання користувачам додаткових переваг. Reddit Gold — це сервіс на основі підписки, який надає користувачам різноманітні переваги, зокрема перегляд без реклами, можливість створювати власні аватари та теми, а також доступ до ексклюзивних субредітів. Reddit Premium — це оновлена версія Reddit Gold, яка включає додаткові переваги, наприклад щомісячні монети, які можна витрачати на нагороди та значки.

Однак успіх Reddit не можна пояснити лише зусиллями з монетизації. Унікальний контент платформи, створений користувачами, і підхід, керований спільнотою, також є основними факторами її успіху. Користувачі можуть створювати та керувати вмістом на основі своїх інтересів, а система голосування дозволяє найпопулярнішому вмісту підніматися на вершину. Це створює відчуття спільноти та заохочує взаємодію та взаємодію між користувачами.

На додаток до цих факторів, успіх Reddit частково пояснюється його раннім впровадженням мобільних технологій і його здатністю адаптуватися до мінливої поведінки та уподобань користувачів. Компанія інвестувала в мобільні додатки та покращила дизайн і функціональність свого веб-сайту, щоб забезпечити безперебійну роботу користувача на будь-якому пристрої.

Однією з головних причин успіху Reddit є важливість створення та підтримки спільнот. Ці спільноти або субредити надають користувачам простір для обміну своїми інтересами, спілкування з однодумцями та участі в дискусіях. Можливість створювати ці спільноти та приєднуватися до них стала суттєвою рушійною силою залучення та утримання користувачів.

Ще одним важливим фактором є взаємодія з користувачами. Дизайн Reddit заохочує користувачів брати участь в обговореннях і ділитися власним вмістом. Це створює відчуття власності та інвестицій у платформу, що допомагає створити базу лояльних користувачів. Крім того, система голосування за та проти платформи дозволяє користувачам керувати вмістом, який вони бачать, і винагороджує якісні внески.

Успіх Reddit також можна пояснити його інноваційними функціями та підходом до курування вмісту. Система голосування на платформі дозволяє користувачам голосувати за або проти дописів і коментарів, що визначає їх видимість і рейтинг у subreddit. Ця система допомагає висвітлювати найбільш релевантний і цікавий контент, а також заохочує користувачів створювати високоякісні публікації, які, ймовірно, отримають підтримку.

Іншим ключовим фактором успіху Reddit є його здатність адаптуватися та розвиватися з часом. Платформа постійно впроваджує нові функції та інструменти для покращення взаємодії з користувачем, такі як редизайн веб-сайту та запровадження мобільного додатку. Ця гнучкість і готовність до змін допомогли Reddit залишатися актуальним і привабливим для широкого кола користувачів.

Крім того, Reddit зміг використати створений користувачами контент для створення рекламних можливостей для бізнесу. Платформа пропонує варіанти цільової реклами, які дозволяють компаніям охопити певні спільноти та демографічні показники, а також пропонує можливості для фірмового контенту та партнерства.

\section*{ВИСНОВКИ}
\addcontentsline{toc}{section}{ВИСНОВКИ}

Reddit — це успішна платформа соціальних медіа, яка швидко розвивається з моменту запуску в 2005 році. Завдяки своїй унікальній системі контенту, створеного користувачами, і модерації, керованої спільнотою, Reddit став улюбленим місцем для мільйонів користувачів, які прагнуть спілкуватися з іншими. , обмінюватися інформацією та брати участь у жвавій дискусії на різноманітні теми.

Дослідження успіху Reddit виявило кілька ключових факторів, які сприяли його популярності та зростанню. До них відноситься дизайн платформи, орієнтований на користувача, який забезпечує легку навігацію та налаштування, а також зосередженість на вихованні почуття спільності та приналежності серед користувачів.

Крім того, бізнес-модель Reddit, яка базується на цільовій рекламі та створеному користувачами контенті, довела високу ефективність у створенні прибутку та підтримці зростання. Це дозволило платформі залишатися незалежною та прибутковою, навіть якщо вона розширила базу користувачів і урізноманітнила пропозицію контенту.

Забігаючи наперед, можна відзначити кілька потенційних шляхів подальшого зростання та розвитку Reddit. Вони включають розширення бази користувачів за допомогою цілеспрямованого маркетингу та інформаційно-просвітницьких заходів, подальшу диверсифікацію пропозицій контенту, щоб зацікавити ширше коло користувачів, і дослідження нових джерел доходу, таких як платні підписки або преміум-контент.

Загалом, успіх Reddit є свідченням потужності контенту, створеного користувачами, і залучення спільноти до сучасного цифрового середовища. Оскільки платформа продовжує розвиватися та рости, буде цікаво спостерігати, як вона адаптується до мінливих потреб і вподобань своїх користувачів, а також які нові інновації вона привносить у світ соціальних медіа та створення онлайн-спільноти.

\newpage
\section*{Список використаної літератури}
\addcontentsline{toc}{section}{Список використаної літератури}
\begin{enumerate}
	\item Ohanian, A. (2013). Without Their Permission: How the 21st Century Will Be Made, Not Managed. Business Plus.
\item Graham, P. (2005). Reddit. PaulGraham.com. (\href{https://PaulGraham.com}{https://paulgraham.com})
\item Swartz, A. (2006). The Weblog: An Extremely Democratic Form of Publishing. Raw Thought. (\href{http://www.aaronsw.com/weblog/}{http://www.aaronsw.com/weblog/})
\end{enumerate}

\end{document}
