\documentclass[oneside,14pt]{extarticle}
\usepackage[utf8]{inputenc}
\usepackage[english,ukrainian]{babel}
\usepackage[T1]{fontenc}
\usepackage{amssymb,amsfonts,amsmath,amsthm,mathtext,textcomp}

\usepackage[includehead, headsep=0pt, footskip=0pt, top=2cm, bottom=2cm, left=2.5cm, right=1cm]{geometry}
\usepackage{indentfirst}
\usepackage[onehalfspacing]{setspace}
\usepackage[headings]{fancyhdr}
\usepackage{etoolbox}
\usepackage{flafter}
\usepackage{listings}
\usepackage{graphicx}
\usepackage{float}
\usepackage[center]{titlesec}
\usepackage{array}
\fancyhf{}
\renewcommand{\headrulewidth}{0pt}
\fancyhead[R]{\thepage}
\pagestyle{fancy}
\lstset{breaklines=true,}
\graphicspath{ {./pictures} }
\counterwithin{figure}{section}
\titlelabel{\thetitle.\quad}
\renewcommand*{\thesection}{Розділ~\arabic{section}}
\renewcommand*{\thesubsection}{\roman{subsection}}
\usepackage{tocloft}
\setlength{\cftsecnumwidth}{5em}

\begin{document}
\begin{titlepage}
	\begin{center}
		Національний університет <<Львівська політехніка>>\\
		Кафедра програмного забезпечення
		
		\vspace{40pt}
		\textbf{\LARGE КУРСОВА РОБОТА}\\
		{\large
		\textbf{з дисципліни <<Бази даних>>}\\
		на тему:\\
		<<Інформаційна система для ...>>
		}
		\vspace*{40pt}
		
		\begin{flushright}
		    \begin{minipage}{0.6\textwidth}
		        \underline{Виконав}:\\
                Стедент спеціальності 121\\
			    <<Інженерія програмного забезпечення>>\\
			    групи ПЗ-32\\
			    Коваленко Д.М.
			    \bigbreak
			    
			    \underline{Керівник}:\\
			    асистент кафедри програмного забезпечення\\
			    Білоіваненко М.В.
			    \bigbreak
			    
			    \underline{Оцінка}:\\
			    Національна шкала \rule{6.35cm}{0.15mm}\\			
			    Кількість балів \rule{2cm}{0.15mm} Оцінка ECTS \rule{2cm}{0.15mm}
			    \bigbreak
            \end{minipage}
		\end{flushright}
		\vspace{40pt}
		Члени комісії \hspace{1.9cm} \rule{3cm}{0.15mm} \hspace{1cm} Білоіваненко М.В.\\
		{\small\vspace{-5pt}(підпис)}\\
		\hspace{2.65cm}\hspace{1.9cm}  \rule{3cm}{0.15mm} \hspace{1cm} Цимбалюк Т.М.\\
		{\small\vspace{-5pt}(підпис)}\\
		
		\vspace{\fill}
		Львів — 2024
	\end{center}
\end{titlepage}
\setcounter{page}{2}
\tableofcontents
\newpage

\section{Аналіз предметної області та постановка завдання}
\subsection{Опис предметної області}
\subsection{Вимоги до обробки даних}

\subsection{Постановка завдання}
\begin{list}{•}{Функції інформаційної системи:}
    \item Структуроване зберігання даних в реляційній СУБД відповідно до потреб предметної області.
    \item Зчитування та завантаження даних для 3-5 таблиць із зовнішнього файлу, первинне заповнення тестовими даними решти таблиць шляхом випадкової генерації за правилами.
    \item Зручні функції внесення, отримання та пошуку, фільтрації даних через візуальний інтерфейс користувача (мінімально для 3 сутностей предметної області).
    \item Реалізація моделі обмеження доступу користувачів на основі ролей СУБД (мінімальна кількість ролей – 3).
    \item Розробка API-інтерфейсу для інтеграції із зовнішніми системами (мінімально 1 запит на додавання, 1 на отримання інформації).
    \item Збереження протоколу дій користувачів при управлінні даними, інструмент отримання зрізу протоколу за обраний період часу.
    \item Реалізація одного засобу візуалізації, моніторингу даних, формування зведених звітів (можливість обрати діапазони виведення або інші фільтри).
    \item Експорт накопичених даних, що відповідають декільком сутностям, у файл CSV, XML, JSON або RDF-формату.
\end{list}
\newpage

\section{Розробка моделей зберігання даних}
\subsection{Концептуальне проектування}
\subsection{Вимоги до системи накопичення даних}
\subsection{Логічне проектування схеми бази даних}
\subsection{Реалізація процедур бізнес-логіки}
\newpage

\section{Розробка програмного коду}
\subsection{Обґрунтування обраної архітектури}
\subsection{Структурна модель інформаційної системи}
\subsection{Призначення модулів та компонентів системи}
\subsection{Особливості реалізації та елементи інтерфейсу}
\newpage

\section{Наповнення бази даних}
\subsection{Опис джерела історичних даних}
\subsection{Опис процесів генерації тестових даних}
\subsection{Трансформація та первинне завантаження даних}
\newpage

\section{Функціональні можливості для користувача}
\subsection{Опис інтерфейсу у відповідності до бізнес-процесів}
\subsection{Засоби аналітичного представлення даних}
\subsection{Засоби експорту даних (обмеження, формат файлу)}
\subsection{Засоби програмного інтерфейсу (якщо реалізовано)}
\newpage

\section{Висновки}
\subsection{Встановлені недоліки}
\subsection{Перспективи покращення}
\newpage

\section{Список літератури}
\newpage

\section{Додатки}
\subsection{Скрипт створення бази даних}
\subsection{Скрипт завантаження історичних даних}
\subsection{Інший програмний код}
\subsection{Інший код створених моделей}
\newpage

\end{document}
