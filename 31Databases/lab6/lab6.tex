\documentclass[14pt]{extreport}
\usepackage{cmap}
\usepackage[utf8]{inputenc}
\usepackage[english,ukrainian]{babel}
\usepackage{graphicx}
\usepackage{geometry}
\usepackage{listings}
\usepackage{amsmath}
\usepackage{float}
\geometry{
	a4paper,
	left=20mm,
	right=20mm,
	top=20mm,
	bottom=20mm
}
\lstset{
	columns=fullflexible, 
	frame=single, 
	breaklines=true, 
	tabsize=4,
	postbreak=\raisebox{0ex}[0ex][0ex]{\ensuremath{\hookrightarrow\space}},
	showstringspaces=false,% no symbol for spaces in strings
}
\graphicspath{ {./pictures} }
\setlength{\parindent}{4em}

\newcommand\subject{Бази даних}
\newcommand\lecturer{асистент кафедри ПЗ\\Білоіваненко М.В.}
\newcommand\teacher{асистент кафедри ПЗ\\Білоіваненко М.В.}
\newcommand\mygroup{ПЗ-32}
\newcommand\lab{6}
\newcommand\theme{Функції, тригери та процедури}
\newcommand\purpose{Розробити функції, що виконують обчислення даних по вхідним аргументам. Розробити процедури, що реалізують важливі правила оновлення даних предметної області}

\begin{document}
\begin{normalsize}
	\begin{titlepage}
		\thispagestyle{empty}
		\begin{center}
			\textbf{МІНІСТЕРСТВО ОСВІТИ І НАУКИ УКРАЇНИ\\
				НАЦІОНАЛЬНИЙ УНІВЕРСИТЕТ "ЛЬВІВСЬКА ПОЛІТЕХНІКА"}
		\end{center}
		\begin{flushright}
			Інститут \textbf{КНІТ}\\
			Кафедра \textbf{ПЗ}
		\end{flushright}
		\vspace{200pt}
		\begin{center}
			\textbf{ЗВІТ}\\
			\vspace{10pt}
			До лабораторної роботи № \lab\\
			\textbf{На тему}: “\textit{\theme}”\\
			\textbf{З дисципліни}: “\subject”
		\end{center}
		\vspace{40pt}
		\begin{flushright}
			
			\textbf{Лектор}:\\
			\lecturer\\
			\vspace{10pt}
			\textbf{Виконав}:\\
			
			студент групи \mygroup\\
			Коваленко Д.М.\\
			\vspace{10pt}
			\textbf{Прийняв}:\\
			
			\teacher\\
			
			\vspace{28pt}
			«\rule{1cm}{0.15mm}» \rule{1.5cm}{0.15mm} 2023 р.\\
			$\sum$ = \rule{1cm}{0.15mm}……………\\
			
		\end{flushright}
		\vspace{\fill}
		\begin{center}
			\textbf{Львів — 2023}
		\end{center}
	\end{titlepage}
		
	\begin{description}
		\item[Тема.] \theme.
		\item[Мета.] \purpose.
	\end{description}

	\section*{Лабораторне завдання}
	Розробити не менше 2 функцій, що виконують обчислення даних по вхідним аргументам. Розробити не менше 3 процедур, що реалізують важливі правила оновлення даних предметної області. Опис призначення процедур та зміст їх параметрів зафіксувати текстом. Деякі процедури використати як тригери (TRIGGER) до таблиць бази даних
	
	Код процедури повинен містити:
	\begin{enumerate}
		\item курсор та ітератор по рядкам
		\item умовні оператори, базові функції 
		\item команди вибірки, оновлення таблиць
	\end{enumerate}
	
	\section*{Хід роботи}
	
	Тригер для валідації географічних координат
	
	\begin{small}
		\begin{lstlisting}[language=sql]
create or replace procedure validate_coords(
	in lat decimal(8,2),
	in lon decimal(9,2)
)
language plpgsql
as $$
begin
	begin
		if lat < -90 or lat > 90 then
			raise exception 'latitude is invalid';
		end if;
		if lon < -180 or lon > 180 then
			raise exception 'longitude is invalid';
		end if;
	end;
end;
$$;

create or replace function tr_validate_coords()
returns trigger as $$
begin
	perform validate_coords(new.lat, new.lon);
	return new;
end;
$$ language plpgsql;

create or replace trigger stop_trigger
before insert or update on stop
for each row
execute function tr_validate_coords();
		\end{lstlisting}
	\end{small}
	
	Функція для знаходження наступної зупинки на маршруті.
	
	\begin{small}
		\begin{lstlisting}[language=sql]
create or replace function get_next_stop(
	route_id_param integer,
	stop_id_param integer
)
returns integer as $$
declare
	next_stop_id integer;
begin
	select s.id
	from route_stop rs
	join stop s on rs.stop_id = s.id
	where rs.route_id = route_id_param and rs.ord_index > (
		select ord_index 
		from route_stop 
		where stop_id = stop_id_param 
		and rs.route_id = route_id_param
	)
	order by rs.ord_index
	limit 1;
	
	return next_stop_id;
end;
$$ language plpgsql;
		\end{lstlisting}
	\end{small}
	
	Процедура, що оновлює наступну зупинку для транспортних засобів, якщо їх географічні координати дорівнюють географічним координатам попередньої зупинки.
	\begin{small}
		\begin{lstlisting}[language=sql]
create or replace procedure update_next_stop_on_coords_change(
	inout new_row route_vehicle,
	in old_row route_vehicle
)
as $$
declare
	current_lat numeric;
	current_lon numeric;
begin
	current_lat := new_row.last_lat;
	current_lon := new_row.last_lon;

	if current_lat = old_row.last_lat and current_lon = old_row.last_lon then
	new_row.next_stop := get_next_stop(new_row.route_id, new_row.vehicle_id);
	end if;
end;
$$ language plpgsql;
		\end{lstlisting}
	\end{small}
	
	Функція для обчислення часу у хвилинах до прибуття транспортного засобу на вказану зупинку
	\begin{small}
		\begin{lstlisting}[language=sql]
create or replace function calculate_time_to_arrival(stop_id_param integer)
returns integer as
$$
declare
	curr_time timestamp;
	arrival_time timestamp;
	time_to_arrival_minutes integer := -1;
	
	schedule_cursor cursor for
		select s.arrival_time
		from schedule s
		where stop_id = stop_id_param
		order by s.arrival_time;
begin
	curr_time := now();
	open schedule_cursor;
	fetch schedule_cursor into arrival_time;
	if arrival_time is not null then
		time_to_arrival_minutes := extract(epoch from (arrival_time - curr_time))/60;
	end if;
	close schedule_cursor;
	return time_to_arrival_minutes;
end;
$$ language plpgsql;
		\end{lstlisting}
	\end{small}
	
	Процедура для збільшення на заробітної плати водіям  10 \%, що мають позитивні відгуки.
	\begin{small}
		\begin{lstlisting}[language=sql]
create or replace procedure update_driver_salary()
language plpgsql
as $$
declare
	driver_record driver%rowtype;
	feedback_message text;
begin
	for driver_record in select * from driver
	loop
		select message into feedback_message
		from feedback
		where driver_name = driver_record.first_name || ' ' || driver_record.last_name
		and message ilike '%positive%';
		
		if feedback_message is not null then
			update driver
			set salary = salary * 1.1
			where id = driver_record.id;
		end if;
	end loop;
end;
$$;
		\end{lstlisting}
	\end{small}
	
	\section*{Висновок}
	Під час виконання лабораторної роботи я розробив 2 функції, що виконують обчислення даних по вхідним аргументам. Розробив 3 процедури, що реалізують важливі правила оновлення даних предметної області. Описав призначення процедур та зміст їх параметрів. Деякі процедури використав як тригери (TRIGGER) до таблиць бази даних.
	 
\end{normalsize}
\end{document}
