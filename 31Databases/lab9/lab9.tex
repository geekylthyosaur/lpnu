\documentclass[14pt]{extreport}
\usepackage{cmap}
\usepackage[utf8]{inputenc}
\usepackage[english,ukrainian]{babel}
\usepackage{graphicx}
\usepackage{geometry}
\usepackage{listings}
\usepackage{amsmath}
\usepackage{float}
\geometry{
	a4paper,
	left=20mm,
	right=20mm,
	top=20mm,
	bottom=20mm
}
\lstset{
	columns=fullflexible, 
	frame=single, 
	breaklines=true, 
	tabsize=4,
	postbreak=\raisebox{0ex}[0ex][0ex]{\ensuremath{\hookrightarrow\space}},
	showstringspaces=false,% no symbol for spaces in strings
}
\graphicspath{ {./pictures} }
\setlength{\parindent}{4em}

\newcommand\subject{Бази даних}
\newcommand\lecturer{асистент кафедри ПЗ\\Білоіваненко М.В.}
\newcommand\teacher{асистент кафедри ПЗ\\Білоіваненко М.В.}
\newcommand\mygroup{ПЗ-32}
\newcommand\lab{9}
\newcommand\theme{Виправлення недоліків}
\newcommand\purpose{Проаналізувати розроблену схему бази даних стосовно її відповідності нормальним формам та з метою підвищення ефективності доступу. Встановити структурні недоліки існуючого опису таблиць (зайві ключі, надлишкові дані, некоректні типи або назви колонок, неповнота встановлених обмежень цілісності, невідповідність особливостям даних предметної області)}

\begin{document}
\begin{normalsize}
	\begin{titlepage}
		\thispagestyle{empty}
		\begin{center}
			\textbf{МІНІСТЕРСТВО ОСВІТИ І НАУКИ УКРАЇНИ\\
				НАЦІОНАЛЬНИЙ УНІВЕРСИТЕТ "ЛЬВІВСЬКА ПОЛІТЕХНІКА"}
		\end{center}
		\begin{flushright}
			Інститут \textbf{КНІТ}\\
			Кафедра \textbf{ПЗ}
		\end{flushright}
		\vspace{200pt}
		\begin{center}
			\textbf{ЗВІТ}\\
			\vspace{10pt}
			До лабораторної роботи № \lab\\
			\textbf{На тему}: “\textit{\theme}”\\
			\textbf{З дисципліни}: “\subject”
		\end{center}
		\vspace{40pt}
		\begin{flushright}
			
			\textbf{Лектор}:\\
			\lecturer\\
			\vspace{10pt}
			\textbf{Виконав}:\\
			
			студент групи \mygroup\\
			Коваленко Д.М.\\
			\vspace{10pt}
			\textbf{Прийняв}:\\
			
			\teacher\\
			
			\vspace{28pt}
			«\rule{1cm}{0.15mm}» \rule{1.5cm}{0.15mm} 2023 р.\\
			$\sum$ = \rule{1cm}{0.15mm}……………\\
			
		\end{flushright}
		\vspace{\fill}
		\begin{center}
			\textbf{Львів — 2023}
		\end{center}
	\end{titlepage}
		
	\begin{description}
		\item[Тема.] \theme.
		\item[Мета.] \purpose.
	\end{description}

	\section*{Лабораторне завдання}


Проаналізувати розроблену схему бази даних стосовно її відповідності нормальним формам та з метою підвищення ефективності доступу. Встановити структурні недоліки існуючого опису таблиць (зайві ключі, надлишкові дані, некоректні типи або назви колонок, неповнота встановлених обмежень цілісності, невідповідність особливостям даних предметної області).

На 4 бали: детально описати якнайменш 3 важливих недоліки схеми та запропонувати обґрунтовані шляхи їх усунення. Використати підходи декомпозиції, денормалізації або інші варіанти зміни таблиць.

На 7 балів: створити послідовний набір SQL-команд, що виправляє в поточній базі даних встановлені структурні недоліки. При модифікації таблиць (застосувати ALTER, DROP, CREATE) слід врахувати існуючі дані, які частково повинні бути трансформовані для зберігання в оновленій схемі (застосувати SELECT, UPDATE, INSERT, DELETE).
	
	\section*{Хід роботи}
	
	\subsection*{Надлишкова інформація у route\_vehicle}
	\textbf{Проблема}: В таблиці route\_vehicle зберігається зайве повторюючеся інформація, яка може призвести до неоднозначностей та збільшення обсягу даних.
	
	\textbf{Опис змін}: Винести last\_lat, last\_lon, last\_coords\_time, next\_stop в окрему таблицю route\_vehicle\_history. Зберегти існуючі дані в тимчасову таблицю, виконати модифікації та повернути дані назад у відповідні таблиці.
	\begin{lstlisting}
CREATE TABLE route_vehicle_temp AS
SELECT id, route_id, vehicle_id
FROM route_vehicle;

ALTER TABLE route_vehicle
DROP COLUMN last_lat,
DROP COLUMN last_lon,
DROP COLUMN last_coords_time,
DROP COLUMN next_stop;

INSERT INTO route_vehicle (id, route_id, vehicle_id)
SELECT id, route_id, vehicle_id
FROM route_vehicle_temp;

DROP TABLE route_vehicle_temp;

CREATE TABLE route_vehicle_history (
id               SERIAL PRIMARY KEY,
route_vehicle_id INTEGER REFERENCES route_vehicle ON UPDATE CASCADE ON DELETE CASCADE,
lat              NUMERIC(8, 6),
lon              NUMERIC(9, 6),
coords_time      TIMESTAMP WITH TIME ZONE
);
	\end{lstlisting}
	
	\subsection*{Неповні обмеження цілісності у vehicle}
	\textbf{Проблема}: В таблиці vehicle відсутній унікальний ключ для поля license\_plate, що може призвести до додавання дублікатів.
	
	\textbf{Опис змін}: Додано унікальний ключ для поля license\_plate, щоб гарантувати унікальність номерів автомобілів.
	\begin{lstlisting}
ALTER TABLE vehicle
	ADD CONSTRAINT vehicle_license_plate_unique UNIQUE (license_plate);
	\end{lstlisting}
	
	
	\subsection*{Недостатня кількість інформації у feedback}
	\textbf{Проблема}: В таблиці feedback відсутня інформація про маршрут, час та контактні дані, що може призвести до обмеженої можливості аналізу та відповідей.
	
	\textbf{Опис змін}: Додано нові стовпці для ідентифікації маршруту (route\_id - зовнішній ключ до таблиці route), контактного email (contact\_email). Залишено їх значення за замовчуванням як NULL.
	\begin{lstlisting}
ALTER TABLE feedback
	ADD COLUMN route_id INTEGER REFERENCES route(id) ON UPDATE CASCADE, 
	ADD COLUMN contact_email VARCHAR(255);
	\end{lstlisting}
	\section*{Висновок}
	Під час виконання лабораторної роботи я проаналізував розроблену схему бази даних стосовно її відповідності нормальним формам та з метою підвищення ефективності доступу. Встановив структурні недоліки існуючого опису таблиць (зайві ключі, надлишкові дані, некоректні типи або назви колонок, неповнота встановлених обмежень цілісності, невідповідність особливостям даних предметної області).
	 
\end{normalsize}
\end{document}
