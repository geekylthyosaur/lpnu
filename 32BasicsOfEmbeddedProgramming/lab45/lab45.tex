\documentclass[oneside,14pt]{extarticle}
\usepackage{cmap}
\usepackage[utf8]{inputenc}
\usepackage[english,ukrainian]{babel}
\usepackage{graphicx}
\usepackage{geometry}
\usepackage{listings}
\usepackage{float}
\usepackage{amsmath}
\usepackage{subfig}
\geometry{
	a4paper,
	left=20mm,
	right=20mm,
	top=15mm,
	bottom=15mm,
}
\lstset{
	language=c,
	tabsize=4,
	keepspaces,
	showstringspaces=false,
	frame=single,
	breaklines,
	language=C,
}
\graphicspath{ {./pictures} }
\setlength{\parindent}{4em}

\newcommand\subject{Основи програмування вбудованих систем}
\newcommand\lecturer{доцентка кафедри ПЗ\\Марусенкова Т.А.}
\newcommand\teacher{доцент кафедри ПЗ\\Крук О.Г.}
\newcommand\mygroup{ПЗ-32}
\newcommand\lab{4-5}
\newcommand\theme{Дослідження протоколу SPI на прикладі акселерометра у складі STM32F4-Discovery та LCD-дисплею}
\newcommand\purpose{Засвоїти принципи роботи інтерфейсу SPI та здобути практичні навички організації взаємодії мікроконтролера з периферійними пристроями через SPI за допомогою бібліотек CMSIS, SPL та HAL, акселерометра LIS302DL у складі оціночної плати stm32f4Discovery та дисплею моделі ET-NOKIA LCD 5110}

\begin{document}
\begin{normalsize}
	\begin{titlepage}
		\thispagestyle{empty}
		\begin{center}
			\textbf{МІНІСТЕРСТВО ОСВІТИ І НАУКИ УКРАЇНИ\\
				НАЦІОНАЛЬНИЙ УНІВЕРСИТЕТ "ЛЬВІВСЬКА ПОЛІТЕХНІКА"}
		\end{center}
		\begin{flushright}
			\textbf{ІКНІ}\\
			Кафедра \textbf{ПЗ}
		\end{flushright}
		\vspace{80pt}
		\begin{center}
			\textbf{ЗВІТ}\\
			\vspace{10pt}
			до лабораторних робіт № \lab\\
			\textbf{на тему}: <<\textit{\theme}>>\\
			\textbf{з дисципліни}: <<\subject>>
		\end{center}
		\vspace{80pt}
		\begin{flushright}
			
			\textbf{Лекторка}:\\
			\lecturer\\
			\vspace{28pt}
			\textbf{Виконав}:\\
			
			студент групи \mygroup\\
			Коваленко Д.М.\\
			\vspace{28pt}
			\textbf{Прийняв}:\\
			
			\teacher\\
			
			\vspace{28pt}
			«\rule{1cm}{0.15mm}» \rule{1.5cm}{0.15mm} 2024 р.\\
			$\sum$ = \rule{1cm}{0.15mm}……………\\
			
		\end{flushright}
		\vspace{\fill}
		\begin{center}
			\textbf{Львів — 2024}
		\end{center}
	\end{titlepage}
		
	\begin{description}
		\item[Тема.] \theme.
		\item[Мета.] \purpose.
	\end{description}

    \section*{Лабораторне завдання}
    \begin{enumerate}
        \item Ознайомитися з викладеним вище теоретичним матеріалом.
        \item Написати функцію ініціалізації GPIO та SPI для обміну даними з акселерометром за допомогою CMSIS SPI Driver, SPL або HAL. Студент з порядковим номером N  у журналі вибирає CMSIS SPI Driver, якщо N\%3 = 1, SPL – якщо N\%3 = 2, HAL – якщо N\%3 = 0.
        \item Розробити функції для обміну даними (читання та запису) за допомогою  CMSIS SPI Driver, SPL або HAL. Бібліотеку вибрати з тих же міркувань, що й у пункті 2.
        Зчитати модель акселерометра. За допомогою вікна Watch у середовищі Keil uVision перевірити покази акселерометра. Злегка обертаючи плату, приблизно оцінити правильність показів. Продемонструвати результати роботи.
        \item Додати у проект функції для ініціалізації дисплею ET-NOKIA LCD 5110. Визначити за наданими кодами, які з виводів GPIO задіяні для обміну даними з дисплеєм і за що відповідає кожен вивід.
        \item Підключити макетну плату з дисплеєм до визначених виводів (усі виводи дисплею промарковані). Вивід «Живлення» дисплею необхідно підключити до 3 В, як показано на рис. 9.
        \item Написати додаткові функції для роботи з дисплеєм у відповідності до індивідуального завдання.
        \item Підготувати та здати звіт про виконання лабораторної роботи. 
    \end{enumerate}

	\section*{Індивідуальне завдання}
	6. Написати функцію для відображення на дисплеї парних рядків зображення.

	\section*{Теоретичні відомості}
	6. Які є режими роботи SPI? Що означає CPOL? CPHA?
	
	SPI працює в чотирьох режимах, визначених комбінацією двох параметрів: CPOL (Polarity of the Clock) та CPHA (Phase of the Clock). CPOL визначає базовий рівень годинникового сигналу: 0 - низький, 1 - високий. CPHA визначає, коли дані мають бути зчитані: на першому (0) або другому (1) фронті сигналу годинника.
	
	\section*{Хід роботи}
	
	\subsection*{Код програми}
	Файл \textit{main.c}:
	{\small
		\begin{lstlisting}
#include "main.h"

extern SPI_HandleTypeDef SPI_Handle;
extern uint8_t LCD_Buffer[LCD_WIDTH*LCD_HEIGHT/8];

int main(void) {
	HAL_Init();
	GPIO_Init();
	SPI_Init();
	LCD_Init();
	LIS3DSH_Init(TM_LIS3DSH_Sensitivity_6G, TM_LIS3DSH_Filter_200Hz);
	
	LCD_ClearOddRows();
	LCD_Update();
	
	TM_LIS302DL_LIS3DSH_t axesData;
	
	while (1) {
		LIS3DSH_ReadAxes(&axesData);
	}
	
	return 0;
}

void SPI_Init(void) {
	__HAL_RCC_SPI3_CLK_ENABLE();

	SPI_Handle.Instance = SPI3;
	SPI_Handle.Init.Mode = SPI_MODE_MASTER;
	SPI_Handle.Init.Direction = SPI_DIRECTION_2LINES;
	SPI_Handle.Init.DataSize = SPI_DATASIZE_8BIT;
	SPI_Handle.Init.CLKPolarity = SPI_POLARITY_HIGH;
	SPI_Handle.Init.CLKPhase = SPI_PHASE_2EDGE;
	SPI_Handle.Init.NSS = SPI_NSS_SOFT;
	SPI_Handle.Init.BaudRatePrescaler = SPI_BAUDRATEPRESCALER_128;
	SPI_Handle.Init.FirstBit = SPI_FIRSTBIT_MSB;
	SPI_Handle.Init.TIMode = SPI_TIMODE_DISABLED;
	SPI_Handle.Init.CRCCalculation = SPI_CRCCALCULATION_DISABLED;
	SPI_Handle.Init.CRCPolynomial = 10;
	HAL_SPI_Init(&SPI_Handle);
}

void GPIO_Init(void) {
	GPIO_InitTypeDef GPIO_InitStruct = {0};

	__GPIOC_CLK_ENABLE();
	__GPIOE_CLK_ENABLE();

	GPIO_InitStruct.Pin = LCD_CS_PIN | LCD_RST_PIN | LCD_DC_PIN;
	GPIO_InitStruct.Mode = GPIO_MODE_OUTPUT_PP;
	GPIO_InitStruct.Pull = GPIO_NOPULL;
	GPIO_InitStruct.Speed = GPIO_SPEED_FREQ_LOW;
	HAL_GPIO_Init(LCD_PORT, &GPIO_InitStruct);
	
	memset(&GPIO_InitStruct, 0, sizeof(GPIO_InitTypeDef));
	GPIO_InitStruct.Pin = LIS3DSH_CS_PIN;
	GPIO_InitStruct.Mode = GPIO_MODE_OUTPUT_PP;
	GPIO_InitStruct.Pull = GPIO_NOPULL;
	GPIO_InitStruct.Speed = GPIO_SPEED_FREQ_LOW;
	HAL_GPIO_Init(LCD_PORT, &GPIO_InitStruct);
}

void LCD_Init(void) {
	LCD_RST0;
	Delay(10);
	LCD_RST1;
	LCD_Command(0x21);
	LCD_Command(0xC0);
	LCD_Command(0x07);
	LCD_Command(0x13);
	LCD_Command(0x20);
	LCD_Command(0x0C);
}

void LCD_Command(uint8_t cmd) {
	LCD_CS0;
	LCD_COMMAND;
	LCD_Write(cmd);
	LCD_CS1;
}

void LCD_Data(uint8_t data) {
	LCD_CS0;
	LCD_DATA;
	LCD_Write(data);
	LCD_CS1;
}

void LCD_Write(uint8_t byte) {
	HAL_SPI_Transmit(&SPI_Handle, &byte, 1, HAL_MAX_DELAY);
}

void LCD_Clear(void) {
	memset(LCD_Buffer, 0, sizeof(LCD_Buffer));
	LCD_Update();
}

void LCD_Update(void) {
	LCD_Goto(0, 0);
  for (int i = 0; i < (LCD_WIDTH * LCD_HEIGHT / 8); i++)
    LCD_Data(LCD_Buffer[i]);
}

void LCD_Goto(uint8_t x, uint8_t y) {
  LCD_Command(0x80 | x);
  LCD_Command(0x40 | y);
}

void LCD_ClearOddRows(void) {
	const int bytesPerRow = LCD_WIDTH / 8;
	const int totalRows = LCD_HEIGHT / 8;

	for (int row = 0; row < totalRows; row++) {
		if (row % 2 == 0) continue;

		int startIndex = row * bytesPerRow * 8;
		
		for (int i = 0; i < bytesPerRow * 8; i++) 
			LCD_Buffer[startIndex + i] = 0;
	}
}

void Delay(int ms) {
	SysTick->LOAD = 16000-1;
	SysTick->VAL = 0;
	SysTick->CTRL = 0x5;
		for (int i = 0; i < ms; i++)
			while (!(SysTick->CTRL & 0x10000));
	SysTick->CTRL=0;
}

void LIS3DSH_Read(uint8_t reg, uint8_t* val) {
	LIS3DSH_CS0;
	reg = reg | 0x80;
	HAL_SPI_TransmitReceive(&SPI_Handle, &reg, val, 1, 10000);
	reg = 0;
	HAL_SPI_TransmitReceive(&SPI_Handle, &reg, val, 1, 10000);
	LIS3DSH_CS1;
}

void LIS3DSH_Write(uint8_t reg, uint8_t val) {
  uint8_t data[1];
	LIS3DSH_CS0;
	HAL_SPI_TransmitReceive(&SPI_Handle, &reg, data, 1, 10000);
	HAL_SPI_TransmitReceive(&SPI_Handle, &val, data, 1, 10000);
	LIS3DSH_CS1;
}

void LIS3DSH_ReadAxes(TM_LIS302DL_LIS3DSH_t* axesData) {
	uint8_t buffer[6];

	LIS3DSH_Read(LIS3DSH_OUT_X_L_ADDR, &buffer[0]);
	LIS3DSH_Read(LIS3DSH_OUT_X_H_ADDR, &buffer[1]);
	LIS3DSH_Read(LIS3DSH_OUT_Y_L_ADDR, &buffer[2]);
	LIS3DSH_Read(LIS3DSH_OUT_Y_H_ADDR, &buffer[3]);
	LIS3DSH_Read(LIS3DSH_OUT_Z_L_ADDR, &buffer[4]);
	LIS3DSH_Read(LIS3DSH_OUT_Z_H_ADDR, &buffer[5]);
	
	axesData->X = (int16_t)((buffer[1] << 8) + buffer[0]) * 0.06;
	axesData->Y = (int16_t)((buffer[3] << 8) + buffer[2]) * 0.06;
	axesData->Z = (int16_t)((buffer[5] << 8) + buffer[4]) * 0.06;
}

void LIS3DSH_Init(TM_LIS302DL_LIS3DSH_Sensitivity_t Sensitivity, TM_LIS302DL_LIS3DSH_Filter_t Filter) {
	uint8_t tmpreg;
	uint16_t temp;

	temp = (uint16_t) (LIS3DSH_DATARATE_100 | LIS3DSH_XYZ_ENABLE);
	temp |= (uint16_t) (LIS3DSH_SERIALINTERFACE_4WIRE | LIS3DSH_SELFTEST_NORMAL);
	
	if (Sensitivity == TM_LIS3DSH_Sensitivity_2G) {
		temp |= (uint16_t) (LIS3DSH_FULLSCALE_2);
		LIS3DSH_Sensitivity = LIS3DSH_SENSITIVITY_0_06G;
	} else if (Sensitivity == TM_LIS3DSH_Sensitivity_4G) {
		temp |= (uint16_t) (LIS3DSH_FULLSCALE_4);
		LIS3DSH_Sensitivity = LIS3DSH_SENSITIVITY_0_12G;
	} else if (Sensitivity == TM_LIS3DSH_Sensitivity_6G) {
		temp |= (uint16_t) (LIS3DSH_FULLSCALE_6);
		LIS3DSH_Sensitivity = LIS3DSH_SENSITIVITY_0_18G;
	} else if (Sensitivity == TM_LIS3DSH_Sensitivity_8G) {
		temp |= (uint16_t) (LIS3DSH_FULLSCALE_8);
		LIS3DSH_Sensitivity = LIS3DSH_SENSITIVITY_0_24G;
	} else if (Sensitivity == TM_LIS3DSH_Sensitivity_16G) {
		temp |= (uint16_t) (LIS3DSH_FULLSCALE_16);
		LIS3DSH_Sensitivity = LIS3DSH_SENSITIVITY_0_73G;
	} else {
		return;
	}
	
	if (Filter == TM_LIS3DSH_Filter_800Hz) {
		temp |= (uint16_t) (LIS3DSH_FILTER_BW_800 << 8);
	} else if (Filter == TM_LIS3DSH_Filter_400Hz) {
		temp |= (uint16_t) (LIS3DSH_FILTER_BW_400 << 8);
	} else if (Filter == TM_LIS3DSH_Filter_200Hz) {
		temp |= (uint16_t) (LIS3DSH_FILTER_BW_200 << 8);
	} else if (Filter == TM_LIS3DSH_Filter_50Hz) {
		temp |= (uint16_t) (LIS3DSH_FILTER_BW_50 << 8);
	} else {
		return;
	}
	
	tmpreg = (uint8_t) (temp);
	LIS3DSH_Write(LIS3DSH_CTRL_REG4_ADDR, tmpreg);
	tmpreg = (uint8_t) (temp >> 8);
	LIS3DSH_Write(LIS3DSH_CTRL_REG5_ADDR, tmpreg);
}
		\end{lstlisting}
	}
	
	Файл \textit{main.h}:
	{\small
		\begin{lstlisting}
#include <stdbool.h>
#include <string.h>
#include <stm32f4xx_hal.h>
#include <stm32f4xx_hal_gpio.h>
#include <stm32f4xx_hal_spi.h>
#include <tm_stm32_lis302dl_lis3dsh.h>

#define LCD_PORT GPIOC

#define LCD_CS_PIN GPIO_PIN_7
#define LCD_CS0 HAL_GPIO_WritePin(LCD_PORT, LCD_CS_PIN, GPIO_PIN_RESET)
#define LCD_CS1 HAL_GPIO_WritePin(LCD_PORT, LCD_CS_PIN, GPIO_PIN_SET)

#define LCD_RST_PIN GPIO_PIN_8
#define LCD_RST0 HAL_GPIO_WritePin(LCD_PORT, LCD_RST_PIN, GPIO_PIN_RESET)
#define LCD_RST1 HAL_GPIO_WritePin(LCD_PORT, LCD_RST_PIN, GPIO_PIN_SET)

#define LCD_DC_PIN GPIO_PIN_9
#define LCD_COMMAND HAL_GPIO_WritePin(LCD_PORT, LCD_DC_PIN, GPIO_PIN_RESET)
#define LCD_DATA 		HAL_GPIO_WritePin(LCD_PORT, LCD_DC_PIN, GPIO_PIN_SET)

#define LIS3DSH_PORT GPIOE

#define LIS3DSH_CS_PIN GPIO_PIN_3
#define LIS3DSH_CS0 HAL_GPIO_WritePin(LIS3DSH_PORT, LIS3DSH_CS_PIN, GPIO_PIN_RESET)
#define LIS3DSH_CS1 HAL_GPIO_WritePin(LIS3DSH_PORT, LIS3DSH_CS_PIN, GPIO_PIN_SET)

#define LCD_WIDTH 84
#define LCD_HEIGHT 48

void GPIO_Init(void);
void SPI_Init(void);
void LCD_Init(void);
void LCD_Command(uint8_t cmd);
void LCD_Data(uint8_t data);
void LCD_Write(uint8_t byte);
void LCD_Clear(void);
void LCD_Update(void);
void LCD_Goto(uint8_t x, uint8_t y);
void LCD_ClearOddRows(void);

void LIS3DSH_Read(uint8_t reg, uint8_t* val);
void LIS3DSH_Write(uint8_t reg, uint8_t val);
void LIS3DSH_ReadAxes(TM_LIS302DL_LIS3DSH_t* axesData);
void LIS3DSH_Init(TM_LIS302DL_LIS3DSH_Sensitivity_t Sensitivity, TM_LIS302DL_LIS3DSH_Filter_t Filter);

void Delay(int ms);

static SPI_HandleTypeDef SPI_Handle;
static uint8_t LCD_Buffer[LCD_WIDTH*LCD_HEIGHT/8] = {
	0x00, 0x00, 0x00, 0x00, 0x00, 0x00, 0x00, 0x00, 0x00, 0x00, 0x00, 0x00, // (0,0)->(11,7)
  0x00, 0x00, 0x00, 0x00, 0x00, 0x00, 0x00, 0x00, 0x00, 0x00, 0x00, 0x00, // (12,0)->(23,7)
  0x00, 0x00, 0x00, 0x00, 0x00, 0x00, 0x00, 0x00, 0x00, 0x00, 0x00, 0xE0, // (24,0)->(35,7)
  0xF0, 0xF8, 0xFC, 0xFC, 0xFE, 0xFE, 0xFE, 0xFE, 0x1E, 0x0E, 0x02, 0x00, // (36,0)->(47,7)
  0x00, 0x00, 0x00, 0x00, 0x00, 0x00, 0x00, 0x00, 0x00, 0x00, 0x00, 0x00, // (48,0)->(59,7)
  0x00, 0x00, 0x00, 0x00, 0x00, 0x00, 0x00, 0x00, 0x00, 0x00, 0x00, 0x00, // (60,0)->(71,7)
  0x00, 0x00, 0x00, 0x00, 0x00, 0x00, 0x00, 0x00, 0x00, 0x00, 0x00, 0x00, // (72,0)->(83,7)
  0x00, 0x00, 0x00, 0x00, 0x00, 0x00, 0x00, 0x00, 0x00, 0x00, 0x00, 0x00, // (0,8)->(11,15)
  0x00, 0x00, 0x00, 0x00, 0x00, 0x00, 0x00, 0x00, 0x00, 0x00, 0x00, 0x00, // (12,8)->(23,15)
  0x00, 0x00, 0x00, 0x00, 0x00, 0x00, 0x00, 0x00, 0x00, 0x00, 0x00, 0x03, // (24,8)->(35,15)
  0x0F, 0x1F, 0x3F, 0x7F, 0xFF, 0xFF, 0xFF, 0xFF, 0xFF, 0xFE, 0xFC, 0xF8, // (36,8)->(47,15)
  0xF8, 0xF0, 0xF8, 0xFE, 0xFE, 0xFC, 0xF8, 0xE0, 0x00, 0x00, 0x00, 0x00, // (48,8)->(59,15)
  0x00, 0x00, 0x00, 0x00, 0x00, 0x00, 0x00, 0x00, 0x00, 0x00, 0x00, 0x00, // (60,8)->(71,15)
  0x00, 0x00, 0x00, 0x00, 0x00, 0x00, 0x00, 0x00, 0x00, 0x00, 0x00, 0x00, // (72,8)->(83,15)
  0x00, 0x00, 0x00, 0x00, 0x00, 0x00, 0x00, 0x00, 0x00, 0x00, 0x00, 0x00, // (0,16)->(11,23)
  0x00, 0x00, 0x00, 0x00, 0x00, 0x00, 0x00, 0x00, 0x00, 0x00, 0x00, 0x00, // (12,16)->(23,23)
  0x00, 0x00, 0xF8, 0xFC, 0xFE, 0xFE, 0xFF, 0xFF, 0xF3, 0xE0, 0xE0, 0xC0, // (24,16)->(35,23)
  0xC0, 0xC0, 0xE0, 0xE0, 0xF1, 0xFF, 0xFF, 0xFF, 0xFF, 0xFF, 0xFF, 0xFF, // (36,16)->(47,23)
  0xFF, 0xFF, 0xFF, 0xFF, 0xFF, 0xFF, 0xFF, 0xFF, 0x3E, 0x00, 0x00, 0x00, // (48,16)->(59,23)
  0x00, 0x00, 0x00, 0x00, 0x00, 0x00, 0x00, 0x00, 0x00, 0x00, 0x00, 0x00, // (60,16)->(71,23)
  0x00, 0x00, 0x00, 0x00, 0x00, 0x00, 0x00, 0x00, 0x00, 0x00, 0x00, 0x00, // (72,16)->(83,23)
  0x00, 0x00, 0x00, 0x00, 0x00, 0x00, 0x00, 0x00, 0x00, 0x00, 0x00, 0x00, // (0,24)->(11,31)
  0x00, 0x00, 0x00, 0x00, 0x00, 0x00, 0x00, 0x00, 0x00, 0x00, 0x00, 0x00, // (12,24)->(23,31)
  0x00, 0x00, 0xFF, 0xFF, 0xFF, 0xFF, 0xFF, 0xFF, 0xFF, 0xFF, 0xFF, 0xFF, // (24,24)->(35,31)
  0xFF, 0xFF, 0xFF, 0xFF, 0xFF, 0xFF, 0xFF, 0xFF, 0xFF, 0xFF, 0xFF, 0xFF, // (36,24)->(47,31)
  0xFF, 0xFF, 0xFF, 0x7F, 0x3F, 0x1F, 0x07, 0x01, 0x00, 0x00, 0x00, 0x00, // (48,24)->(59,31)
  0x00, 0x00, 0x00, 0x00, 0x00, 0x00, 0x00, 0x00, 0x00, 0x00, 0x00, 0x00, // (60,24)->(71,31)
  0x00, 0x00, 0x00, 0x00, 0x00, 0x00, 0x00, 0x00, 0x00, 0x00, 0x00, 0x00, // (72,24)->(83,31)
  0x00, 0x00, 0x00, 0x00, 0x00, 0x00, 0x00, 0x00, 0x00, 0x00, 0x00, 0x00, // (0,32)->(11,39)
  0x00, 0x00, 0x00, 0x00, 0x00, 0x00, 0x00, 0x00, 0x00, 0x00, 0x00, 0x00, // (12,32)->(23,39)
  0x00, 0x00, 0xFF, 0xFF, 0xFF, 0xFF, 0xFF, 0xFF, 0xFF, 0x7F, 0x3F, 0x1F, // (24,32)->(35,39)
  0x0F, 0x0F, 0x0F, 0x07, 0x07, 0x07, 0x07, 0x07, 0x07, 0x07, 0x03, 0x03, // (36,32)->(47,39)
  0x01, 0x01, 0x00, 0x00, 0x00, 0x00, 0x00, 0x00, 0x00, 0x00, 0x00, 0x00, // (48,32)->(59,39)
  0x00, 0x00, 0x00, 0x00, 0x00, 0x00, 0x00, 0x00, 0x00, 0x00, 0x00, 0x00, // (60,32)->(71,39)
  0x00, 0x00, 0x00, 0x00, 0x00, 0x00, 0x00, 0x00, 0x00, 0x00, 0x00, 0x00, // (72,32)->(83,39)
  0x00, 0x00, 0x00, 0x00, 0x00, 0x00, 0x00, 0x00, 0x00, 0x00, 0x00, 0x00, // (0,40)->(11,47)
  0x00, 0x00, 0x00, 0x00, 0x00, 0x00, 0x00, 0x00, 0x00, 0x00, 0x00, 0x00, // (12,40)->(23,47)
  0x00, 0x00, 0x3F, 0x1F, 0x0F, 0x07, 0x03, 0x01, 0x00, 0x00, 0x00, 0x00, // (24,40)->(35,47)
  0x00, 0x00, 0x00, 0x00, 0x00, 0x00, 0x00, 0x00, 0x00, 0x00, 0x00, 0x00, // (36,40)->(47,47)
  0x00, 0x00, 0x00, 0x00, 0x00, 0x00, 0x00, 0x00, 0x00, 0x00, 0x00, 0x00, // (48,40)->(59,47)
  0x00, 0x00, 0x00, 0x00, 0x00, 0x00, 0x00, 0x00, 0x00, 0x00, 0x00, 0x00, // (60,40)->(71,47)
  0x00, 0x00, 0x00, 0x00, 0x00, 0x00, 0x00, 0x00, 0x00, 0x00, 0x00, 0x00, // (72,40)->(83,47)
};

static float LIS3DSH_Sensitivity;
		\end{lstlisting}
	}
	
	\section*{Висновки}
	Під час виконання лабораторної роботи я засвоїв принципи роботи інтерфейсу SPI та здобув практичні навички організації взаємодії мікроконтролера з периферійними пристроями через SPI за допомогою бібліотек CMSIS, SPL та HAL, акселерометра LIS302DL у складі оціночної плати stm32f4Discovery та дисплею моделі ET-NOKIA LCD 5110. Розробив функцію для відображення на дисплеї парних рядків зображення.
	    
\end{normalsize}
\end{document}
