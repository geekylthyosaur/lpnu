\documentclass[oneside,14pt]{extarticle}
\usepackage{cmap}
\usepackage[utf8]{inputenc}
\usepackage[english,ukrainian]{babel}
\usepackage{graphicx}
\usepackage{geometry}
\usepackage{listings}
\usepackage{float}
\usepackage{amsmath}
\usepackage{subfig}
\geometry{
	a4paper,
	left=20mm,
	right=20mm,
	top=15mm,
	bottom=15mm,
}
\lstset{
	language=c,
	tabsize=4,
	keepspaces,
	showstringspaces=false,
	frame=single,
	breaklines,
	language=C,
}
\graphicspath{ {./pictures} }
\setlength{\parindent}{4em}

\newcommand\subject{Основи програмування вбудованих систем}
\newcommand\lecturer{доцентка кафедри ПЗ\\Марусенкова Т.А.}
\newcommand\teacher{доцент кафедри ПЗ\\Крук О.Г.}
\newcommand\mygroup{ПЗ-32}
\newcommand\lab{6}
\newcommand\theme{Робота з АЦП}
\newcommand\purpose{Ознайомитися на практиці з параметрами АЦП у Keil uVision}

\begin{document}
\begin{normalsize}
	\begin{titlepage}
		\thispagestyle{empty}
		\begin{center}
			\textbf{МІНІСТЕРСТВО ОСВІТИ І НАУКИ УКРАЇНИ\\
				НАЦІОНАЛЬНИЙ УНІВЕРСИТЕТ "ЛЬВІВСЬКА ПОЛІТЕХНІКА"}
		\end{center}
		\begin{flushright}
			\textbf{ІКНІ}\\
			Кафедра \textbf{ПЗ}
		\end{flushright}
		\vspace{80pt}
		\begin{center}
			\textbf{ЗВІТ}\\
			\vspace{10pt}
			до лабораторної роботи № \lab\\
			\textbf{на тему}: <<\textit{\theme}>>\\
			\textbf{з дисципліни}: <<\subject>>
		\end{center}
		\vspace{80pt}
		\begin{flushright}
			
			\textbf{Лекторка}:\\
			\lecturer\\
			\vspace{28pt}
			\textbf{Виконав}:\\
			
			студент групи \mygroup\\
			Коваленко Д.М.\\
			\vspace{28pt}
			\textbf{Прийняв}:\\
			
			\teacher\\
			
			\vspace{28pt}
			«\rule{1cm}{0.15mm}» \rule{1.5cm}{0.15mm} 2024 р.\\
			$\sum$ = \rule{1cm}{0.15mm}……………\\
			
		\end{flushright}
		\vspace{\fill}
		\begin{center}
			\textbf{Львів — 2024}
		\end{center}
	\end{titlepage}
		
	\begin{description}
		\item[Тема.] \theme.
		\item[Мета.] \purpose.
	\end{description}

    \section*{Лабораторне завдання}
    \begin{enumerate}
        \item Створити у Keil uVision проект, в якому при температурі 15С і вище засвічується червоний світлодіод, інакше – синій. Створити різні температурні умови і перевірити роботу проекту.
        \item З’ясувати значення кожного рядка коду у проекті. 
    \end{enumerate}

	\section*{Хід роботи}
	
	\subsection*{Код програми}
	Файл \textit{main.c}:
	{\small
		\begin{lstlisting}
#include "stm32f4xx_rcc.h"
#include "stm32f4xx_adc.h"

void gpio_init(void);
void adc_init(void); 
uint16_t readADC1(uint8_t channel); 

int main(void) {
	unsigned int result; 
	adc_init(); 
	
	result = readADC1(ADC_Channel_1); 
	if(((result - 0.76)/0.0025 + 25)<15){ 
		GPIOD->ODR = 0x4000; 
	} 
	else { 
		GPIOD->ODR = 0x8000; 
	} 
		
	while (1); 
		
	return 0;
}

void gpio_init(void){ 
	RCC->AHB1ENR |= RCC_AHB1ENR_GPIODEN; 
	GPIOD->MODER = 0x55000000; 
	GPIOD->OTYPER = 0;
	GPIOD->OSPEEDR = 0; 
}

void adc_init() { 
	ADC_InitTypeDef ADC_InitStructure; 
	ADC_CommonInitTypeDef adc_init; 
	RCC_APB2PeriphClockCmd(RCC_APB2Periph_ADC1, ENABLE); 
	ADC_DeInit(); 
	ADC_StructInit(&ADC_InitStructure); 
	adc_init.ADC_Mode = ADC_Mode_Independent; 
	adc_init.ADC_Prescaler = ADC_Prescaler_Div2; 
	ADC_InitStructure.ADC_ScanConvMode = DISABLE; 
	ADC_InitStructure.ADC_ContinuousConvMode = DISABLE; 
	ADC_InitStructure.ADC_ExternalTrigConv = ADC_ExternalTrigConvEdge_None;
	ADC_InitStructure.ADC_DataAlign = ADC_DataAlign_Right; 
	ADC_InitStructure.ADC_Resolution = ADC_Resolution_12b; 
	ADC_CommonInit(&adc_init); 
	ADC_Init(ADC1, &ADC_InitStructure); 
	ADC_Cmd(ADC1, ENABLE); 
} 

uint16_t readADC1(uint8_t channel) { 
	ADC_RegularChannelConfig(ADC1, channel, 1, ADC_SampleTime_3Cycles);
	ADC_SoftwareStartConv(ADC1); 
	while (ADC_GetFlagStatus(ADC1, ADC_FLAG_EOC) == RESET); 
	return ADC_GetConversionValue(ADC1); 
} 

		\end{lstlisting}
	}
	
	\section*{Висновки}
	Під час виконання лабораторної роботи я ознайомився на практиці з параметрами АЦП у Keil uVision. Розробив проект в якому при температурі 15С і вище засвічується червоний світлодіод, інакше – синій. Створив різні температурні умови і перевірив роботу проекту.
	    
\end{normalsize}
\end{document}
