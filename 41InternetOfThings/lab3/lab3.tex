\documentclass[oneside,14pt]{extarticle}
\usepackage{cmap}
\usepackage[utf8]{inputenc}
\usepackage[english,ukrainian]{babel}
\usepackage{graphicx}
\usepackage{geometry}
\usepackage[labelsep=period]{caption}
\usepackage{indentfirst}
\usepackage{listings}
\usepackage{float}
\usepackage{amsmath}
\usepackage{subfig}
\usepackage{tempora}
\geometry{
	a4paper,
	left=20mm,
	right=20mm,
	top=15mm,
	bottom=15mm,
}
\lstset{
	language=c,
	tabsize=4,
	keepspaces,
	showstringspaces=false,
	frame=single,
	breaklines,
	language=C,
}
\graphicspath{ {./pictures} }
\setlength{\parindent}{4em}

\newcommand\subject{Основи інтернету речей}
\newcommand\lecturer{професор кафедри ПЗ\\Фечан А.В.}
\newcommand\teacher{асистент кафедри ПЗ\\Далявський В.С.}
\newcommand\mygroup{ПЗ-42}
\newcommand\lab{3}
\newcommand\theme{Дослідження рідкокристалічного дисплею LCD1602 за допомогою
	середовища STM32CubeIDE та плати STM32F4 Discovery}
\newcommand\purpose{Дослідити роботу пристроїв відображення інформації на
	прикладі символьного рідкокристалічного дисплею}

\begin{document}
\begin{normalsize}
	\begin{titlepage}
		\thispagestyle{empty}
		\begin{center}
			\textbf{МІНІСТЕРСТВО ОСВІТИ І НАУКИ УКРАЇНИ\\
				НАЦІОНАЛЬНИЙ УНІВЕРСИТЕТ "ЛЬВІВСЬКА ПОЛІТЕХНІКА"}
		\end{center}
		\begin{flushright}
			\textbf{ІКНІ}\\
			Кафедра \textbf{ПЗ}
		\end{flushright}
		\vspace{80pt}
		\begin{center}
			\textbf{ЗВІТ}\\
			\vspace{10pt}
			до лабораторної роботи № \lab\\
			\textbf{на тему}: <<\textit{\theme}>>\\
			\textbf{з дисципліни}: <<\subject>>
		\end{center}
		\vspace{80pt}
		\begin{flushright}
			
			\textbf{Лектор}:\\
			\lecturer\\
			\vspace{28pt}
			\textbf{Виконав}:\\
			
			студент групи \mygroup\\
			Коваленко Д.М.\\
			\vspace{28pt}
			\textbf{Прийняв}:\\
			
			\teacher\\
			
			\vspace{28pt}
			«\rule{1cm}{0.15mm}» \rule{1.5cm}{0.15mm} 2024 р.\\
			$\sum$ = \rule{1cm}{0.15mm}……………\\
			
		\end{flushright}
		\vspace{\fill}
		\begin{center}
			\textbf{Львів – 2024}
		\end{center}
	\end{titlepage}
		
	\begin{description}
		\item[Тема.] \theme.
		\item[Мета.] \purpose.
	\end{description}

    \section*{\hfil Лабораторне завдання\hfil}
	\begin{enumerate}
		\item Створити проект згідно індивідуального завдання для борду STM32F4
		Discovery.
		\item Створити проект згідно індивідуального завдання для мікроконтролера
		STM32F407VG.
		\item Виконати індивідуальне завдання.
	\end{enumerate}

	\section*{\hfil Хід роботи\hfil}
	
	\begin{figure}[H]
		\centering
		\includegraphics[width=\columnwidth]{1}
		\caption{Налаштування пінів на вихід}
	\end{figure}
	
	\textbf{Код програми:}
	\lstinputlisting{project/Core/Src/display.h}
	
	\section*{\hfil Висновки\hfil}
	Під час виконання лабораторної роботи, я дослідив роботу пристроїв відображення інформації на
	прикладі символьного рідкокристалічного дисплею.
		    
\end{normalsize}
\end{document}
