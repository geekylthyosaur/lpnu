\documentclass[oneside,14pt]{extarticle}
\usepackage{cmap}
\usepackage[utf8]{inputenc}
\usepackage{longtable}
\usepackage[english,ukrainian]{babel}
\usepackage{graphicx}
\usepackage{geometry}
\usepackage{listings}
\usepackage{float}
\usepackage{amsmath}
\usepackage{subfig}
\usepackage{tempora}
\renewcommand{\arraystretch}{1.5}
\geometry{
	a4paper,
	left=20mm,
	right=20mm,
	top=15mm,
	bottom=15mm,
}
\lstset{
	language=c,
	tabsize=4,
	keepspaces,
	showstringspaces=false,
	frame=single,
	breaklines,
	language=C,
}
\graphicspath{ {./pictures} }
\setlength{\parindent}{4em}

\newcommand\subject{Якість програмного забезпечення та тестування}
\newcommand\lecturer{доцент кафедри ПЗ\\Фоменко А.В.}
\newcommand\teacher{асистент кафедри ПЗ\\Джумеля Е.А.}
\newcommand\mygroup{ПЗ-42}
\newcommand\lab{1}
\newcommand\theme{Види тестування. Планування тестування}
\newcommand\purpose{Вивчити класифікацію видів тестування, розробити перевірки для
	різних видів тестування, навчитися планувати тестові активності в залежності
	від особливостей продукції, що поставляється на тестування
	функціональності}

\begin{document}
\begin{normalsize}
	\begin{titlepage}
		\thispagestyle{empty}
		\begin{center}
			\textbf{МІНІСТЕРСТВО ОСВІТИ І НАУКИ УКРАЇНИ\\
				НАЦІОНАЛЬНИЙ УНІВЕРСИТЕТ "ЛЬВІВСЬКА ПОЛІТЕХНІКА"}
		\end{center}
		\begin{flushright}
			\textbf{ІКНІ}\\
			Кафедра \textbf{ПЗ}
		\end{flushright}
		\vspace{80pt}
		\begin{center}
			\textbf{ЗВІТ}\\
			\vspace{10pt}
			до лабораторної роботи № \lab\\
			\textbf{на тему}: <<\textit{\theme}>>\\
			\textbf{з дисципліни}: <<\subject>>
		\end{center}
		\vspace{80pt}
		\begin{flushright}
			
			\textbf{Лектор}:\\
			\lecturer\\
			\vspace{28pt}
			\textbf{Виконав}:\\
			
			студент групи \mygroup\\
			Коваленко Д.М.\\
			\vspace{28pt}
			\textbf{Прийняла}:\\
			
			\teacher\\
			
			\vspace{28pt}
			«\rule{1cm}{0.15mm}» \rule{1.5cm}{0.15mm} 2024 р.\\
			$\sum$ = \rule{1cm}{0.15mm}……………\\
			
		\end{flushright}
		\vspace{\fill}
		\begin{center}
			\textbf{Львів — 2024}
		\end{center}
	\end{titlepage}
		
	\begin{description}
		\item[Тема.] \theme.
		\item[Мета.] \purpose.
	\end{description}

    \section*{Лабораторне завдання}
    \begin{enumerate}
    	\item Обрати будь-якій об’єкт реального світу і створити інтуїтивний план
    	його тестування за зразком.
    	\item Виберіть свій модуль, якій розробляється в АППЗ з метою подальшої
    	розробки тестових перевірок для нього.
    	\item Розробити різні перевірки відповідно до класифікації видів тестування
    	для вибраного об'єкта реального світу.
    	\item Розробити композицію тестів для першої поставки програмного
    	забезпечення (build 1), що складається з трьох модулів (модуль 1, модуль 2,
    	модуль 3).
    	\item Розробити композицію тестів для другої поставки програмного
    	забезпечення (build 2): виправлені заведені дефекти, доставлена нова
    	функціональність - модуль 4.
    	\item Розробити композицію тестів для третьої поставки програмного
    	забезпечення (build 3): замовник вирішив розширювати ринки збуту і просить
    	здійснити підтримку програмного забезпечення на англійській мові.
    	\item Розробити композицію тестів для четвертої поставки програмного
    	забезпечення (build 4): замовник хоче переконатися, що програмне
    	забезпечення витримає навантаження в 2000 користувачів.
    	\item Оформити звіт і захистити лабораторну роботу.
    \end{enumerate}
	\section*{Хід роботи}
	
	Об'єкт тестування: годинник.
	
	\begin{longtable}{|p{4.5cm}|p{5cm}|p{7cm}|}
		\hline
		\textbf{Вид тестування} & \textbf{Короткий опис виду тестування} & \textbf{Тестові перевірки} \\
		\hline
		\textbf{Functional Testing} & Тестування функціональності механізму згідно з його специфікацією & Перевірка точності відображення часу; перевірка чи заводить годинник пружину і чи рухаються стрілки після заведення. \\
		\hline
		\textbf{Safety Testing} & Перевірка здатності годинника залишатися безпечним для користувача & Перевірка чи механізм безпечний для шкіри (немає гострих частин), чи годинник не розбивається при легкому падінні. \\
		\hline
		\textbf{Security Testing} & Тестування стійкості до зовнішніх впливів і довговічності & Перевірка водонепроникності годинника; стійкість до перепадів температур і фізичних пошкоджень. \\
		\hline
		\textbf{Compatibility Testing} & Перевірка чи зручний годинник для різних типів зап’ястя і рук & Перевірка чи регулюється ремінець для різних розмірів руки, чи підходить для використання як чоловіками, так і жінками. \\
		\hline
		\textbf{GUI Testing} & Тестування, що виконується шляхом взаємодії з системою через графічний інтерфейс користувача & Перевірка наскільки чітко і зрозуміло відображається час на циферблаті; перевірка зручності керування кнопками на корпусі годинника. \\
		\hline
		\textbf{Usability Testing} & Тестування зручності використання годинника користувачем & Перевірка наскільки легко заводити годинник, регулювати час, надягати на руку та знімати. \\
		\hline
		\textbf{Accessibility Testing} & Тестування, яке визначає ступінь легкості, з якою користувачі з обмеженими здібностями можуть використовувати годинник & Перевірка чи годинник зручний для використання особами з обмеженими можливостями (наприклад, візуально ослабленими). \\
		\hline
		\textbf{Internationalization Testing} & Тестування адаптації продукту до мовних і культурних особливостей регіонів, де він буде використовуватися & Перевірка чи годинник не має символів або написів, що можуть бути неприйнятними в певних культурних або мовних контекстах. \\
		\hline
		\textbf{Performance Testing} & Тестування продуктивності і точності годинника & Перевірка наскільки довго годинник може працювати після одного заводження, а також точність ходу протягом доби. \\
		\hline
		\textbf{Stress Testing} & Тестування годинника при екстремальних умовах & Перевірка роботи годинника при сильних ударах, різких перепадах температур або при зануренні у воду. \\
		\hline
		\textbf{Negative Testing} & Тестування роботи годинника при некоректних діях & Перевірка як працює годинник при неправильному заводженні або при надмірному тиску на заводний механізм. \\
		\hline
		\textbf{Black Box Testing} & Тестування без знання внутрішньої структури механізму & Перевірка чи правильно показує час, не дивлячись на те, як влаштований механізм. \\
		\hline
		\textbf{Automated Testing} & Автоматизовані тести з метою виключити людський фактор & Автоматизоване заводження годинника кілька разів і перевірка точності ходу після кожного циклу. \\
		\hline
		\textbf{Unit/Component Testing} & Тестування окремих частин механізму & Перевірка точності ходу головної пружини; перевірка роботи механізму стрілок; перевірка заводного механізму на зносостійкість. \\
		\hline
		\textbf{Integration Testing} & Тестування взаємодії всіх частин механізму & Перевірка наскільки точно механізм працює при зібранні годинника в цілому; перевірка інтеграції між заводним механізмом, стрілками і циферблатом. \\
		\hline
	\end{longtable}
	
	Об'єкт тестування: модуль проектування.
	
	\begin{longtable}{|p{4.5cm}|p{5cm}|p{7cm}|}
		\hline
		\textbf{Вид тестування} & \textbf{Короткий опис виду тестування} & \textbf{Тестові перевірки} \\
		\hline
		\textbf{Functional Testing} & Порівняння специфікацій з фактичною функціональністю & 1. Перевірка можливості створювати, редагувати, зберігати UML-діаграми.
		2. Перевірка додавання нових елементів UML (класів, відносин).
		3. Перевірка функції експорту діаграм у різні формати (PNG, XML).\\
		\hline
		\textbf{Safety Testing} & Визначення можливих ризиків при використанні & 1. Перевірка захисту моделі від некоректних дій, що можуть призвести до втрати даних.
		2. Перевірка збереження резервних копій UML-діаграм.
		3. Перевірка обмеження доступу до критичних функцій для звичайних користувачів.\\
		\hline
		\textbf{Security Testing} & Оцінка захищеності продукту & 1. Перевірка доступу до UML-діаграм тільки авторизованих користувачів.
		2. Перевірка наявності шифрування даних UML при збереженні на сервері.
		3. Перевірка на захищеність від SQL-ін'єкцій у запитах до бази даних. \\
		\hline
		\textbf{Compatibility Testing} & Перевірка роботи в різних середовищах & 1. Перевірка коректного відображення UML-діаграм в різних браузерах (Google Chrome, Mozilla Firefox).
		2. Перевірка роботи на мобільних пристроях з різними операційними системами (Android, iOS).
		3. Перевірка сумісності з різними версіями операційних систем (Windows, macOS, Linux). \\
		\hline
		\textbf{GUI Testing} & Взаємодія з системою через графічний інтерфейс & 1. Перевірка правильності відображення UML-діаграм на різних розмірах екрану.
		2. Перевірка інтуїтивності інтерфейсу для користувача.
		3. Перевірка коректності відображення елементів UML при масштабуванні вікна. \\
		\hline
		\textbf{Usability Testing} & Оцінка зручності використання & 1. Перевірка інтуїтивності інтерфейсу для створення UML-діаграм.
		2. Перевірка логіки розташування кнопок і меню.
		3. Оцінка зручності роботи з UML-діаграмами для нових користувачів.\\
		\hline
		\textbf{Accessibility Testing} & Перевірка доступності для користувачів з обмеженими можливостями & 1. Перевірка коректності відображення для користувачів з проблемами зору.
		2. Тестування сумісності з програмами для зчитування тексту (TTS).
		3. Перевірка доступності інтерфейсу для людей з проблемами розрізнення кольорів.\\
		\hline
		\textbf{Internationalization Testing} & Адаптація продукту до мовних та культурних особливостей & 1. Перевірка коректності перекладу текстів у різних мовах.
		2. Перевірка відсутності образливих або некоректних термінів у різних культурах.
		3. Перевірка сумісності з правописом і форматом дат у різних регіонах.\\
		\hline
		\textbf{Performance Testing} & Оцінка продуктивності & 1. Перевірка часу відгуку при одночасному редагуванні великої кількості UML-діаграм.
		2. Оцінка продуктивності при роботі з великою кількістю елементів UML на діаграмі.
		3. Перевірка швидкості збереження UML-діаграм на сервері.\\
		\hline
		\textbf{Stress Testing} & Оцінка стабільності при збільшенні навантаження & 1. Перевірка роботи модуля при великій кількості одночасних користувачів.
		2. Оцінка стійкості системи при додаванні складних UML-діаграм з великою кількістю зв'язків.
		3. Перевірка стабільності серверів при максимальних навантаженнях на збереження UML-діаграм.\\
		\hline
		\textbf{Negative Testing} & Перевірка роботи системи при некоректних діях & 1. Перевірка внесення некоректних UML-елементів і їх обробка системою.
		2. Перевірка реакції на несанкціоновані зміни у файлах UML.
		3. Тестування можливості створення діаграм з некоректними параметрами (наприклад, відсутність обов'язкових полів).\\
		\hline
		\textbf{Black Box Testing} & Тестування без знання внутрішньої структури & 1. Перевірка правильності збереження UML-діаграм.
		2. Тестування функціоналу імпорту UML-діаграм з зовнішніх файлів.
		3. Перевірка створення складних діаграм без знання алгоритмів модуля.\\
		\hline
		\textbf{Automated Testing} & Автоматизоване тестування сценаріїв використання & 1. Автоматичне тестування створення і збереження UML-діаграм.
		2. Автоматична перевірка коректності експорту UML-діаграм.
		3. Автоматичне тестування роботи з великими UML-моделями.\\
		\hline
		\textbf{Unit/Component Testing} & Перевірка окремих частин модуля & 1. Протестувати методи додавання елементів UML (класів, відносин).
		2. Тестування функції видалення елементів UML з діаграми.
		3. Протестувати методи валідації зв'язків між елементами UML. \\
		\hline
		\textbf{Integration Testing} & Перевірка взаємодії між модулями & 1. Перевірка взаємодії модуля проектування з модулем зберігання UML-діаграм.
		2. Тестування коректної передачі даних між модулями проектування і моделювання.
		3. Перевірка інтеграції з системою контролю версій UML-діаграм.\\
		\hline
	\end{longtable}
	
	\section*{Тестування першої поставки програмного забезпечення (Build 1)}
	\subsection*{Модуль 1: Аналіз вимог}
	(Smoke) + NFT + NFT + NFT
	\begin{longtable}{|p{7cm}|p{3.5cm}|p{3.5cm}|}
		\hline
		\textbf{ID | Name} & \multicolumn{2}{|p{7cm}|}{\textbf{Summary}}
		\\\hline
		№1.1.1 | Перевірка запуску сторінки аналізу вимог & \multicolumn{2}{|p{7cm}|}{Перевірка коректності відображення сторінки аналізу вимог.}
		\\\hline
		\textbf{Steps} & \textbf{Expected Results} & \textbf{Resolution}
		\\\hline
		1. Запустити веб-браузер. 2. Перейти до модуля аналізу вимог.
		& Сторінка завантажується без помилок і відображає всі елементи.
		& Pending
		\\\hline
	\end{longtable}
	
	\subsection*{Модуль 2: Дизайн}
	\begin{longtable}{|p{7cm}|p{3.5cm}|p{3.5cm}|}
		\hline
		\textbf{ID | Name} & \multicolumn{2}{|p{7cm}|}{\textbf{Summary}}
		\\\hline
		№2.2.1 | Перевірка створення UML діаграм & \multicolumn{2}{|p{7cm}|}{Перевірка функції створення UML діаграм у модулі дизайну.}
		\\\hline
		\textbf{Steps} & \textbf{Expected Results} & \textbf{Resolution}
		\\\hline
		1. Запустити веб-браузер. 2. Перейти до модуля дизайну. 3. Створити нову UML діаграму.
		& Діаграма створюється і зберігається коректно.
		& Pending
		\\\hline
	\end{longtable}
	
	\subsection*{Модуль 3: Моделювання}
	\begin{longtable}{|p{7cm}|p{3.5cm}|p{3.5cm}|}
		\hline
		\textbf{ID | Name} & \multicolumn{2}{|p{7cm}|}{\textbf{Summary}}
		\\\hline
		№3.3.1 | Перевірка моделювання компонентів & \multicolumn{2}{|p{7cm}|}{Перевірка функції моделювання компонентів у модулі моделювання.}
		\\\hline
		\textbf{Steps} & \textbf{Expected Results} & \textbf{Resolution}
		\\\hline
		1. Запустити веб-браузер. 2. Перейти до модуля моделювання. 3. Моделювати новий компонент.
		& Компонент створюється і зберігається коректно.
		& Pending
		\\\hline
	\end{longtable}
	
	\section*{Тестування другої поставки програмного забезпечення (Build 2)}
	
	(Smoke) + DV + NFT + RT
	
	\subsection*{Модуль 1: Розробка}
	\begin{longtable}{|p{7cm}|p{3.5cm}|p{3.5cm}|}
		\hline
		\textbf{ID | Name} & \multicolumn{2}{|p{7cm}|}{\textbf{Summary}}
		\\\hline
		№4.1.1 | Перевірка функції редагування документів & \multicolumn{2}{|p{7cm}|}{Перевірка функції редагування документів в модулі розробки.}
		\\\hline
		\textbf{Steps} & \textbf{Expected Results} & \textbf{Resolution}
		\\\hline
		1. Запустити веб-браузер. 2. Перейти до модуля розробки. 3. Відкрити документ і внести зміни.
		& Документ створюється і зберігається коректно.
		& Pending
		\\\hline
	\end{longtable}
	
	\subsection*{Модуль 2: Тестування}
	\begin{longtable}{|p{7cm}|p{3.5cm}|p{3.5cm}|}
		\hline
		\textbf{ID | Name} & \multicolumn{2}{|p{7cm}|}{\textbf{Summary}}
		\\\hline
		№4.2.1 | Перевірка автоматизованих тестів & \multicolumn{2}{|p{7cm}|}{Перевірка коректності виконання автоматизованих тестів.}
		\\\hline
		\textbf{Steps} & \textbf{Expected Results} & \textbf{Resolution}
		\\\hline
		1. Запустити веб-браузер. 2. Перейти до модуля тестування. 3. Запустити автоматизовані тести.
		& Тести виконуються і відображають результати.
		& Pending
		\\\hline
	\end{longtable}
	
	\section*{Тестування третьої поставки програмного забезпечення (Build 3)}
	
	DV + (Internationalization Testing + AT) + RT
	
	\subsection*{Модуль 1: Моделювання}
	\begin{longtable}{|p{7cm}|p{3.5cm}|p{3.5cm}|}
		\hline
		\textbf{ID | Name} & \multicolumn{2}{|p{7cm}|}{\textbf{Summary}}
		\\\hline
		№5.1.1 | Перевірка локалізації & \multicolumn{2}{|p{7cm}|}{Перевірка перекладу інтерфейсу на англійську мову.}
		\\\hline
		\textbf{Steps} & \textbf{Expected Results} & \textbf{Resolution}
		\\\hline
		1. Запустити веб-браузер. 2. Перейти до налаштувань мови. 3. Вибрати англійську мову.
		& Інтерфейс відображається англійською мовою.
		& Pending
		\\\hline
	\end{longtable}
	
	\subsection*{Модуль 2: Тестування}
	\begin{longtable}{|p{7cm}|p{3.5cm}|p{3.5cm}|}
		\hline
		\textbf{ID | Name} & \multicolumn{2}{|p{7cm}|}{\textbf{Summary}}
		\\\hline
		№5.2.1 | Перевірка автоматизованих тестів & \multicolumn{2}{|p{7cm}|}{Перевірка виконання автоматизованих тестів з новими вимогами.}
		\\\hline
		\textbf{Steps} & \textbf{Expected Results} & \textbf{Resolution}
		\\\hline
		1. Запустити веб-браузер. 2. Перейти до модуля тестування. 3. Запустити автоматизовані тести з новими сценаріями.
		& Тести виконуються коректно.
		& Pending
		\\\hline
	\end{longtable}
	
	\section*{Тестування четвертої поставки програмного забезпечення (Build 4)}
	
	DV + (Stress Testing) + RT
	
	\subsection*{Модуль 1: Тестування}
	\begin{longtable}{|p{7cm}|p{3.5cm}|p{3.5cm}|}
		\hline
		\textbf{ID | Name} & \multicolumn{2}{|p{7cm}|}{\textbf{Summary}}
		\\\hline
		№6.1.1 | Перевірка функціоналу при високому навантаженні & \multicolumn{2}{|p{7cm}|}{Перевірка стабільності системи при навантаженні 2000 користувачів.}
		\\\hline
		\textbf{Steps} & \textbf{Expected Results} & \textbf{Resolution}
		\\\hline
		1. Запустити веб-браузер. 2. Симулювати 2000 користувачів. 3. Перевірити функціонал для кожної ролі.
		& Система стабільна під навантаженням.
		& Pending
		\\\hline
	\end{longtable}
		
	\section*{Висновки}
	ПІд час виконання лабораторної роботи, я вивчив класифікацію видів тестування, розробив перевірки для різних видів тестування, навчився планувати тестові активності в залежності від особливостей продукції, що поставляється на тестування 	функціональності.
	    
\end{normalsize}
\end{document}
