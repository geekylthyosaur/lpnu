\documentclass[oneside,14pt]{extarticle}
\usepackage{cmap}
\usepackage[utf8]{inputenc}
\usepackage[english,ukrainian]{babel}
\usepackage{graphicx}
\usepackage{geometry}
\usepackage{listings}
\usepackage{float}
\usepackage{amsmath}
\usepackage{subfig}
\usepackage{tempora}
\geometry{
	a4paper,
	left=20mm,
	right=20mm,
	top=15mm,
	bottom=15mm,
}
\lstset{
	language=c,
	tabsize=4,
	keepspaces,
	showstringspaces=false,
	frame=single,
	breaklines,
	language=C,
}
\graphicspath{ {./pictures} }
\setlength{\parindent}{4em}

\newcommand\subject{Безпека програм та даних}
\newcommand\lecturer{к.т.н., доцент кафедри ПЗ\\Сенів М.М.}
\newcommand\teacher{к.т.н., доцент кафедри ПЗ\\Сенів М.М.}
\newcommand\mygroup{ПЗ-42}
\newcommand\lab{3}
\newcommand\theme{Створення програмного засобу для забезпечення конфіденційності інформації}
\newcommand\purpose{Ознайомитись з методами криптографічного забезпечення
	конфіденційності інформації, навчитись створювати комплексні програмні
	продукти для захисту інформації з використанням алгоритмів симетричного
	шифрування, хешування та генераторів псевдовипадкових чисел}

\begin{document}
\begin{normalsize}
	\begin{titlepage}
		\thispagestyle{empty}
		\begin{center}
			\textbf{МІНІСТЕРСТВО ОСВІТИ І НАУКИ УКРАЇНИ\\
				НАЦІОНАЛЬНИЙ УНІВЕРСИТЕТ "ЛЬВІВСЬКА ПОЛІТЕХНІКА"}
		\end{center}
		\begin{flushright}
			\textbf{ІКНІ}\\
			Кафедра \textbf{ПЗ}
		\end{flushright}
		\vspace{80pt}
		\begin{center}
			\textbf{ЗВІТ}\\
			\vspace{10pt}
			до лабораторної роботи № \lab\\
			\textbf{на тему}: <<\textit{\theme}>>\\
			\textbf{з дисципліни}: <<\subject>>
		\end{center}
		\vspace{80pt}
		\begin{flushright}
			
			\textbf{Лектор}:\\
			\lecturer\\
			\vspace{28pt}
			\textbf{Виконав}:\\
			
			студент групи \mygroup\\
			Коваленко Д.М.\\
			\vspace{28pt}
			\textbf{Прийняв}:\\
			
			\teacher\\
			
			\vspace{28pt}
			«\rule{1cm}{0.15mm}» \rule{1.5cm}{0.15mm} 2024 р.\\
			$\sum$ = \rule{1cm}{0.15mm}……………\\
			
		\end{flushright}
		\vspace{\fill}
		\begin{center}
			\textbf{Львів — 2024}
		\end{center}
	\end{titlepage}
		
	\begin{description}
		\item[Тема.] \theme.
		\item[Мета.] \purpose.
	\end{description}

    \section*{Лабораторне завдання}
    Згідно до варіанту, наведеного в таблиці, створити прикладну програму
    для шифрування інформації за алгоритмом RC5.
    Програма повинна отримувати від користувача парольну фразу і, на її
    основі, шифрувати файли довільного розміру, а результат зберігати у вигляді
    файлу з можливістю подальшого дешифрування (при введенні тієї самої
    парольної фрази).
    Для перетворення парольної фрази у ключ шифрування використати
    алгоритм MD5, реалізований в лабораторній роботі № 2 – ключем шифрування
    повинен бути хеш парольної фрази. Якщо згідно варіанту довжина ключа
    становить 64 біти, беруться молодші 64 біти хешу; якщо довжина ключа
    повинна бути 256 бітів, то хеш парольної фрази стає старшими 128 бітами, а
    молодшими є хеш від старших 128 бітів (тобто, позначивши парольну фразу
    через P, отримаємо K=H(H(P))||H(P)).
    Для забезпечення можливості роботи створеного програмного продукту з
    відкритим текстом довільної довжини, програмну реалізацію здійснити в
    режимі RC5-CBC-Pad. В якості вектора ініціалізації (IV) використати генератор
    псевдовипадкових чисел, реалізований в лабораторній роботі № 1. Для
    кожного нового шифрованого повідомлення слід генерувати новий вектор
    ініціалізації. Вектор ініціалізації зашифровується в режимі ECB і зберігається в
    першому блоці зашифрованого файлу.
    
	\section*{Хід роботи}
	
	\subsection*{Виконання програми}
	\begin{figure}[H]
		\centering
		\includegraphics[width=\columnwidth]{1}
		\caption{Робота програми}
	\end{figure}
	
	\subsection*{Код програми}
	Файл \textit{main.rs}:
	{\small	\lstinputlisting{src/main.rs}}
	
	\section*{Висновки}
	Під час виконання лабораторної роботи я ознайомився з методами криптографічного забезпечення
	конфіденційності інформації, навчився створювати комплексні програмні
	продукти для захисту інформації з використанням алгоритмів симетричного
	шифрування, хешування та генераторів псевдовипадкових чисел.
	    
\end{normalsize}
\end{document}
