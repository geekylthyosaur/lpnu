\documentclass[14pt]{extreport}

\usepackage[utf8]{inputenc}
\usepackage[T2A]{fontenc}
\usepackage[english,ukrainian]{babel}
\usepackage{tempora}

\linespread{1.5}
\setlength{\parskip}{0pt}
\usepackage{indentfirst}

\usepackage[a4paper,top=20mm,bottom=20mm,left=25mm,right=10mm]{geometry}

\usepackage{fancyhdr}
\fancypagestyle{plain}{
  \pagestyle{myheadings}
}
\pagestyle{myheadings}
\setcounter{page}{4}

\usepackage{titlesec}
\titleformat{\chapter}{\centering\bfseries\MakeUppercase}{\chaptername~\thechapter.}{1pc}{}
\titleformat{\section}{\bfseries}{\thesection.}{1pc}{}
\titlespacing*{\chapter}{0pt}{10mm}{14pt}
\titlespacing*{\section}{\parindent}{14pt}{0pt}

\usepackage{enumitem}
\setlist{nolistsep}

\begin{document}
  \chapter*{Анотація}
  
  \tableofcontents
  \newpage
  
  \chapter*{Вступ}
  \addcontentsline{toc}{chapter}{Вступ}
  
  У сучасному світі цифрові технології відіграють ключову роль у забезпеченні прозорості та довіри в різних сферах суспільного життя. Однією з таких сфер є електронне голосування, яке дедалі частіше використовується як альтернатива традиційним методам голосування. Однак існуючі централізовані системи голосування стикаються з низкою проблем, таких як ризик шахрайства, можливість маніпуляцій та складність забезпечення прозорості процесу. Для вирішення цих проблем необхідно розробляти нові підходи, які б гарантували захист даних, достовірність результатів та незалежність від єдиної керуючої сторони.

  Актуальність теми цієї роботи зумовлена потребою у створенні сучасних електронних систем голосування, які забезпечуватимуть високий рівень безпеки та довіри. З розвитком технологій все більше організацій, компаній та державних установ прагнуть впроваджувати цифрові рішення для голосування, що потребує детального аналізу існуючих підходів, порівняння централізованих і децентралізованих систем, а також розробки нових моделей, що враховуватимуть переваги та недоліки кожного підходу.

  Метою цієї роботи є дослідження та розробка системи для проведення голосувань та опитувань, яка відповідатиме вимогам прозорості, безпеки та достовірності результатів. У процесі виконання роботи будуть розглянуті як централізовані, так і децентралізовані системи голосування, проаналізовані їхні переваги та недоліки, а також запропоноване рішення, яке об'єднує найкращі властивості цих підходів.

  Результати цієї роботи можуть бути використані для подальшого розвитку електронних систем голосування, а також як основа для впровадження подібних рішень у різних сферах діяльності – від державного управління до корпоративних голосувань та соціальних опитувань.
  
  % Оглядовий розділ
  \chapter{Аналіз сучасних підходів до побудови електронних систем голосування}

  \section{Традиційні системи електронного голосування}
  
  Традиційні системи електронного голосування використовуються для автоматизації процесу голосування у різних сферах, включаючи вибори в державних органах, корпоративні голосування та опитування громадської думки. Основною характеристикою таких систем є їхня залежність від централізованих компонентів, які відповідають за обробку та зберігання голосів.

  Існуючі рішення поділяються на дві основні категорії: системи для голосування на виборчих дільницях та системи для віддаленого голосування через інтернет. Системи першого типу зазвичай базуються на спеціалізованих апаратних пристроях, які дозволяють виборцям віддати свій голос шляхом взаємодії з електронними терміналами. Дані з терміналів передаються до центрального серверу, де здійснюється підрахунок голосів та формування підсумкових результатів.

  Другий тип – системи віддаленого голосування – надає можливість виборцям брати участь у голосуванні через інтернет. Такі системи забезпечують зручність і доступність для широкого кола користувачів, проте водночас мають вищі вимоги до безпеки та автентифікації. Однією з найважливіших проблем є забезпечення конфіденційності виборців, автентичності голосів та захисту від зовнішніх атак.

  Недоліками традиційних систем електронного голосування є можливість маніпуляцій з боку адміністраторів системи, низька прозорість процесу та обмежені можливості для незалежної перевірки результатів. Зокрема, користувачі повинні довіряти операторам системи, що викликає сумніви у випадках, коли йдеться про вибори, що мають значний вплив на суспільство.

  Попри ці проблеми, традиційні системи голосування широко використовуються у багатьох країнах, оскільки дозволяють значно знизити витрати на організацію виборів та скоротити час обробки голосів. Однак для подолання вказаних недоліків потрібні нові підходи до побудови електронних систем голосування, одним із яких є використання децентралізованих технологій.
  
  \section{Технологія блокчейн та її застосування у голосуванні}
  
  Блокчейн є розподіленою технологією зберігання даних, що дозволяє створювати децентралізовані системи, де кожен учасник мережі має копію всього реєстру транзакцій. Основною особливістю блокчейну є його незмінність: після додавання даних до блоку вони стають постійними і не можуть бути змінені без згоди більшості учасників мережі. Така властивість забезпечує високий рівень прозорості та довіри до системи.

  Основні компоненти блокчейн-системи включають блоки, які містять інформацію про транзакції, ноди — вузли мережі, що перевіряють та підтверджують транзакції, а також механізм консенсусу, що дозволяє узгоджувати порядок додавання нових блоків у мережу. Найбільш поширеними є алгоритми Proof of Work (доказ виконаної роботи), де вузли змагаються за право додати новий блок, розв’язуючи складну криптографічну задачу, та Proof of Stake (доказ частки володіння), де ймовірність вибору вузла для додавання нового блоку залежить від кількості криптовалюти, якою володіє цей вузол.
  
  Окрім цих підходів, існують альтернативні алгоритми, зокрема Proof of History (PoH). PoH є механізмом, що забезпечує впорядкування транзакцій у часі за допомогою криптографічних міток часу. Завдяки PoH кожен вузол може підтверджувати порядок подій без необхідності узгодження з іншими вузлами, що значно прискорює обробку транзакцій і підвищує продуктивність мережі.

  Технологія блокчейн має низку переваг для застосування у голосуванні:
  \begin{enumerate}
    \item Прозорість, оскільки будь-який учасник може перевірити результати голосування.
    \item Захищеність від шахрайства, що забезпечується за рахунок криптографічного захисту даних.
    \item Незалежність від єдиного центру, що знижує ризик маніпуляцій.
  \end{enumerate}

  Попри ці переваги, блокчейн-системи стикаються із такими викликами, як низька масштабованість більшості популярних платформ, складність впровадження у великих виборчих процесах та значні витрати ресурсів на підтримку мережі. Тим не менш, блокчейн залишається перспективною технологією, яка може забезпечити значні переваги в організації безпечних та прозорих електронних голосувань у майбутньому.
  
  \section{Платформа Solana як інструмент для розробки децентралізованих додатків}
  
  Solana є блокчейн-платформою, що вирізняється високою продуктивністю, низькою латентністю та стабільною вартістю транзакцій. Однією з ключових переваг платформи є використання механізму консенсусу Proof of History, який дозволяє значно підвищити швидкість обробки транзакцій та забезпечити високу пропускну здатність мережі.

  Solana підтримує розробку смарт-контрактів за допомогою мови програмування Rust, що дозволяє створювати безпечні та ефективні додатки. Rust відомий своїми можливостями забезпечення пам’яткової безпеки та ефективного управління ресурсами, що особливо важливо для децентралізованих додатків із великим навантаженням. Крім того, розробники мають доступ до екосистеми бібліотек і фреймворків, які спрощують створення та тестування смарт-контрактів.

  Причина вибору Solana для виконання цього завдання полягає у високій пропускній здатності платформи, що дозволяє обробляти тисячі транзакцій за секунду без суттєвих затримок. Це критично важливо для систем голосування, де від швидкості обробки даних залежить зручність користування системою та її масштабованість. Додатковим фактором є низька вартість транзакцій, що дозволяє користувачам взаємодіяти із системою без значних фінансових витрат.

  Таким чином, вибір Solana як платформи для реалізації системи голосування обумовлений її технічними перевагами, що забезпечують високу продуктивність, безпеку та масштабованість, необхідні для успішної реалізації поставленого завдання.
  
  \section{Висновки до розділу}
  
  У розділі було розглянуто основні підходи до організації електронних голосувань, починаючи від традиційних систем та закінчуючи сучасними децентралізованими рішеннями. Традиційні системи, хоч і є поширеними, мають значні обмеження щодо прозорості, надійності та захищеності даних. Використання блокчейн-технології дозволяє вирішити ці проблеми завдяки децентралізованій структурі, незмінності даних та можливості їх відкритої перевірки всіма учасниками.

  Серед доступних блокчейн-платформ було обрано Solana, оскільки вона забезпечує високу продуктивність, мінімальні затримки та низькі транзакційні витрати, що є критично важливими для системи голосування. Завдяки цьому Solana може слугувати надійною основою для створення децентралізованих додатків, які відповідають вимогам безпеки та прозорості.

  % Розділ постановки завдання
  \chapter{Постановка задачі розробки системи для голосування}

  \section{Формулювання проблеми}
  
  Електронні системи голосування все частіше використовуються як альтернатива традиційним паперовим голосуванням завдяки їхній зручності та можливості автоматизації підрахунку голосів. Однак, попри очевидні переваги, такі системи мають ряд серйозних недоліків, пов’язаних із прозорістю процесу, захистом даних та довірою до результатів.

У традиційних електронних системах голосування значна частина операцій виконується централізованими серверами, що створює можливість маніпуляцій з боку адміністраторів або зовнішніх атак. Виборці, як правило, не мають доступу до детальної інформації про перебіг голосування, а результати підрахунку голосів залежать від довіри до операторів системи. Це породжує сумніви у справедливості виборчого процесу та знижує довіру до результатів голосування.

Крім того, питання безпеки є одним із ключових для електронних систем голосування. Зовнішні атаки, спроби фальсифікації голосів або витоку особистих даних виборців можуть суттєво підірвати легітимність голосування. У таких умовах необхідно створити систему, яка б гарантувала прозорість процесу, захист голосів від маніпуляцій та конфіденційність виборців.

Проблема полягає в тому, щоб розробити таку систему електронного голосування, яка забезпечуватиме:  
- прозорість усіх етапів голосування для кожного учасника процесу;  
- захищеність від фальсифікацій і шахрайства;  
- анонімність виборців із гарантією того, що голос не може бути відстежений;  
- відсутність єдиної точки контролю, яка могла б впливати на результати голосування.

Таким чином, основною проблемою, яку необхідно вирішити, є забезпечення довіри до системи голосування шляхом створення надійної, прозорої та децентралізованої архітектури, що усуне залежність від централізованих операторів та підвищить захищеність голосів.

  \section{Визначення цілей та завдань розробки}
  \section{Сценарії використання}
  \section{Специфікація вимог до системи}
  \section{Висновки до розділу}

  % Проєктний розділ
  \chapter{Проєктування системи для проведення голосувань}

  \section{Загальна архітектура системи}
  \section{Проектування смарт-контракту для голосувань}
  \section{Моделювання клієнтської частини та API}
  \section{Забезпечення безпеки та стійкості системи}
  \section{Висновки до розділу}
  
  % Розділ програмної реалізації та тестування
  \chapter{<Розділ програмної реалізації та тестування>}
  
  % Розділ з економіки
  \chapter{<Розділ з економіки>}
  
  \chapter*{Висновки}
  \addcontentsline{toc}{chapter}{Висновки}
  
  \renewcommand\bibname{\MakeUppercase{Список літератури}}
  \addcontentsline{toc}{chapter}{Список літератури}
  \begin{thebibliography}{99}
    \bibitem{source1} Автор. Назва книги. Видавництво, рік.
  \end{thebibliography}
  
  \appendix
  \renewcommand{\thechapter}{\Alph{chapter}}
  \renewcommand{\chaptername}{Додаток}
  
  \chapter{<Назва додатку>}
  
\end{document}