\documentclass[14pt]{extreport}

\usepackage[utf8]{inputenc}
\usepackage[T2A]{fontenc}
\usepackage[english,ukrainian]{babel}
\usepackage{tempora}

\linespread{1.5}
\usepackage{indentfirst}

\usepackage[a4paper,top=20mm,bottom=20mm,left=25mm,right=10mm]{geometry}

\usepackage{fancyhdr}
\fancypagestyle{plain}{
  \pagestyle{myheadings}
}
\pagestyle{myheadings}
\setcounter{page}{4}

\usepackage{titlesec}
\titleformat{\chapter}{\centering\bfseries\MakeUppercase}{\chaptername~\thechapter.}{1pc}{}
\titleformat{\section}{\bfseries}{\thesection.}{1pc}{}
\titlespacing*{\chapter}{0pt}{-30pt}{0pt}

\begin{document}
  \chapter*{Анотація}
  
  \tableofcontents
  \newpage
  
  \chapter*{Вступ}
  \addcontentsline{toc}{chapter}{Вступ}
  
  
  % Оглядовий розділ
  \chapter{Аналіз сучасних технологій та інструментів для побудови децентралізованих систем голосування}

  \section{Огляд електронних систем голосування}
  
  
  \section{Основи технології блокчейн та її застосування у голосуваннях}
  
  
  \section{Характеристики платформи Solana для смарт-контрактів}
  
  
  \section{Порівняльний аналіз платформ для розробки децентралізованих додатків}
  
  
  \section{Висновки до розділу}
  

  % Розділ постановки завдання
  \chapter{Постановка задачі розробки системи для голосування на базі блокчейну Solana}

  \section{Формулювання проблеми}
  \section{Основні вимоги до системи}
  \section{Сценарії використання}
  \section{Специфікація вимог до системи}
  \section{Висновки до розділу}

  % Проєктний розділ
  \chapter{Проєктування системи для проведення голосувань}

  \section{Загальна архітектура системи}
  \section{Проектування смарт-контракту для голосувань}
  \section{Моделювання клієнтської частини та API}
  \section{Забезпечення безпеки та стійкості системи}
  \section{Висновки до розділу}
  
  % Розділ програмної реалізації та тестування
  \chapter{<Розділ програмної реалізації та тестування>}
  
  % Розділ з економіки
  \chapter{<Розділ з економіки>}
  
  \chapter*{Висновки}
  \addcontentsline{toc}{chapter}{Висновки}
  
  \renewcommand\bibname{\MakeUppercase{Список літератури}}
  \addcontentsline{toc}{chapter}{Список літератури}
  \begin{thebibliography}{99}
    \bibitem{source1} Автор. Назва книги. Видавництво, рік.
  \end{thebibliography}
  
  \appendix
  \renewcommand{\thechapter}{\Alph{chapter}}
  \renewcommand{\chaptername}{Додаток}
  
  \chapter{<Назва додатку>}
  
\end{document}