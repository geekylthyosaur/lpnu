\documentclass[14pt]{extreport}
\usepackage{caption}
\usepackage[utf8]{inputenc}
\usepackage[T2A]{fontenc}
\usepackage[english,ukrainian]{babel}
\usepackage{tempora}
\usepackage{float}
\linespread{1.5}
\setlength{\parskip}{0pt}
\usepackage{indentfirst}
\usepackage{graphicx}
\usepackage{cmap}
\usepackage{longtable}
\usepackage{titlesec}
\usepackage{geometry}
\usepackage{listings}
\renewcommand{\arraystretch}{1.5}
\geometry{
	a4paper,
	left=20mm,
	right=20mm,
	top=15mm,
	bottom=15mm,
}

\graphicspath{ {./pictures} }
\setlength{\parindent}{4em}

\newcommand\subject{Програмування мікроконтролерів}
\newcommand\lecturer{доцент кафедри ПЗ\\Федорчук Є.Н.}
\newcommand\teacher{доцент кафедри ПЗ\\Федорчук Є.Н.}
\newcommand\mygroup{ПЗ-42}
\newcommand\lab{1}
\newcommand\theme{Моделювання характеристик людино-машинної взаємодії (ЛМВ) і проектування інтерфейсу}
\newcommand\purpose{Навчитися проектувати інтерфейс з використанням характеристик людино-машинної взаємодії}

\begin{document}
\begin{normalsize}
	\begin{titlepage}
		\thispagestyle{empty}
		\begin{center}
			\textbf{МІНІСТЕРСТВО ОСВІТИ І НАУКИ УКРАЇНИ\\
				НАЦІОНАЛЬНИЙ УНІВЕРСИТЕТ "ЛЬВІВСЬКА ПОЛІТЕХНІКА"}
		\end{center}
		\begin{flushright}
			\textbf{ІКНІ}\\
			Кафедра \textbf{ПЗ}
		\end{flushright}
		\vspace{20pt}
		\begin{center}
			\textbf{ЗВІТ}\\
			\vspace{10pt}
			до лабораторної роботи № \lab\\
			\textbf{на тему}: <<\textit{\theme}>>\\
			\textbf{з дисципліни}: <<\subject>>
		\end{center}
		\vspace{20pt}
		\begin{flushright}
			
			\textbf{Лектор}:\\
			\lecturer\\
			\vspace{28pt}
			\textbf{Виконав}:\\
			
			студенти групи \mygroup\\
			Коваленко Д.М.\\
			\vspace{28pt}
			\textbf{Прийняв}:\\
			
			\teacher\\
			
			\vspace{28pt}
			«\rule{1cm}{0.15mm}» \rule{1.5cm}{0.15mm} 2025 р.\\
			$\sum$ = \rule{1cm}{0.15mm}……………\\
			
		\end{flushright}
		\vspace{\fill}
		\begin{center}
			\textbf{Львів — 2025}
		\end{center}
	\end{titlepage}
		
	\begin{description}
		\item[Тема.] \theme.
		\item[Мета.] \purpose.
	\end{description}

  \section*{Лабораторне завдання}
  \begin{enumerate}
  	\item Обрати кількість вікон-кадрів інтерфейсу (від 3 до 5).
  \item Обрати формат діалогу у вигляді сценаріїв-послідовне відображення кадрів кожного вікна.
  \item Розробити шаблони кадрів (уведення даних, виведення повідомлень, виведення тривожних повідомлень).
  \item Для кожного шаблону описати необхідну множину процедур виведення повідомлень.
  \item Описати множину процедур уведення даних.
  \item Обрати для шаблонів ергономічні часові межі опрацювання інформації.
  \item Оформити звіт у формі таблиці. Стовбці – назви завдань за пунктами списку. Для окремих завдань може бути декілька стовбців.Рядки – чисельно-символьний опис реалізації завдань.
  \item Розробити ПЗ для реалізації діалогу ЛМВ за допомогою багатовіконного інтерфейсу.
  \item Виконати тестування ПЗ.
  \item Оформити звіт з використанням візуальних форм ЛМВ.
  \end{enumerate}
  
  \section*{Хід роботи}
  
  \begin{table}[H]
  \centering
  \renewcommand{\tablename}{Таблиця}
  \renewcommand{\thetable}{\arabic{table}}
  \captionsetup{justification=raggedleft, singlelinecheck=false, labelsep=period}
  \caption{}
  \textbf{Результати проектування інтерфейсу\vspace{5pt}}
  \resizebox{\textwidth}{!}{
  \begin{tabular}{|l|p{4cm}|p{4cm}|p{3cm}|p{3cm}|}
  \hline
  \textbf{№} & \textbf{Формат діалогу} & \textbf{Шаблон кадру} & \textbf{Множина процедур введення даних} & \textbf{Ергономічні часові межі} \\ \hline
  \textbf{1} &  &  &  & 0,1 \\ \hline
  \textbf{2} &  &  &  & 0,1 \\ \hline
  \textbf{3} &  &  &  & 0,1 \\ \hline
  \end{tabular}
  }
  \end{table}
	
	\section*{Висновки}
	
	Під час даної лабораторної роботи навчився проектувати інтерфейс з використанням характеристик людино-машинної взаємодії. Визначив скільки вікон матиме багатовіконний інтерфейс, формат діалогу вікон, їх шаблони, множини процеду введення даних, ергономічні часові межі опрацювання інформації. Створив діаграми прецедентів та класів розроблюваної системи, завдяки ним реалізовував проект. Під час процесу тестування знайшов та виправив помилки системи.
	
	    
\end{normalsize}
\end{document}
