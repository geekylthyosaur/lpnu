\documentclass[14pt]{extreport}

\usepackage[utf8]{inputenc}
\usepackage[T2A]{fontenc}
\usepackage[english,ukrainian]{babel}
\usepackage{tempora}

\usepackage{float}
\usepackage{caption}
\captionsetup[table]{justification=raggedleft, singlelinecheck=false, labelsep=period}
\captionsetup[figure]{labelsep=space}
\usepackage{graphicx}
\graphicspath{ {./pictures} }

\linespread{1.5}
\setlength{\parskip}{0pt}
\usepackage{indentfirst}

\usepackage[a4paper,top=20mm,bottom=20mm,left=25mm,right=10mm]{geometry}

\usepackage{fancyhdr}
\fancypagestyle{plain}{
  \pagestyle{myheadings}
}
\pagestyle{myheadings}
\setcounter{page}{4}

\setcounter{secnumdepth}{3}
\newcounter{req}[subsubsection]
\newcommand\req{\arabic{req}\stepcounter{req}}

\usepackage{titlesec}
\titleformat{\chapter}{\centering\bfseries\MakeUppercase}{\chaptername~\thechapter.}{1pc}{}
\titleformat{\section}{\bfseries}{\thesection.}{1pc}{}
\titleformat{\subsection}{\bfseries}{\thesubsection.}{1pc}{}
\titleformat{\subsubsection}{\bfseries}{\thesubsubsection.}{1pc}{}
% \titleformat{\paragraph}{\bfseries}{}{}{}

\titlespacing*{\chapter}{0pt}{10mm}{14pt}
\titlespacing*{\section}{\parindent}{14pt}{0pt}
\titlespacing*{\subsection}{\parindent}{0pt}{0pt}
\titlespacing*{\subsubsection}{\parindent}{0pt}{0pt}
\titlespacing*{\paragraph}{\parindent}{0pt}{7pt}

\usepackage{enumitem}
\setlist{nolistsep}

\usepackage{hyperref}
\def\UrlBreaks{\do\/\do-}

\newcommand\subject{Людино-машинна взаємодія (вбудовані системи)}
\newcommand\lecturer{доцент кафедри ПЗ\\Федорчук Є.Н.}
\newcommand\teacher{доцент кафедри ПЗ\\Федорчук Є.Н.}
\newcommand\mygroup{ПЗ-42}
\newcommand\lab{2}
\newcommand\theme{Вбудовані системи для управління автомобілями}
\newcommand\purpose{Навчитися моделювати людино-машинну взаємодію у
вбудованих системах}

\begin{document}
	\begin{titlepage}
		\thispagestyle{empty}
		\begin{center}
			\textbf{МІНІСТЕРСТВО ОСВІТИ І НАУКИ УКРАЇНИ\\
				НАЦІОНАЛЬНИЙ УНІВЕРСИТЕТ "ЛЬВІВСЬКА ПОЛІТЕХНІКА"}
		\end{center}
		\begin{flushright}
			\textbf{ІКНІ}\\
			Кафедра \textbf{ПЗ}
		\end{flushright}
		\vspace{20pt}
		\begin{center}
			\textbf{РЕФЕРАТ}\\
			\vspace{10pt}
			\textbf{на тему}: <<\textit{\theme}>>\\
			\textbf{з дисципліни}: <<\subject>>
		\end{center}
		\vspace{20pt}
		\begin{flushright}
			
			\textbf{Лектор}:\\
			\lecturer\\
			\vspace{28pt}
			\textbf{Виконав}:\\
			
			студенти групи \mygroup\\
			Коваленко Д.М.\\
			\vspace{28pt}
			\textbf{Прийняв}:\\
			
			\teacher\\
			
			\vspace{28pt}
			«\rule{1cm}{0.15mm}» \rule{1.5cm}{0.15mm} 2025 р.\\
			$\sum$ = \rule{1cm}{0.15mm}……………\\
			
		\end{flushright}
		\vspace{\fill}
		\begin{center}
			\textbf{Львів — 2025}
		\end{center}
	\end{titlepage}  
	
	Сучасний автомобіль можна порівняти з живим організмом, де кожен елемент виконує свою функцію під керівництвом центральної нервової системи. Тільки замість біологічних процесів тут працюють складні електронні системи, які постійно аналізують, обчислюють і приймають рішення. Розвиток цих технологій перетворив звичайний транспортний засіб на високотехнологічний пристрій, де механіка тісно переплетена з електронікою та програмним забезпеченням.

Поява перших електронних систем управління в автомобілях датується 1970-ми роками, коли інженери почали замінювати механічні карбюратори на електронні системи вприску палива. Це був революційний крок, який дозволив точніше контролювати процес згоряння палива. Проте справжній прорив стався з розвитком мікропроцесорів, що дало можливість створювати складні алгоритми управління. Сьогодні навіть бюджетні автомобілі містять десятки електронних блоків управління, які постійно обмінюються інформацією через внутрішні цифрові мережі.

Однією з найважливіших систем сучасного автомобіля є електронний блок управління двигуном. Цей складний пристрій безперервно отримує дані від численних датчиків, розкиданих по всій конструкції автомобіля. Температурні датчики контролюють стан охолоджуючої рідини, повітря на впуску та моторної оливи. Датчики тиску стежать за параметрами у впускному колекторі та паливній системі. Особливу роль відіграють датчики положення, які точно визначають позицію дросельної заслінки, розподільного та колінчастого валів. Лямбда-зонди, встановлені у вихлопній системі, аналізують склад відпрацьованих газів, допомагаючи підтримувати оптимальний склад паливно-повітряної суміші.

На основі всіх цих даних блок управління двигуном приймає безліч рішень. Він точно визначає момент запалювання, регулює кількість палива, що впорскується, керує роботою турбонаддува (якщо він передбачений конструкцією). У сучасних двигунах зі змінними фазами газорозподілу, таких як VTEC або Valvetronic, електроніка також відповідає за оптимальне перемикання режимів роботи. Ці системи настільки розумні, що здатні адаптуватися до стилю водіння конкретної людини, враховувати якість палива та навіть компенсувати зміни атмосферного тиску при русі у горах.

Не менш складною є система управління трансмісією, яка тісно взаємодіє з двигуном. Електронний блок управління коробкою передач постійно аналізує безліч параметрів: швидкість руху, положення педалі акселератора, нахил дороги, навантаження на автомобіль. У автоматичних коробках передач ця система визначає оптимальний момент для перемикання ступенів, забезпечуючи плавність ходу та економію палива. Особливо складною є робота варіаторів, де електроніка постійно підбирає ідеальне передавальне число. У роботизованих коробках типу DSG програмне забезпечення забезпечує неймовірно швидке і плавне перемикання передач, яке неможливо реалізувати за допомогою механічних пристроїв.

Системи безпеки сучасних автомобілів також побудовані на складній електроніці. Антиблокувальна гальмівна система, яка з'явилася ще в 1980-х роках, зазнала значних вдосконалень. Сьогодні вона працює в комплексі з системою розподілу гальмівних сил, яка автоматично змінює тиск у гальмівних контурах для кожного колеса окремо. Система допомоги при екстреному гальмуванні здатна розпізнати раптове натискання педалі і автоматично застосувати максимальний гальмівний тиск. Протибуксувальна система постійно моніторить ковзання коліс і втручається, коли це необхідно.

Найбільш просунутою системою безпеки є електронна система курсової стійкості. Вона використовує цілу мережу датчиків: акселерометри для вимірювання бічних прискорень, датчики кутової швидкості для визначення напрямку руху, датчики кута повороту керма для розуміння намірів водія. Коли система виявляє розбіжність між запланованою і фактичною траєкторією руху, вона миттєво втручається - підгальмовує окремі колеса або зменшує потужність двигуна, щоб запобігти заносу.

Останнім часом особливу увагу приділяють системам допомоги водієві, які отримали загальну назву ADAS. Ці технології використовують камери, радари та інші сенсори для постійного моніторингу дорожньої обстановки. Адаптивний круїз-контроль здатний не просто підтримувати задану швидкість, але й автоматично регулювати її, зберігаючи безпечну дистанцію до попереднього автомобіля. Система утримання в смузі попереджає про ненавмисний виїзд зі смуги руху, а в деяких випадках може автоматично підкоригувати траєкторію. Автоматичне екстрене гальмування стало справжнім проривом у безпеці - воно здатне розпізнати пішоходів або перешкоди на шляху і ініціювати гальмування ще до того, як водій встигне среагувати.

Сучасні інформаційно-розважальні системи вже далеко вийшли за рамки простих аудіопрогравачів. Вони інтегрують навігацію з онлайн-оновленням карт, голосове управління, бездротові технології для підключення смартфонів. Деякі виробники впроваджують віртуальні панелі приладів, які проектують інформацію прямо на вітрове скло, дозволяючи водієві отримувати всі необхідні дані, не відводячи очей від дороги.

Для забезпечення зв'язку між усіма цими системами використовуються спеціалізовані протоколи передачі даних. CAN-шина служить основним каналом зв'язку між критично важливими системами, такими як двигун, гальма чи трансмісія. Для більш простих пристроїв, наприклад електроприводів дзеркал або підігріву сидінь, використовується більш проста LIN-мережа. Для систем, які вимагають особливо швидкої реакції, таких як активна підвіска, застосовують технологію FlexRay. А для мультимедійних систем і майбутніх автономних функцій все частіше використовується автомобільний Ethernet, який забезпечує високу швидкість передачі даних.

З ростом кількості електронних систем особливої актуальності набуває питання кібербезпеки. Сучасні автомобілі стають вразливими до хакерських атак, тому виробники впроваджують складні системи захисту. Шифрування даних, захищені канали для оновлення програмного забезпечення "по повітрю", системи виявлення несанкціонованого доступу - все це стало невід'ємною частиною автомобільної електроніки.

Майбутнє автомобільної промисловості пов'язане з розвитком автономного водіння. Автомобілі п'ятого рівня автономності, які зможуть повністю обходитися без водія, вимагають неймовірно складних вбудованих систем. Для цього використовуються потужні нейромережі, здатні обробляти величезні обсяги даних з камер, радарів і лідарів. Такі компанії як NVIDIA розробляють спеціалізовані обчислювальні платформи для автономних автомобілів, а системи супутникової навігації досягають неймовірної точності завдяки корекції в реальному часі.

Еволюція вбудованих систем управління автомобілями пройшла шлях від простих електромеханічних пристроїв до складних інтелектуальних комплексів, які здатні самостійно аналізувати ситуацію і приймати рішення. У майбутньому ці технології продовжать розвиватися, наближаючи нас до ери повністю автономних транспортних засобів, де людина буде лише пасажиром, а всі функції керування візьме на себе штучний інтелект.

Сучасні вбудовані системи управління автомобілями перетворили транспортні засоби з простих механічних пристроїв у високотехнологічні комплекси, де електроніка та програмне забезпечення відіграють ключову роль. Від електронного управління двигуном до автономних систем водіння – кожен елемент автомобіля сьогодні є частиною складного цифрового організму, що постійно аналізує, оптимізує та приймає рішення в реальному часі.

Однією з найважливіших переваг вбудованих систем є підвищення безпеки. Антиблокувальні гальмівні системи, електронна стабілізація, адаптивний круїз-контроль та автоматичне екстрене гальмування значно знижують ризик аварій, компенсуючи можливі помилки водія. Крім того, сучасні системи управління двигуном і трансмісією забезпечують оптимальну продуктивність при мінімальній витраті палива та шкідливих викидах, що робить автомобілі екологічнішими.

Розвиток інформаційно-розважальних систем і цифрових інтерфейсів покращив комфорт користувачів, дозволивши інтегрувати навігацію, мультимедіа та зв’язок із зовнішніми пристроями. Водночас, поява мереж передачі даних (CAN, LIN, FlexRay, Automotive Ethernet) забезпечила надійну взаємодію між усіма електронними компонентами автомобіля.

Майбутнє автомобільної індустрії пов’язане з повною автономізацією. Вже зараз тестуються системи 4-го та 5-го рівнів автономності, де штучний інтелект, потужні сенсори та нейромережі дозволяють автомобілю самостійно оцінювати дорожню обстановку та приймати рішення. Однак із зростанням складності електронних систем виникають нові виклики, такі як кібербезпека та надійність програмного забезпечення.

Отже, вбудовані системи управління автомобілями не лише змінили принципи функціонування транспортних засобів, але й відкрили шлях до повністю автономного майбутнього. Їхній подальший розвиток обіцяє ще більшу ефективність, безпеку та інтелектуальність, роблячи автомобілі не просто засобами пересування, а справжніми учасниками цифрової екосистеми.
	
\end{document}