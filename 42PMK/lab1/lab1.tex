\documentclass[oneside,14pt]{extarticle}
\usepackage{cmap}
\usepackage[utf8]{inputenc}
\usepackage{longtable}
\usepackage[english,ukrainian]{babel}
\usepackage{graphicx}
\usepackage{geometry}
\usepackage{listings}
\usepackage{float}
\usepackage{amsmath}
\usepackage{subfig}
\usepackage{tempora}
\renewcommand{\arraystretch}{1.5}
\usepackage{mwe}
\geometry{
	a4paper,
	left=20mm,
	right=20mm,
	top=15mm,
	bottom=15mm,
}
\lstset{
	language=c,
	tabsize=4,
	keepspaces,
	showstringspaces=false,
	frame=single,
	breaklines,
	language=C,
}
\graphicspath{ {./pictures} }
\setlength{\parindent}{4em}

\newcommand\subject{Програмування мікроконтролерів}
\newcommand\lecturer{доцент кафедри ПЗ\\Марусенкова Т.А.}
\newcommand\teacher{асистент кафедри ПЗ\\Щур В.І.}
\newcommand\mygroup{ПЗ-42}
\newcommand\lab{1}
\newcommand\theme{Робота з EEPROM}
\newcommand\purpose{Навчитися організовувати взаємодію між мікроконтролером і чіпом EEPROM за інтерфейсом І2С, а також закріпити навички роботи з технічною
документацією}

\begin{document}
\begin{normalsize}
	\begin{titlepage}
		\thispagestyle{empty}
		\begin{center}
			\textbf{МІНІСТЕРСТВО ОСВІТИ І НАУКИ УКРАЇНИ\\
				НАЦІОНАЛЬНИЙ УНІВЕРСИТЕТ "ЛЬВІВСЬКА ПОЛІТЕХНІКА"}
		\end{center}
		\begin{flushright}
			\textbf{ІКНІ}\\
			Кафедра \textbf{ПЗ}
		\end{flushright}
		\vspace{80pt}
		\begin{center}
			\textbf{ЗВІТ}\\
			\vspace{10pt}
			до лабораторної роботи № \lab\\
			\textbf{на тему}: <<\textit{\theme}>>\\
			\textbf{з дисципліни}: <<\subject>>
		\end{center}
		\vspace{80pt}
		\begin{flushright}
			
			\textbf{Лектор}:\\
			\lecturer\\
			\vspace{28pt}
			\textbf{Виконав}:\\
			
			студенти групи \mygroup\\
			Коваленко Д.М.\\
			\vspace{28pt}
			\textbf{Прийняв}:\\
			
			\teacher\\
			
			\vspace{28pt}
			«\rule{1cm}{0.15mm}» \rule{1.5cm}{0.15mm} 2025 р.\\
			$\sum$ = \rule{1cm}{0.15mm}……………\\
			
		\end{flushright}
		\vspace{\fill}
		\begin{center}
			\textbf{Львів — 2025}
		\end{center}
	\end{titlepage}
		
	\begin{description}
		\item[Тема.] \theme.
		\item[Мета.] \purpose.
	\end{description}

  \section*{Лабораторне завдання}
  \begin{enumerate}
  	\item Ініціалізувати інтерфейсну схему, визначену індивідуальним варіантом, а
також два виводи GPIO з доступного банку виводів. Наявні банки виводів
подивитися в документації на мікроконтролер stm32f407vg.
  \item Підключити плату розширення зі встановленим чіпом EEPROM до
ініціалізованих виводів (можливо виконати лише в лабораторії).
  \item Проаналізувати технічну документацію мікросхеми EEPROM (файл
Microchip 24AA256/24LC256/24FC256 на ВНС) та зчитати основні
характеристики цієї моделі.
  \item Проаналізувати наданий для прикладу проект і адаптувати його до даної
моделі EEPROM (опційний пункт, можна писати коди повністю
самостійно, «з нуля»).
  \item Реалізувати читання та запис одного байту та N байтів поспіль.
Використати бібліотеку, визначену індивідуальним завданням. Варіант
індивідуального завдання співпадає з номером студента в підгрупі.
  \item Оформити звіт. В розділі «Теоретичні відомості» дати відповідь на
контрольне запитання, визначене порядковим номером у підгрупі. В звіті
навести ту частину коду проекту, що відповідає за ініціалізацію модуля
І2С, портів вводу/виводу, адресацію пристрою, функції читання/запису та
індивідуальне завдання. При захисті демонструється цілий проект в
середовищі програмування.
  \end{enumerate}
  
  I2C3; SPL; N = 13, початок читання – байт 10 блока №5
  
  \section*{Хід роботи}
  
  Код програми індивідуального варіанту, main.c:
  
	{\small\begin{lstlisting}
#define EEPROM_ADDRESS  0xA0
#define EEPROM_PAGE_SIZE 64
#define READ_START_ADDR (5 * EEPROM_PAGE_SIZE + 10)  // 10th byte of 5th block

static void MX_I2C3_Init(void)
{
  hi2c3.Instance = I2C3;
  hi2c3.Init.ClockSpeed = 100000;
  hi2c3.Init.DutyCycle = I2C_DUTYCYCLE_2;
  hi2c3.Init.OwnAddress1 = 0;
  hi2c3.Init.AddressingMode = I2C_ADDRESSINGMODE_7BIT;
  hi2c3.Init.DualAddressMode = I2C_DUALADDRESS_DISABLE;
  hi2c3.Init.OwnAddress2 = 0;
  hi2c3.Init.GeneralCallMode = I2C_GENERALCALL_DISABLE;
  hi2c3.Init.NoStretchMode = I2C_NOSTRETCH_DISABLE;
  if (HAL_I2C_Init(&hi2c3) != HAL_OK)
  {
    Error_Handler();
  }
}

void I2C3_WriteByte(uint16_t addr, uint8_t data) {
    uint8_t buffer[2] = {addr, data};
    HAL_I2C_Master_Transmit(&hi2c3, EEPROM_ADDRESS, buffer, 2, HAL_MAX_DELAY);
}

uint8_t I2C3_ReadByte(uint16_t addr) {
    uint8_t data;
    HAL_I2C_Master_Transmit(&hi2c3, EEPROM_ADDRESS, (uint8_t*)&addr, 1, HAL_MAX_DELAY);
    HAL_I2C_Master_Receive(&hi2c3, EEPROM_ADDRESS, &data, 1, HAL_MAX_DELAY);
    return data;
}

void I2C3_ReadNBytes(uint16_t addr, uint8_t *buffer, uint16_t len) {
    HAL_I2C_Master_Transmit(&hi2c3, EEPROM_ADDRESS, (uint8_t*)&addr, 1, HAL_MAX_DELAY);
    HAL_I2C_Master_Receive(&hi2c3, EEPROM_ADDRESS, buffer, len, HAL_MAX_DELAY);
}
	
uint8_t writeData = 0xAB;
I2C3_WriteByte(READ_START_ADDR, writeData);

uint8_t readData = I2C3_ReadByte(READ_START_ADDR);

uint8_t buffer[13];
I2C3_ReadNBytes(READ_START_ADDR, buffer, 13);\end{lstlisting}}
	
	Код програми для запуску на мікроконтролері з використанням EEPROM 24AA02E48T, main.c:
  
	{\small\begin{lstlisting}
#define EEPROM_ADDRESS  0xA0
#define EEPROM_PAGE_SIZE 64
#define READ_START_ADDR (5 * EEPROM_PAGE_SIZE + 10)  // 10th byte of 5th block

static void MX_I2C1_Init(void)
{
  hi2c1.Instance = I2C1;
  hi2c1.Init.ClockSpeed = 100000;
  hi2c1.Init.DutyCycle = I2C_DUTYCYCLE_2;
  hi2c1.Init.OwnAddress1 = 0;
  hi2c1.Init.AddressingMode = I2C_ADDRESSINGMODE_7BIT;
  hi2c1.Init.DualAddressMode = I2C_DUALADDRESS_DISABLE;
  hi2c1.Init.OwnAddress2 = 0;
  hi2c1.Init.GeneralCallMode = I2C_GENERALCALL_DISABLE;
  hi2c1.Init.NoStretchMode = I2C_NOSTRETCH_DISABLE;
  if (HAL_I2C_Init(&hi2c1) != HAL_OK)
  {
    Error_Handler();
  }
}

void I2C1_WriteByte(uint16_t addr, uint8_t data) {
    uint8_t buffer[2] = {addr, data};
    HAL_I2C_Master_Transmit(&hi2c3, EEPROM_ADDRESS, buffer, 2, HAL_MAX_DELAY);
}

uint8_t I2C1_ReadByte(uint16_t addr) {
    uint8_t data;
    HAL_I2C_Master_Transmit(&hi2c3, EEPROM_ADDRESS, (uint8_t*)&addr, 1, HAL_MAX_DELAY);
    HAL_I2C_Master_Receive(&hi2c3, EEPROM_ADDRESS, &data, 1, HAL_MAX_DELAY);
    return data;
}

void I2C1_ReadNBytes(uint16_t addr, uint8_t *buffer, uint16_t len) {
    HAL_I2C_Master_Transmit(&hi2c3, EEPROM_ADDRESS, (uint8_t*)&addr, 1, HAL_MAX_DELAY);
    HAL_I2C_Master_Receive(&hi2c3, EEPROM_ADDRESS, buffer, len, HAL_MAX_DELAY);
}
	
uint8_t writeData = 0xAB;
I2C1_WriteByte(READ_START_ADDR, writeData);

uint8_t readData = I2C1_ReadByte(READ_START_ADDR);

uint8_t buffer[13];
I2C1_ReadNBytes(READ_START_ADDR, buffer, 13);\end{lstlisting}}
	
	\section*{Висновки}
	Під час виконання лабораторної роботи, я навчився організовувати взаємодію між мікроконтролером і чіпом EEPROM за допомогою інтерфейсу І2С, а також закріпив навички роботи з технічною документацією.
	    
\end{normalsize}
\end{document}
