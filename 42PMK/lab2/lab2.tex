\documentclass[oneside,14pt]{extarticle}
\usepackage{cmap}
\usepackage[utf8]{inputenc}
\usepackage{longtable}
\usepackage[english,ukrainian]{babel}
\usepackage{graphicx}
\usepackage{geometry}
\usepackage{listings}
\usepackage{float}
\usepackage{amsmath}
\usepackage{subfig}
\usepackage{tempora}
\renewcommand{\arraystretch}{1.5}
\usepackage{mwe}
\geometry{
	a4paper,
	left=20mm,
	right=20mm,
	top=15mm,
	bottom=15mm,
}
\lstset{
	language=c,
	tabsize=4,
	keepspaces,
	showstringspaces=false,
	frame=single,
	breaklines,
	language=C,
}
\graphicspath{ {./pictures} }
\setlength{\parindent}{4em}

\newcommand\subject{Програмування мікроконтролерів}
\newcommand\lecturer{доцент кафедри ПЗ\\Марусенкова Т.А.}
\newcommand\teacher{асистент кафедри ПЗ\\Щур В.І.}
\newcommand\mygroup{ПЗ-42}
\newcommand\lab{2}
\newcommand\theme{Робота з Flash-пам'яттю на прикладі SST25VF016B}
\newcommand\purpose{Навчитися організовувати взаємодію між мікроконтролером та Flash-
пам’яттю SST25VF016B за інтерфейсом SPI, а також закріпити навички роботи з
технічною документацією}

\begin{document}
\begin{normalsize}
	\begin{titlepage}
		\thispagestyle{empty}
		\begin{center}
			\textbf{МІНІСТЕРСТВО ОСВІТИ І НАУКИ УКРАЇНИ\\
				НАЦІОНАЛЬНИЙ УНІВЕРСИТЕТ "ЛЬВІВСЬКА ПОЛІТЕХНІКА"}
		\end{center}
		\begin{flushright}
			\textbf{ІКНІ}\\
			Кафедра \textbf{ПЗ}
		\end{flushright}
		\vspace{80pt}
		\begin{center}
			\textbf{ЗВІТ}\\
			\vspace{10pt}
			до лабораторної роботи № \lab\\
			\textbf{на тему}: <<\textit{\theme}>>\\
			\textbf{з дисципліни}: <<\subject>>
		\end{center}
		\vspace{80pt}
		\begin{flushright}
			
			\textbf{Лектор}:\\
			\lecturer\\
			\vspace{28pt}
			\textbf{Виконав}:\\
			
			студенти групи \mygroup\\
			Коваленко Д.М.\\
			\vspace{28pt}
			\textbf{Прийняв}:\\
			
			\teacher\\
			
			\vspace{28pt}
			«\rule{1cm}{0.15mm}» \rule{1.5cm}{0.15mm} 2025 р.\\
			$\sum$ = \rule{1cm}{0.15mm}……………\\
			
		\end{flushright}
		\vspace{\fill}
		\begin{center}
			\textbf{Львів — 2025}
		\end{center}
	\end{titlepage}
		
	\begin{description}
		\item[Тема.] \theme.
		\item[Мета.] \purpose.
	\end{description}

  \section*{Лабораторне завдання}
  \begin{enumerate}
  	\item Адаптувати наведені коди для однієї з перелічених бібліотек: SPL, HAL або
CMSIS Driver. Використати бібліотеку згідно з індивідуальним завданням.
Згідно зі схемою підключення використано саме SPI1, однак теоретично міг
би бути застосований і інший модуль. Адаптувати коди для іншого модуля,
визначеного індивідуальним варіантом. Номер індивідуального варіанту
співпадає з номером студента в підгрупі. Таким чином, має бути дві версії
кодів: для SPI1 (що може бути відтестована на реальній платі) та для
деякого іншого модуля (що перевіряється лише інспекцією коду).
  \item Здійснити читання даних з регістру Статусу для перевірки працездатності
адаптованих згідно індивідуального завдання функцій.
  \item Встановити відповідний режим роботи Flash-пам'яті.
  \item Виконати дії, що описані у пункті 2. У якості підтвердження успішного
встановлення режиму роботи Flash-пам'яті додати у звіт знімки екрану на
при виконанні п.2 та п.4.
  \item Перевірити правильність запису даних, а також їх наявність та цілісність
даних після повторного увімкнення живлення плати.
  \item Оформити звіт. В розділі «Теоретичні відомості» дати відповідь на
контрольне запитання, визначене порядковим номером у підгрупі. В звіті
навести ту частину коду проекту, що відповідає за ініціалізацію модуля SPI
(у двох версіях), портів вводу/виводу, адресацію пристрою, функції
читання/запису та індивідуальне завдання. При захисті демонструється
цілий проект в середовищі програмування.
  \end{enumerate}
  
  AAI – Yes; CMSIS Driver; N = 10; SPI2
  
  \section*{Хід роботи}
  
  Код програми, main.c:
  
	{\small\begin{lstlisting}
#include "Driver_SPI.h"

extern ARM_DRIVER_SPI Driver_SPI1;
extern ARM_DRIVER_SPI Driver_SPI2;

static ARM_DRIVER_SPI *hspi1 = &Driver_SPI1;
static ARM_DRIVER_SPI *hspi2 = &Driver_SPI2;

void SPI_Init(void) {
    hspi1->Initialize(NULL);
    hspi1->PowerControl(ARM_POWER_FULL);
    hspi1->Control(ARM_SPI_MODE_MASTER | ARM_SPI_CPOL1_CPHA1 | ARM_SPI_MSB_LSB, 1000000);
    hspi1->Control(ARM_SPI_CONTROL_SS, ARM_SPI_SS_INACTIVE);

    hspi2->Initialize(NULL);
    hspi2->PowerControl(ARM_POWER_FULL);
    hspi2->Control(ARM_SPI_MODE_MASTER | ARM_SPI_CPOL1_CPHA1 | ARM_SPI_MSB_LSB, 1000000);
    hspi2->Control(ARM_SPI_CONTROL_SS, ARM_SPI_SS_INACTIVE);
}

uint8_t SPI2_TransferByte(uint8_t data) {
    uint8_t rx_data = 0;
    hspi2->Transfer(&data, &rx_data, 1);
    while (hspi2->GetStatus().busy);
    return rx_data;
}

void SPI2_EnableCS(void) {
    hspi2->Control(ARM_SPI_CONTROL_SS, ARM_SPI_SS_ACTIVE);
}

void SPI2_DisableCS(void) {
    hspi2->Control(ARM_SPI_CONTROL_SS, ARM_SPI_SS_INACTIVE);
}

int main() {
  SPI_Init();
  SPI2_EnableCS();q
  uint8_t status = SPI2_TransferByte(0x05);
  uint8_t data = SPI2_TransferByte(0x00);
  SPI2_DisableCS();
}\end{lstlisting}}
	
	\section*{Висновки}
	Під час виконання лабораторної роботи, я навчився організовувати взаємодію між мікроконтролером та Flash-
пам’яттю SST25VF016B за інтерфейсом SPI, а також закріпив навички роботи з
технічною документацією.
	    
\end{normalsize}
\end{document}
