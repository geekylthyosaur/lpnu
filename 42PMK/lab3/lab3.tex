\documentclass[oneside,14pt]{extarticle}
\usepackage{cmap}
\usepackage[utf8]{inputenc}
\usepackage{longtable}
\usepackage[english,ukrainian]{babel}
\usepackage{graphicx}
\usepackage{geometry}
\usepackage{listings}
\usepackage{float}
\usepackage{amsmath}
\usepackage{subfig}
\usepackage{tempora}
\renewcommand{\arraystretch}{1.5}
\usepackage{mwe}
\geometry{
	a4paper,
	left=20mm,
	right=20mm,
	top=15mm,
	bottom=15mm,
}
\lstset{
	language=c,
	tabsize=4,
	keepspaces,
	showstringspaces=false,
	frame=single,
	breaklines,
	language=C,
}
\graphicspath{ {./pictures} }
\setlength{\parindent}{4em}

\newcommand\subject{Програмування мікроконтролерів}
\newcommand\lecturer{доцент кафедри ПЗ\\Марусенкова Т.А.}
\newcommand\teacher{асистент кафедри ПЗ\\Щур В.І.}
\newcommand\mygroup{ПЗ-42}
\newcommand\lab{3}
\newcommand\theme{Робота з інтерфейсом CAN на прикладі мікроконтролера STM32F407}
\newcommand\purpose{Навчитись створювати зв'язок між кількома мікроконтролерами STM32
через інтерфейс CAN}

\begin{document}
\begin{normalsize}
	\begin{titlepage}
		\thispagestyle{empty}
		\begin{center}
			\textbf{МІНІСТЕРСТВО ОСВІТИ І НАУКИ УКРАЇНИ\\
				НАЦІОНАЛЬНИЙ УНІВЕРСИТЕТ "ЛЬВІВСЬКА ПОЛІТЕХНІКА"}
		\end{center}
		\begin{flushright}
			\textbf{ІКНІ}\\
			Кафедра \textbf{ПЗ}
		\end{flushright}
		\vspace{80pt}
		\begin{center}
			\textbf{ЗВІТ}\\
			\vspace{10pt}
			до лабораторної роботи № \lab\\
			\textbf{на тему}: <<\textit{\theme}>>\\
			\textbf{з дисципліни}: <<\subject>>
		\end{center}
		\vspace{80pt}
		\begin{flushright}
			
			\textbf{Лектор}:\\
			\lecturer\\
			\vspace{28pt}
			\textbf{Виконав}:\\
			
			студенти групи \mygroup\\
			Коваленко Д.М.\\
			\vspace{28pt}
			\textbf{Прийняв}:\\
			
			\teacher\\
			
			\vspace{28pt}
			«\rule{1cm}{0.15mm}» \rule{1.5cm}{0.15mm} 2025 р.\\
			$\sum$ = \rule{1cm}{0.15mm}……………\\
			
		\end{flushright}
		\vspace{\fill}
		\begin{center}
			\textbf{Львів — 2025}
		\end{center}
	\end{titlepage}
		
	\begin{description}
		\item[Тема.] \theme.
		\item[Мета.] \purpose.
	\end{description}

  \section*{Лабораторне завдання}
  \begin{enumerate}
  	\item Використовуючи надані приклади кодів, створити проект з використанням
заданої бібліотеки, в якому в мережу CAN надсилається повідомлення, що
містить Ваше прізвище та ім'я при настанні умови, визначеної
індивідуальним завданням.
\item Оформити звіт. В розділі «Теоретичні відомості» дати відповідь на
контрольне запитання, визначене порядковим номером у підгрупі. В звіті
навести ту частину коду проекту, що відповідає безпосередньо за виконання
завдання (бібліотечні коди можна не наводити).
  \end{enumerate}
  
  3. подвійне натиснення кнопки, SPL
  
  \section*{Теоретичні відомості}
  
  CAN\_InitStructure.CAN\_AWUM — це поле в структурі ініціалізації CAN (Controller Area Network) для мікроконтролерів STM32, яке визначає, чи буде автоматично відбуватися "пробудження" контролера CAN при отриманні повідомлення.
  
  

    AWUM (Automatic Wake-Up Mode)

    Якщо CAN\_AWUM = 1, контролер CAN автоматично виходить із режиму "сплячого режиму" (Sleep Mode) при отриманні повідомлення на шині CAN.

    Якщо CAN\_AWUM = 0, пробудження повинно виконуватися вручну через програмне втручання.

Це корисно для енергозбереження, коли мікроконтролер переводить CAN-периферію в Sleep Mode і хоче автоматично реагувати на трафік у мережі.
  
  \section*{Хід роботи}
  
  Код програми, main.c:
  
	{\small\begin{lstlisting}
uint8_t button_press_count = 0;
CAN_HandleTypeDef hcan1;

void HAL_GPIO_EXTI_Callback(uint16_t GPIO_Pin) {
    if (GPIO_Pin == GPIO_PIN_15) {
        button_press_count++;

        if (button_press_count % 2 == 0) {
            uint8_t message[] = "Dmytro Kovalenko";
            CAN_TxHeaderTypeDef TxHeader;
            uint32_t TxMailbox;

            TxHeader.StdId = 0x0;
            TxHeader.IDE = CAN_ID_STD;
            TxHeader.RTR = CAN_RTR_DATA;
            TxHeader.DLC = sizeof(message) - 1;

            if (HAL_CAN_AddTxMessage(&hcan1, &TxHeader, message, &TxMailbox) == HAL_OK) {
                // Success
            }
        }
    }
}

static void MX_CAN1_Init(void)
{
  hcan1.Instance = CAN1;
  hcan1.Init.Prescaler = 16;
  hcan1.Init.Mode = CAN_MODE_NORMAL;
  hcan1.Init.SyncJumpWidth = CAN_SJW_1TQ;
  hcan1.Init.TimeSeg1 = CAN_BS1_1TQ;
  hcan1.Init.TimeSeg2 = CAN_BS2_1TQ;
  hcan1.Init.TimeTriggeredMode = DISABLE;
  hcan1.Init.AutoBusOff = DISABLE;
  hcan1.Init.AutoWakeUp = DISABLE;
  hcan1.Init.AutoRetransmission = DISABLE;
  hcan1.Init.ReceiveFifoLocked = DISABLE;
  hcan1.Init.TransmitFifoPriority = DISABLE;
  if (HAL_CAN_Init(&hcan1) != HAL_OK)
  {
    Error_Handler();
  }

}\end{lstlisting}}
	
	\section*{Висновки}
	Під час виконання лабораторної роботи, я навчився створювати зв'язок між кількома мікроконтролерами STM32 через інтерфейс CAN.
	    
\end{normalsize}
\end{document}
