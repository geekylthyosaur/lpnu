\documentclass[oneside,14pt]{extarticle}
\usepackage{cmap}
\usepackage[utf8]{inputenc}
\usepackage{longtable}
\usepackage[english,ukrainian]{babel}
\usepackage{graphicx}
\usepackage{geometry}
\usepackage{listings}
\usepackage{float}
\usepackage{amsmath}
\usepackage{subfig}
\usepackage{tempora}
\renewcommand{\arraystretch}{1.5}
\usepackage{mwe}
\geometry{
	a4paper,
	left=20mm,
	right=20mm,
	top=15mm,
	bottom=15mm,
}
\lstset{
	language=c,
	tabsize=4,
	keepspaces,
	showstringspaces=false,
	frame=single,
	breaklines,
	language=C,
}
\graphicspath{ {./pictures} }
\setlength{\parindent}{4em}

\newcommand\subject{Програмування мікроконтролерів}
\newcommand\lecturer{доцент кафедри ПЗ\\Марусенкова Т.А.}
\newcommand\teacher{асистент кафедри ПЗ\\Щур В.І.}
\newcommand\mygroup{ПЗ-42}
\newcommand\lab{4}
\newcommand\theme{Робота з Ethernet на базі мікроконтролера STM32F407}
\newcommand\purpose{Навчитись створювати мережеве підключення з використанням Ethernet}

\begin{document}
\begin{normalsize}
	\begin{titlepage}
		\thispagestyle{empty}
		\begin{center}
			\textbf{МІНІСТЕРСТВО ОСВІТИ І НАУКИ УКРАЇНИ\\
				НАЦІОНАЛЬНИЙ УНІВЕРСИТЕТ "ЛЬВІВСЬКА ПОЛІТЕХНІКА"}
		\end{center}
		\begin{flushright}
			\textbf{ІКНІ}\\
			Кафедра \textbf{ПЗ}
		\end{flushright}
		\vspace{80pt}
		\begin{center}
			\textbf{ЗВІТ}\\
			\vspace{10pt}
			до лабораторної роботи № \lab\\
			\textbf{на тему}: <<\textit{\theme}>>\\
			\textbf{з дисципліни}: <<\subject>>
		\end{center}
		\vspace{80pt}
		\begin{flushright}
			
			\textbf{Лектор}:\\
			\lecturer\\
			\vspace{28pt}
			\textbf{Виконав}:\\
			
			студенти групи \mygroup\\
			Коваленко Д.М.\\
			\vspace{28pt}
			\textbf{Прийняв}:\\
			
			\teacher\\
			
			\vspace{28pt}
			«\rule{1cm}{0.15mm}» \rule{1.5cm}{0.15mm} 2025 р.\\
			$\sum$ = \rule{1cm}{0.15mm}……………\\
			
		\end{flushright}
		\vspace{\fill}
		\begin{center}
			\textbf{Львів — 2025}
		\end{center}
	\end{titlepage}
		
	\begin{description}
		\item[Тема.] \theme.
		\item[Мета.] \purpose.
	\end{description}

  \section*{Лабораторне завдання}
  \begin{enumerate}
  	\item Використовуючи приклади коду, що представлені у методичних вказівках адаптувати надані
коди для відповідної однієї з перелічених бібліотек: SPL, HAL або CMSIS Driver.
  \item Написати програму, що буде реалізовувати індивідуальне завдання згідно варіанту:
- Створити TCP сервер для роботи з N клієнтами; HAL.
- Створити TCP клієнт з використання HAL. Здійснювати надсилання різних повідомлень при
натисканні на відповідну клавішу.
- Створити HTTP сервер з використанням бібліотеки HAL.
  \end{enumerate}
  
  \section*{Хід роботи}
  
Код програми, tcp\_client.c:
  
{\small\begin{lstlisting}
#include "tcp_client.h"
#include "string.h"

static struct tcp_pcb *tcp_client_pcb;
static ip_addr_t server_ip = IPADDR4_INIT_BYTES(1, 1, 1, 1);
#define SERVER_PORT 80

void tcp_client_init(void) {
	tcp_client_pcb = tcp_new();
	if (tcp_client_pcb != NULL) {
		tcp_connect(tcp_client_pcb, &server_ip, SERVER_PORT, NULL);
	}
}

void tcp_client_send(const char *message) {
	if (tcp_client_pcb != NULL) {
		struct pbuf *p = pbuf_alloc(PBUF_TRANSPORT, strlen(message), PBUF_RAM);
		if (p != NULL) {
			memcpy(p->payload, message, strlen(message));
			tcp_write(tcp_client_pcb, p->payload, p->len, TCP_WRITE_FLAG_COPY);
			pbuf_free(p);
		}
	}
}\end{lstlisting}}

Код програми, tcp\_server.c:
  
{\small\begin{lstlisting}
#include "tcp_server.h"

static struct tcp_pcb *tcp_server_pcb;
static struct tcp_pcb *client_pcbs[MAX_CLIENTS];

void tcp_server_init(void) {
    tcp_server_pcb = tcp_new();
    if (tcp_server_pcb != NULL) {
        err_t err = tcp_bind(tcp_server_pcb, IP_ADDR_ANY, TCP_PORT);
        if (err == ERR_OK) {
            tcp_server_pcb = tcp_listen(tcp_server_pcb);
            tcp_accept(tcp_server_pcb, tcp_server_accept);
        }
    }
}

err_t tcp_server_accept(void *arg, struct tcp_pcb *newpcb, err_t err) {
    for (int i = 0; i < MAX_CLIENTS; i++) {
        if (client_pcbs[i] == NULL) {
            client_pcbs[i] = newpcb;
            tcp_arg(newpcb, (void*)(intptr_t)i);
            tcp_recv(newpcb, tcp_server_recv);
            return ERR_OK;
        }
    }
    tcp_close(newpcb);
    return ERR_CONN;
}

err_t tcp_server_recv(void *arg, struct tcp_pcb *tpcb, struct pbuf *p, err_t err) {
    if (p != NULL) {
    	// TODO
        tcp_recved(tpcb, p->tot_len);
        pbuf_free(p);
    } else if (err == ERR_OK) {
        int client_id = (int)(intptr_t)arg;
        client_pcbs[client_id] = NULL;
        tcp_close(tpcb);
    }
    return ERR_OK;
}\end{lstlisting}}

Код програми, http\_server.c:
  
{\small\begin{lstlisting}
#include "http_server.h"

#define HTTP_PORT 80

static const char http_header[] = "HTTP/1.1 200 OK\r\nContent-type: text/html\r\n\r\n";
static const char http_body[] = "Hello world";

static err_t http_server_recv(void *arg, struct tcp_pcb *pcb, struct pbuf *p, err_t err) {
    if (p != NULL) {
        tcp_recved(pcb, p->tot_len);
        tcp_write(pcb, http_header, sizeof(http_header)-1, TCP_WRITE_FLAG_COPY);
        tcp_write(pcb, http_body, sizeof(http_body)-1, TCP_WRITE_FLAG_COPY);
        pbuf_free(p);
    }
    return ERR_OK;
}

static err_t http_server_accept(void *arg, struct tcp_pcb *pcb, err_t err) {
    tcp_recv(pcb, http_server_recv);
    return ERR_OK;
}

void http_server_init(void) {
    struct tcp_pcb *pcb = tcp_new();
    tcp_bind(pcb, IP_ADDR_ANY, HTTP_PORT);
    pcb = tcp_listen(pcb);
    tcp_accept(pcb, http_server_accept);
}\end{lstlisting}}
	
	\section*{Висновки}
	Під час виконання лабораторної роботи, я навчився створювати мережеве підключення з використанням Ethernet.
	    
\end{normalsize}
\end{document}
