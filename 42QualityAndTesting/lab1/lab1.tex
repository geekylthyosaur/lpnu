\documentclass[oneside,14pt]{extarticle}
\usepackage{cmap}
\usepackage[utf8]{inputenc}
\usepackage{longtable}
\usepackage[english,ukrainian]{babel}
\usepackage{graphicx}
\usepackage{geometry}
\usepackage{listings}
\usepackage{float}
\usepackage{amsmath}
\usepackage{subfig}
\usepackage{tempora}
\geometry{
	a4paper,
	left=20mm,
	right=20mm,
	top=15mm,
	bottom=15mm,
}
\lstset{
	language=c,
	tabsize=4,
	keepspaces,
	showstringspaces=false,
	frame=single,
	breaklines,
	language=C,
}
\graphicspath{ {./pictures} }
\setlength{\parindent}{4em}

\newcommand\subject{Якість програмного забезпечення та тестування}
\newcommand\lecturer{доцент кафедри ПЗ\\Фоменко А.В.}
\newcommand\teacher{асистент кафедри ПЗ\\Джумеля Е.А.}
\newcommand\mygroup{ПЗ-42}
\newcommand\lab{1}
\newcommand\theme{Види тестування. Планування тестування}
\newcommand\purpose{Вивчити класифікацію видів тестування, розробити перевірки для
	різних видів тестування, навчитися планувати тестові активності в залежності
	від особливостей продукції, що поставляється на тестування
	функціональності}

\begin{document}
\begin{normalsize}
	\begin{titlepage}
		\thispagestyle{empty}
		\begin{center}
			\textbf{МІНІСТЕРСТВО ОСВІТИ І НАУКИ УКРАЇНИ\\
				НАЦІОНАЛЬНИЙ УНІВЕРСИТЕТ "ЛЬВІВСЬКА ПОЛІТЕХНІКА"}
		\end{center}
		\begin{flushright}
			\textbf{ІКНІ}\\
			Кафедра \textbf{ПЗ}
		\end{flushright}
		\vspace{80pt}
		\begin{center}
			\textbf{ЗВІТ}\\
			\vspace{10pt}
			до лабораторної роботи № \lab\\
			\textbf{на тему}: <<\textit{\theme}>>\\
			\textbf{з дисципліни}: <<\subject>>
		\end{center}
		\vspace{80pt}
		\begin{flushright}
			
			\textbf{Лектор}:\\
			\lecturer\\
			\vspace{28pt}
			\textbf{Виконав}:\\
			
			студент групи \mygroup\\
			Коваленко Д.М.\\
			\vspace{28pt}
			\textbf{Прийняла}:\\
			
			\teacher\\
			
			\vspace{28pt}
			«\rule{1cm}{0.15mm}» \rule{1.5cm}{0.15mm} 2024 р.\\
			$\sum$ = \rule{1cm}{0.15mm}……………\\
			
		\end{flushright}
		\vspace{\fill}
		\begin{center}
			\textbf{Львів — 2024}
		\end{center}
	\end{titlepage}
		
	\begin{description}
		\item[Тема.] \theme.
		\item[Мета.] \purpose.
	\end{description}

    \section*{Лабораторне завдання}
    \begin{enumerate}
    	\item Обрати будь-якій об’єкт реального світу і створити інтуїтивний план
    	його тестування за зразком.
    	\item 1. Виберіть будь-якій електронний магазин (ROZETKA, Comfy, Foxtrot,
    	Agromarket, Stylus, Allo, тощо) з метою подальшої розробки тестових
    	перевірок для нього.
    	\item Описати, що і яким чином, виходячи з видів і типів тестування, Ви би
    	тестували, вносить в таблицю назви і короткій опис різних перевірок,
    	відповідно до класифікації видів тестування для вибраного існуючого
    	електронного магазину.
    	\item Оформити звіт і захистити лабораторну роботу.
    \end{enumerate}
	\section*{Хід роботи}
	
	Об'єкт тестування: годинник.
	
	\begin{longtable}{|p{4.5cm}|p{5cm}|p{7cm}|}
		\hline
		\textbf{Вид тестування} & \textbf{Короткий опис виду тестування} & \textbf{Тестові перевірки} \\
		\hline
		\textbf{Functional Testing} & Тестування функціональності механізму згідно з його специфікацією & Перевірка точності відображення часу; перевірка чи заводить годинник пружину і чи рухаються стрілки після заведення. \\
		\hline
		\textbf{Safety Testing} & Перевірка здатності годинника залишатися безпечним для користувача & Перевірка чи механізм безпечний для шкіри (немає гострих частин), чи годинник не розбивається при легкому падінні. \\
		\hline
		\textbf{Security Testing} & Тестування стійкості до зовнішніх впливів і довговічності & Перевірка водонепроникності годинника; стійкість до перепадів температур і фізичних пошкоджень. \\
		\hline
		\textbf{Compatibility Testing} & Перевірка чи зручний годинник для різних типів зап’ястя і рук & Перевірка чи регулюється ремінець для різних розмірів руки, чи підходить для використання як чоловіками, так і жінками. \\
		\hline
		\textbf{GUI Testing} & Тестування, що виконується шляхом взаємодії з системою через графічний інтерфейс користувача & Перевірка наскільки чітко і зрозуміло відображається час на циферблаті; перевірка зручності керування кнопками на корпусі годинника. \\
		\hline
		\textbf{Usability Testing} & Тестування зручності використання годинника користувачем & Перевірка наскільки легко заводити годинник, регулювати час, надягати на руку та знімати. \\
		\hline
		\textbf{Accessibility Testing} & Тестування, яке визначає ступінь легкості, з якою користувачі з обмеженими здібностями можуть використовувати годинник & Перевірка чи годинник зручний для використання особами з обмеженими можливостями (наприклад, візуально ослабленими). \\
		\hline
		\textbf{Internationalization Testing} & Тестування адаптації продукту до мовних і культурних особливостей регіонів, де він буде використовуватися & Перевірка чи годинник не має символів або написів, що можуть бути неприйнятними в певних культурних або мовних контекстах. \\
		\hline
		\textbf{Performance Testing} & Тестування продуктивності і точності годинника & Перевірка наскільки довго годинник може працювати після одного заводження, а також точність ходу протягом доби. \\
		\hline
		\textbf{Stress Testing} & Тестування годинника при екстремальних умовах & Перевірка роботи годинника при сильних ударах, різких перепадах температур або при зануренні у воду. \\
		\hline
		\textbf{Negative Testing} & Тестування роботи годинника при некоректних діях & Перевірка як працює годинник при неправильному заводженні або при надмірному тиску на заводний механізм. \\
		\hline
		\textbf{Black Box Testing} & Тестування без знання внутрішньої структури механізму & Перевірка чи правильно показує час, не дивлячись на те, як влаштований механізм. \\
		\hline
		\textbf{Automated Testing} & Автоматизовані тести з метою виключити людський фактор & Автоматизоване заводження годинника кілька разів і перевірка точності ходу після кожного циклу. \\
		\hline
		\textbf{Unit/Component Testing} & Тестування окремих частин механізму & Перевірка точності ходу головної пружини; перевірка роботи механізму стрілок; перевірка заводного механізму на зносостійкість. \\
		\hline
		\textbf{Integration Testing} & Тестування взаємодії всіх частин механізму & Перевірка наскільки точно механізм працює при зібранні годинника в цілому; перевірка інтеграції між заводним механізмом, стрілками і циферблатом. \\
		\hline
	\end{longtable}
	
	Об'єкт тестування: інтернет магазин COMFY.
	
	\begin{longtable}{|p{4.5cm}|p{5cm}|p{7cm}|}
		\hline
		\textbf{Вид тестування} & \textbf{Короткий опис виду тестування} & \textbf{Тестові перевірки} \\
		\hline
		\textbf{Functional Testing} & Тестування функціональних можливостей інтернет-магазину & Перевірка можливості пошуку товарів, додавання в кошик, оформлення замовлення, оплата.\\
		\hline
		\textbf{Safety Testing} & Оцінка безпеки користування сайтом & Перевірка, чи можна безпечно здійснювати оплату, чи захищені дані користувачів.\\
		\hline
		\textbf{Security Testing} & Тестування захищеності сайту від зовнішніх загроз & Оцінка захисту від DDoS-атак, тестування системи входу з паролем, шифрування даних. \\
		\hline
		\textbf{Compatibility Testing} & Тестування сайту на різних пристроях і браузерах & Перевірка роботи сайту на мобільних пристроях, в різних браузерах (Chrome, Firefox, Safari, Edge). \\
		\hline
		\textbf{GUI Testing} & Тестування графічного інтерфейсу сайту & Перевірка коректного відображення елементів інтерфейсу, зручність навігації по сайту. \\
		\hline
		\textbf{Usability Testing} & Оцінка зручності використання сайту & Перевірка інтуїтивності користування, чи легко знайти товар, чи просто оформити замовлення. \\
		\hline
		\textbf{Accessibility Testing} & Тестування сайту для людей з обмеженими можливостями & Перевірка підтримки екранних читалок, адаптації інтерфейсу для людей з порушенням зору або моторики. \\
		\hline
		\textbf{Internationalization Testing} & Адаптація сайту до різних мов та валют & Перевірка правильного відображення текстів різними мовами, коректність цін в різних валютах. \\
		\hline
		\textbf{Performance Testing} & Перевірка продуктивності сайту при великому навантаженні & Тестування швидкості завантаження сторінок, час обробки замовлень під час акцій та знижок. \\
		\hline
		\textbf{Stress Testing} & Перевірка роботи сайту під великим навантаженням & Оцінка працездатності при збільшеній кількості користувачів, наприклад під час Black Friday. \\
		\hline
		\textbf{Negative Testing} & Тестування некоректного введення даних & Перевірка реакції на некоректні дані при заповненні форм замовлення або введенні неправильної платіжної інформації. \\
		\hline
		\textbf{Black Box Testing} & Тестування без доступу до внутрішньої структури сайту & Перевірка основних функцій: пошук, покупка, оплата, без доступу до коду або баз даних. \\
		\hline
		\textbf{Automated Testing} & Автоматизоване тестування сценаріїв використання сайту & Автоматичне тестування сценаріїв покупки товарів, оформлення доставки, відправки листів на пошту. \\
		\hline
		\textbf{Unit/Component Testing} & Тестування окремих компонентів сайту & Перевірка роботи окремих модулів: пошук, обробка замовлень, система оплати. \\
		\hline
		\textbf{Integration Testing} & Тестування взаємодії між різними модулями сайту & Оцінка, як працюють разом пошук, кошик, оформлення замовлення та система оплати. \\
		\hline
	\end{longtable}
	
	\section*{Висновки}
	ПІд час виконання лабораторної роботи, я вивчив класифікацію видів тестування, розробив перевірки для різних видів тестування, навчився планувати тестові активності в залежності від особливостей продукції, що поставляється на тестування 	функціональності.
	    
\end{normalsize}
\end{document}
