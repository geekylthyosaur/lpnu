\documentclass[oneside,14pt]{extarticle}
\usepackage{cmap}
\usepackage[utf8]{inputenc}
\usepackage{longtable}
\usepackage[english,ukrainian]{babel}
\usepackage{graphicx}
\usepackage{geometry}
\usepackage{listings}
\usepackage{float}
\usepackage{amsmath}
\usepackage{subfig}
\usepackage{tempora}
\renewcommand{\arraystretch}{1.5}
\geometry{
	a4paper,
	left=20mm,
	right=20mm,
	top=15mm,
	bottom=15mm,
}
\lstset{
	language=c,
	tabsize=4,
	keepspaces,
	showstringspaces=false,
	frame=single,
	breaklines,
	language=C,
}
\graphicspath{ {./pictures} }
\setlength{\parindent}{4em}

\newcommand\subject{Архітектура і проєктування програмного забезпечення}
\newcommand\lecturer{доцент кафедри ПЗ\\Фоменко А.В.}
\newcommand\teacher{старший викладач кафедри ПЗ\\Шкраб Р.Р.}
\newcommand\mygroup{ПЗ-42}
\newcommand\lab{1}
\newcommand\theme{Case системи}
\newcommand\purpose{Введення особливо сучасних методів і засобів проектування інформаційних систем, заснованих на використанні CASE-технології}

\begin{document}
\begin{normalsize}
	\begin{titlepage}
		\thispagestyle{empty}
		\begin{center}
			\textbf{МІНІСТЕРСТВО ОСВІТИ І НАУКИ УКРАЇНИ\\
				НАЦІОНАЛЬНИЙ УНІВЕРСИТЕТ "ЛЬВІВСЬКА ПОЛІТЕХНІКА"}
		\end{center}
		\begin{flushright}
			\textbf{ІКНІ}\\
			Кафедра \textbf{ПЗ}
		\end{flushright}
		\vspace{80pt}
		\begin{center}
			\textbf{ЗВІТ}\\
			\vspace{10pt}
			до лабораторної роботи № \lab\\
			\textbf{на тему}: <<\textit{\theme}>>\\
			\textbf{з дисципліни}: <<\subject>>
		\end{center}
		\vspace{80pt}
		\begin{flushright}
			
			\textbf{Лектор}:\\
			\lecturer\\
			\vspace{28pt}
			\textbf{Виконав}:\\
			
			студент групи \mygroup\\
			Коваленко Д.М.\\
			\vspace{28pt}
			\textbf{Прийняла}:\\
			
			\teacher\\
			
			\vspace{28pt}
			«\rule{1cm}{0.15mm}» \rule{1.5cm}{0.15mm} 2024 р.\\
			$\sum$ = \rule{1cm}{0.15mm}……………\\
			
		\end{flushright}
		\vspace{\fill}
		\begin{center}
			\textbf{Львів — 2024}
		\end{center}
	\end{titlepage}
		
	\begin{description}
		\item[Тема.] \theme.
		\item[Мета.] \purpose.
	\end{description}

    \section*{Лабораторне завдання}
    \begin{enumerate}
    	\item Опішить модуль системи по його BPMN2 моделі, виділіть та опішить елементи моделі та їх призначення.
    	\item Опішить модуль системи по його IDEF0 моделі, виділіть та опішить елементи моделі та їх призначення.
    	\item Опішить модуль системи по його IDEF3 моделі, виділіть та опішить елементи моделі та їх призначення.
    	\item Опішить модуль системи по його DFD моделі, виділіть та опішить елементи моделі та їх призначення.
    \end{enumerate}
	\section*{Хід роботи}
	
	\section*{BPMN2 модель для опису модуля}
		BPMN (Business Process Model and Notation) використовується для опису бізнес-процесів у вигляді діаграм.
		
		\textbf{Елементи моделі:}
		\begin{enumerate}
			\item Пул (Pool): Моделює межі модуля системи. Наприклад, пул для "Модуля проектування".
			\item Діяльності (Activities): Представляють основні операції користувачів у модулі. Наприклад, створення UML-діаграми, запуск тесту.
			\item Події (Events): Сигналізують про початок або завершення процесу. Наприклад, подія "Користувач відкрив модуль тестування".
			\item Потоки (Flows): Показують напрямок виконання процесів. Наприклад, з'єднання між "Збереженням діаграми" та "Тестуванням діаграми".
			\item Гейти (Gateways): Відображають логічні рішення або розгалуження процесів. Наприклад, після тестування "Успішно/Не успішно".
		\end{enumerate}
	
	Призначення: 
	BPMN допомагає візуалізувати процес взаємодії користувача з модулем та забезпечити зрозумілу схему виконання дій.
	
	\section*{IDEF0 модель для опису модуля}
	IDEF0 використовується для моделювання функцій та процесів в організаційних системах.
	
	\textbf{Елементи моделі:}
	\begin{enumerate}
		\item Функції (Functions): Основні операції модуля, наприклад, "Створити діаграму", "Запустити тест".
		\item Вхідні дані (Inputs): Дані, необхідні для виконання функцій, наприклад, проект користувача або тестовий сценарій.
		\item Контрольні механізми (Controls): Обмеження або правила, які впливають на виконання функцій. Наприклад, інструкції або навчальні матеріали.
		\item Механізми (Mechanisms): Засоби, що підтримують функції, такі як програмні засоби для проектування та тестування.
	\end{enumerate}
	Призначення:
	IDEF0 модель допомагає детально описати взаємодію між функціями модуля та їх залежності від даних і механізмів.
	
	\section*{IDEF3 модель для опису модуля}
	IDEF3 використовується для моделювання процесів у часі та опису потоків інформації або операцій.
	
	\textbf{Елементи моделі:}
	\begin{enumerate}
		\item Процеси (Processes): Основні етапи виконання задач, такі як "Редагування діаграми" або "Запуск автоматичного тестування".
		\item Сценарії (Scenarios): Моделюють різні варіанти роботи модуля, наприклад, сценарії для різних рівнів користувачів.
		\item Послідовності (Sequences): Відображають порядок виконання процесів.
	\end{enumerate}
	
	Призначення:
	IDEF3 дозволяє детально описати різні варіанти роботи користувачів з модулем та представити логіку виконання процесів у часовому порядку.
	
	\section*{DFD модель для опису модуля}
	
	DFD (Data Flow Diagram) використовується для візуалізації потоків даних у системі.
	
	\textbf{Елементи моделі:}
	\begin{enumerate}
		\item Процеси (Processes): Основні функції, що обробляють дані. Наприклад, "Тестування проекту", "Збереження результатів".
		\item Потоки даних (Data Flows): Показують, як дані рухаються між процесами. Наприклад, потік між "Редактором діаграм" та "Базою даних".
		\item Зовнішні сутності (External Entities): Джерела або одержувачі даних поза системою. Наприклад, користувачі.
		\item Сховища даних (Data Stores): Зони зберігання даних, такі як "База даних проектів" або "База результатів тестування".
	\end{enumerate}
	Призначення:
	DFD дає можливість візуалізувати, як дані перетікають між різними частинами системи та підсистемами, що дозволяє оптимізувати обробку даних.
	
	
	\section*{Висновки}
	ПІд час виконання лабораторної роботи, я використав сучасних методи і засоби проектування інформаційних систем, заснованих на використанні CASE-технологій.
	    
\end{normalsize}
\end{document}
