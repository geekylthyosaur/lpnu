\documentclass[oneside,14pt]{extarticle}
\usepackage{cmap}
\usepackage[utf8]{inputenc}
\usepackage{longtable}
\usepackage[english,ukrainian]{babel}
\usepackage{graphicx}
\usepackage{geometry}
\usepackage{listings}
\usepackage{float}
\usepackage{amsmath}
\usepackage{subfig}
\usepackage{tempora}
\renewcommand{\arraystretch}{1.5}
\geometry{
	a4paper,
	left=20mm,
	right=20mm,
	top=15mm,
	bottom=15mm,
}
\lstset{
	language=c,
	tabsize=4,
	keepspaces,
	showstringspaces=false,
	frame=single,
	breaklines,
	language=C,
}
\graphicspath{ {./pictures} }
\setlength{\parindent}{4em}

\newcommand\subject{Архітектура і проєктування програмного забезпечення}
\newcommand\lecturer{доцент кафедри ПЗ\\Фоменко А.В.}
\newcommand\teacher{старший викладач кафедри ПЗ\\Шкраб Р.Р.}
\newcommand\mygroup{ПЗ-42}
\newcommand\lab{1}
\newcommand\theme{Case системи}
\newcommand\purpose{Введення особливо сучасних методів і засобів проектування інформаційних систем, заснованих на використанні CASE-технології}

\begin{document}
\begin{normalsize}
	\begin{titlepage}
		\thispagestyle{empty}
		\begin{center}
			\textbf{МІНІСТЕРСТВО ОСВІТИ І НАУКИ УКРАЇНИ\\
				НАЦІОНАЛЬНИЙ УНІВЕРСИТЕТ "ЛЬВІВСЬКА ПОЛІТЕХНІКА"}
		\end{center}
		\begin{flushright}
			\textbf{ІКНІ}\\
			Кафедра \textbf{ПЗ}
		\end{flushright}
		\vspace{80pt}
		\begin{center}
			\textbf{ЗВІТ}\\
			\vspace{10pt}
			до лабораторної роботи № \lab\\
			\textbf{на тему}: <<\textit{\theme}>>\\
			\textbf{з дисципліни}: <<\subject>>
		\end{center}
		\vspace{80pt}
		\begin{flushright}
			
			\textbf{Лектор}:\\
			\lecturer\\
			\vspace{28pt}
			\textbf{Виконав}:\\
			
			студент групи \mygroup\\
			Коваленко Д.М.\\
			\vspace{28pt}
			\textbf{Прийняла}:\\
			
			\teacher\\
			
			\vspace{28pt}
			«\rule{1cm}{0.15mm}» \rule{1.5cm}{0.15mm} 2024 р.\\
			$\sum$ = \rule{1cm}{0.15mm}……………\\
			
		\end{flushright}
		\vspace{\fill}
		\begin{center}
			\textbf{Львів — 2024}
		\end{center}
	\end{titlepage}
		
	\begin{description}
		\item[Тема.] \theme.
		\item[Мета.] \purpose.
	\end{description}

    \section*{Лабораторне завдання}
    \begin{enumerate}
    	\item Опішить модуль системи по його BPMN2 моделі, виділіть та опішить елементи моделі та їх призначення.
    	\item Опішить модуль системи по його IDEF0 моделі, виділіть та опішить елементи моделі та їх призначенн.
    	\item Опішить модуль системи по його IDEF3 моделі, виділіть та опішить елементи моделі та їх призначення.
    	\item Опішить модуль системи по його DFD моделі, виділіть та опішить елементи моделі та їх призначення.
    \end{enumerate}
    
    \section*{Хід роботи}
    
    \subsection*{BPMN2 Опис підмодулів}
    \textbf{Підмодулі:}
    \begin{itemize}
    	\item \textbf{Підмодуль створення UML-діаграм}
    	\begin{itemize}
    		\item Завдання: створення UML-діаграм користувачами.
    		\item Елементи: користувач ініціює процес створення діаграми, вибирає тип діаграми, додає елементи, зберігає діаграму.
    	\end{itemize}
    	\item \textbf{Підмодуль збереження та управління проектами}
    	\begin{itemize}
    		\item Завдання: зберігання проектів та управління ними.
    		\item Елементи: користувач зберігає проекти, здійснює їх пошук, редагування та видалення.
    	\end{itemize}
    	\item \textbf{Підмодуль тестування}
    	\begin{itemize}
    		\item Завдання: тестування діаграм за заданими сценаріями.
    		\item Елементи: процес вибору тесту, запуск тестування, виведення результатів.
    	\end{itemize}
    	\item \textbf{Підмодуль аналізу результатів}
    	\begin{itemize}
    		\item Завдання: надання аналітики на основі результатів тестування.
    		\item Елементи: аналіз даних, генерація звітів.
    	\end{itemize}
    \end{itemize}
    
    \subsection*{IDEF0 Опис підмодулів}
    \textbf{Підмодулі:}
    \begin{itemize}
    	\item \textbf{Створення UML-діаграм (A1)}
    	\begin{itemize}
    		\item Функція: користувач створює UML-діаграми.
    		\item Вхідні дані: тип діаграми, початкові параметри.
    		\item Контроль: інструкції, правила побудови.
    		\item Механізми: редактор діаграм, бібліотеки UML.
    	\end{itemize}
    	\item \textbf{Управління проектами (A2)}
    	\begin{itemize}
    		\item Функція: зберігання і керування проектами.
    		\item Вхідні дані: збережені проекти, редаговані діаграми.
    		\item Контроль: доступ користувачів.
    		\item Механізми: база даних проектів.
    	\end{itemize}
    	\item \textbf{Тестування діаграм (A3)}
    	\begin{itemize}
    		\item Функція: виконання автоматичних тестів.
    		\item Вхідні дані: UML-діаграма.
    		\item Контроль: тестові сценарії.
    		\item Механізми: інструмент автоматичного тестування.
    	\end{itemize}
    	\item \textbf{Аналіз результатів (A4)}
    	\begin{itemize}
    		\item Функція: аналіз результатів тестування.
    		\item Вхідні дані: результати тестування.
    		\item Контроль: аналітичні правила.
    		\item Механізми: аналітичні інструменти.
    	\end{itemize}
    \end{itemize}
    
    \subsection*{IDEF3 Опис підмодулів}
    \textbf{Підмодулі:}
    \begin{itemize}
    	\item \textbf{Процес створення UML-діаграм}
    	\begin{itemize}
    		\item Послідовність: користувач ініціює процес створення, додає елементи, завершує процес.
    	\end{itemize}
    	\item \textbf{Процес управління проектами}
    	\begin{itemize}
    		\item Послідовність: користувач обирає проект, редагує його або видаляє, зберігає зміни.
    	\end{itemize}
    	\item \textbf{Процес тестування}
    	\begin{itemize}
    		\item Послідовність: вибір тестів, запуск тестування, отримання результатів.
    	\end{itemize}
    	\item \textbf{Процес аналізу результатів}
    	\begin{itemize}
    		\item Послідовність: отримання даних тестування, обробка, створення звітів.
    	\end{itemize}
    \end{itemize}
    
    \subsection*{DFD Опис підмодулів}
    \textbf{Підмодулі:}
    \begin{itemize}
    	\item \textbf{Підмодуль створення UML-діаграм}
    	\begin{itemize}
    		\item Процеси: користувач вводить дані для діаграм, система зберігає діаграми в базу даних.
    	\end{itemize}
    	\item \textbf{Підмодуль управління проектами}
    	\begin{itemize}
    		\item Процеси: збереження/редагування проектів, пошук в базі даних, виведення на екран.
    	\end{itemize}
    	\item \textbf{Підмодуль тестування}
    	\begin{itemize}
    		\item Процеси: передача UML-діаграми на тестування, отримання результатів тестування.
    	\end{itemize}
    	\item \textbf{Підмодуль аналізу результатів}
    	\begin{itemize}
    		\item Процеси: обробка результатів тестування, збереження даних в базі, виведення результатів.
    	\end{itemize}
    \end{itemize}
    
	
	\section*{Висновки}
	ПІд час виконання лабораторної роботи, я використав сучасних методи і засоби проектування інформаційних систем, заснованих на використанні CASE-технологій.
	    
\end{normalsize}
\end{document}
