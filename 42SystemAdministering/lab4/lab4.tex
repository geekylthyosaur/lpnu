\documentclass[oneside,14pt]{extarticle}
\usepackage{cmap}
\usepackage[utf8]{inputenc}
\usepackage[english,ukrainian]{babel}
\usepackage{graphicx}
\usepackage{geometry}
\usepackage{listings}
\usepackage{float}
\usepackage{amsmath}
\usepackage{subfig}
\usepackage{tempora}
\geometry{
	a4paper,
	left=20mm,
	right=20mm,
	top=15mm,
	bottom=15mm,
}
\lstset{
	language=c,
	tabsize=4,
	keepspaces,
	showstringspaces=false,
	frame=single,
	breaklines,
	language=C,
}
\graphicspath{ {./pictures} }
\setlength{\parindent}{4em}

\newcommand\subject{Системне адміністрування}
\newcommand\lecturer{професор кафедри ПЗ\\Фечан А.В.}
\newcommand\teacher{професор кафедри ПЗ\\Фечан А.В.}
\newcommand\mygroup{ПЗ-42}
\newcommand\lab{4}
\newcommand\theme{Реалізація механізму групових політик у Windows 10. Аналіз
	і налаштування безпеки}
\newcommand\purpose{Ознайомлення зі структурою, принципом роботи та
	налаштуванням об’єкта групової політики на локальному комп’ютері під
	управлінням ОС Windows 10. Навчитись використовувати та створювати
	шаблони безпеки для ефективного налаштування та аналізу типових параметрів
	безпеки}

\begin{document}
\begin{normalsize}
	\begin{titlepage}
		\thispagestyle{empty}
		\begin{center}
			\textbf{МІНІСТЕРСТВО ОСВІТИ І НАУКИ УКРАЇНИ\\
				НАЦІОНАЛЬНИЙ УНІВЕРСИТЕТ "ЛЬВІВСЬКА ПОЛІТЕХНІКА"}
		\end{center}
		\begin{flushright}
			\textbf{ІКНІ}\\
			Кафедра \textbf{ПЗ}
		\end{flushright}
		\vspace{80pt}
		\begin{center}
			\textbf{ЗВІТ}\\
			\vspace{10pt}
			до лабораторної роботи № \lab\\
			\textbf{на тему}: <<\textit{\theme}>>\\
			\textbf{з дисципліни}: <<\subject>>
		\end{center}
		\vspace{80pt}
		\begin{flushright}
			
			\textbf{Лектор}:\\
			\lecturer\\
			\vspace{28pt}
			\textbf{Виконав}:\\
			
			студент групи \mygroup\\
			Коваленко Д.М.\\
			\vspace{28pt}
			\textbf{Прийняв}:\\
			
			\teacher\\
			
			\vspace{28pt}
			«\rule{1cm}{0.15mm}» \rule{1.5cm}{0.15mm} 2024 р.\\
			$\sum$ = \rule{1cm}{0.15mm}……………\\
			
		\end{flushright}
		\vspace{\fill}
		\begin{center}
			\textbf{Львів — 2024}
		\end{center}
	\end{titlepage}
		
	\begin{description}
		\item[Тема.] \theme.
		\item[Мета.] \purpose.
	\end{description}

    \section*{Лабораторне завдання}
	\begin{enumerate}
		\item 
	\end{enumerate}

	\section*{Хід роботи}
	
	\section*{Висновки}
	Під час виконання лабораторної роботи я ознайомився зі структурою, принципом роботи та
	налаштуванням об’єкта групової політики на локальному комп’ютері під
	управлінням ОС Windows 10. Навчився використовувати та створювати
	шаблони безпеки для ефективного налаштування та аналізу типових параметрів
	безпеки.
		    
\end{normalsize}
\end{document}
