\documentclass[14pt]{extreport}
\usepackage{cmap}
\usepackage[utf8]{inputenc}
\usepackage[english,ukrainian]{babel}
\usepackage{graphicx}
\usepackage{geometry}
\usepackage{listings}
\usepackage{amsmath}
\usepackage{float}
\usepackage{array}
\geometry{
	a4paper,
	left=20mm,
	right=20mm,
	top=20mm,
	bottom=20mm
}
\lstset{
	language=bash,
	tabsize=4,
	breaklines,
	keepspaces,
	showstringspaces=false,
}
\graphicspath{ {./pictures} }
\setlength{\parindent}{4em}

\newcommand\subject{Аналіз вимог до програмного забезпечення}
\newcommand\lecturer{професор кафедри ПЗ\\Грицюк Ю.І.}
\newcommand\teacher{асистент кафедри ПЗ\\Масюкевич В.В.}
\newcommand\mygroup{ПЗ-32}
\newcommand\lab{1}
\newcommand\theme{Аналіз наявних програм-аналогів для встановлення вимог до нового програмного забезпечення для обраної предметної області}
\newcommand\purpose{Проаналізувати наявні програмні продукти заданої предметної
	області}

\begin{document}
\begin{normalsize}
	\begin{titlepage}
		\thispagestyle{empty}
		\begin{center}
			\textbf{МІНІСТЕРСТВО ОСВІТИ І НАУКИ УКРАЇНИ\\
				НАЦІОНАЛЬНИЙ УНІВЕРСИТЕТ "ЛЬВІВСЬКА ПОЛІТЕХНІКА"}
		\end{center}
		\begin{flushright}
			Інститут \textbf{КНІТ}\\
			Кафедра \textbf{ПЗ}
		\end{flushright}
		\vspace{200pt}
		\begin{center}
			\textbf{ЗВІТ}\\
			\vspace{10pt}
			До лабораторної роботи № \lab\\
			\textbf{На тему}: “\textit{\theme}”\\
			\textbf{З дисципліни}: “\subject”
		\end{center}
		\vspace{40pt}
		\begin{flushright}
			
			\textbf{Лектор}:\\
			\lecturer\\
			\vspace{10pt}
			\textbf{Виконав}:\\
			
			студент групи \mygroup\\
			Коваленко Д.М.\\
			\vspace{10pt}
			\textbf{Прийняв}:\\
			
			\teacher\\
			
			\vspace{28pt}
			«\rule{1cm}{0.15mm}» \rule{1.5cm}{0.15mm} 2023 р.\\
			$\sum$ = \rule{1cm}{0.15mm}……………\\
			
		\end{flushright}
		\vspace{\fill}
		\begin{center}
			\textbf{Львів — 2023}
		\end{center}
	\end{titlepage}
		
	\begin{description}
		\item[Тема.] \theme.
		\item[Мета.] \purpose.
	\end{description}

	\section*{Лабораторне завдання}
	\begin{enumerate}
		\item Вибрати предметну область із запропонованого переліку.
		\item Здійснити пошук в мережі Інтернет кількох (від трьох до п'яти) наявних
		програм-аналогів для обраної предметної області. Якщо у вільному доступі немає
		програм-аналогів, проаналізувати вітчизняні та закордонні веб-ресурси.
		\item Описати кожну із розглянутих програмних систем (сайтів).
		\item Здійснити порівняння наявних програм-аналогів і внести дані в таблицю.
		\item На основі проведеного аналізу скласти перелік вимог до ПС для заданої
		предметної області.
		\item Оформити звіт.
	\end{enumerate}
	
	\section*{Хід роботи}
	
	\begin{table}[H]
		\centering
		\renewcommand*\arraystretch{1.3}
		\begin{tabular}{|m{0.26\linewidth}|m{0.20\linewidth}|m{0.2\linewidth}|m{0.2\linewidth}|}
			\hline
			 & \textbf{EcoLane} & \textbf{TransLoc} & \textbf{OptiBus} \\\hline
			\textbf{Веб-інтерфейс} & & + & + \\\hline
			\textbf{Додаток для водія} & & + & \\\hline
			\textbf{Побудова оптимального маршруту} & + & + & + \\\hline
			\textbf{Відслідковування транспорту на маршрутах} & & + & \\\hline
			\textbf{Комунікація з водієм} & & + & \\\hline
			\textbf{Формування даних для аналітики} & + & & + \\\hline
			\textbf{Інтеграція оплати проїзду} & + & & \\\hline
			\textbf{Генерація оптимального розкладу руху} & + & & + \\\hline
			\textbf{Оцінювання поїздки пасажирами} & + & & \\\hline
			\textbf{Бронювання місць} & + & & + \\\hline
			\textbf{Пробний режим роботи} & + & & \\\hline
		\end{tabular}
	\end{table}

	\section*{Висновок}
	 
\end{normalsize}
\end{document}
