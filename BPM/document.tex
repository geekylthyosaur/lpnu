\documentclass[oneside,14pt]{extarticle}
\usepackage[utf8]{inputenc}
\usepackage[english,ukrainian]{babel}
\usepackage{amssymb,amsfonts,amsmath,amsthm,mathtext,textcomp}

\usepackage[includehead, headsep=0pt, footskip=0pt, top=2cm, bottom=2cm, left=2cm, right=2cm]{geometry}
\usepackage{indentfirst}
\usepackage[onehalfspacing]{setspace}
\usepackage[headings]{fancyhdr}
\usepackage{etoolbox}
\usepackage{flafter}
\usepackage{hyperref}
\usepackage{multirow}
\usepackage{listings}
\usepackage{graphicx}
\usepackage{float}
\usepackage[center]{titlesec}
\titlelabel{\thetitle.\quad}
\usepackage{array}
\fancyhf{}
\renewcommand{\headrulewidth}{0pt}
\renewcommand{\baselinestretch}{1.5}
\pagestyle{fancy}
\fancyfoot[R]{\thepage}
\lstset{breaklines=true,}
\graphicspath{ {./pictures} }
\counterwithin{figure}{section}

\begin{document}
\begin{titlepage}
	\begin{center}
МІНІСТЕРСТВО ОСВІТИ І НАУКИ УКРАЇНИ\\
НАЦІОНАЛЬНИЙ УНІВЕРСИТЕТ <<ЛЬВІВСЬКА ПОЛІТЕХНІКА>>\\
ННІ АДМІНІСТРУВАННЯ ТА ПІСЛЯДИПЛОМНОЇ ОСВІТИ\\
КАФЕДРА ТЕХНОЛОГІЙ УПРАВЛІННЯ

		
		\vspace{80pt}
Контрольна робота\\
з дисципліни <<Бізнес-планування та управління проектами>>\\
Варіант №16
\begin{flushleft}
	1 теор. пит. №16\\
	2 теор. пит. №6\\
	1 прак. зав. №3\\
	2 прак. зав. №9\\
\end{flushleft}
		\vspace*{20pt}
		
		\begin{flushright}
			\textbf{Виконав}:\\
			
			студент групи БПУ-ТУ-213\\
			Коваленко Д.М.\\
			\vspace{10pt}
			\textbf{Прийняла}:\\
			ст викл. Лебідь Т.В.
			
			\vspace{28pt}
			«\rule{1cm}{0.15mm}» \rule{1.5cm}{0.15mm} 2023 р.\\
			$\sum$ = \rule{1cm}{0.15mm}……………\\
			
		\end{flushright}
		\vspace{\fill}
		Львів — 2023
	\end{center}
\end{titlepage}
\setcounter{page}{2}
\tableofcontents
\newpage

\section*{ВСТУП}
\addcontentsline{toc}{section}{ВСТУП}
Метою виконання контрольної роботи є вивчення сучасної методико-прикла-дної бази бізнес-планування та управління проектами і програмами.

Навчальними завданнями є – вивчення наукових категорій, набуття практичних навиків визначення пріоритетного напряму діяльності бізнесу, розрахунку показників бізнес-плану та оформлення проектів його розділів, формування бізнес-плану, оцінювання економічної ефективності проектів, налагодження, розвитку та узгодження взаємодії зі стейкхолдерами у проектній діяльності, застосування методів фінансового планування та проектного бюджетування, визначення джерел фінансування, схем фінансування проектів. 

Окремими навчальними завданнями є набуття знань та розуміння параметрів і чинників впливу на розвиток проекту з метою формування його бюджету, ризиків проектної діяльності, методів керівництва проектами і програмами, зокрема фінансових ризиків проекту, їхнє врахування під час застосування бюджетного методу управління проектом, 

До навчальних завдань належить також набуття умінь самостійно працювати з навчальною, монографічною та періодичною літературою, оброблення та узагальнення, критичне аналізування даних наукових джерел, використання законодавчої та нормативної бази, набуття вмінь застосовувати методи дослідження та формулювати висновки відповідно до отриманих в результаті дослідження результатів.

\newpage
\section*{Теоретичне питання 1}
\addcontentsline{toc}{section}{Теоретичне питання 1}

Основні стейкхолдери проекту та їхня роль у проектах.

Основні зацікавлені сторони проекту можуть відрізнятися залежно від характеру та обсягу проекту. Однак я можу надати вам загальний огляд деяких поширених зацікавлених сторін та їхні потенційні ролі в проекті:

\begin{enumerate}
	\item Спонсор проекту: Спонсор проекту зазвичай є керівником вищої ланки або групою керівників, які забезпечують загальне бачення, стратегічний напрямок і фінансові ресурси для проекту. Вони підтримують проект і забезпечують його відповідність цілям і завданням організації.
	\item Менеджер проекту: Менеджер проекту відповідає за планування, виконання та закриття проекту. Вони координують і контролюють усі дії проекту, керують ресурсами, спілкуються із зацікавленими сторонами та забезпечують виконання проекту вчасно, у межах бюджету та відповідно до визначеного обсягу.
	\item Команда проекту: Команда проекту складається з осіб, які відповідають за виконання завдань і заходів проекту. Члени команди можуть включати експертів, технічних спеціалістів, аналітиків, дизайнерів, розробників та інших професіоналів, які вносять свої навички та досвід у проект.
	\item Клієнти/замовник: клієнт або замовник — це особа чи організація, для яких виконується проект. Вони надають вимоги до проекту, очікування та можуть брати активну участь у прийнятті рішень протягом життєвого циклу проекту. Їхнє задоволення результатами проекту є критичним показником успіху.
	\item Кінцеві користувачі: Кінцеві користувачі – це особи чи групи, які використовуватимуть кінцевий продукт, послугу чи рішення проекту або отримають вигоду від них. Розуміння їхніх потреб, уподобань і відгуків має важливе значення для того, щоб переконатися, що проект відповідає їхнім вимогам і забезпечує цінність.
	\item Внутрішні стейкхолдери: Внутрішні стейкхолдери можуть включати керівників, співробітників, відділи або бізнес-підрозділи всередині організації, на яких проект прямо чи опосередковано впливає. Вони можуть надавати ресурси, досвід або на них впливають результати проекту.
	\item Зовнішні стейкхолдери: Зовнішні стейкхолдери – це особи чи організації, які не належать до команди проекту та організації, але мають особистий інтерес до проекту. Вони можуть включати постачальників, продавців, партнерів, регулюючі органи, державні установи або широку громадськість. Їхня підтримка, відповідність вимогам або відгуки можуть бути необхідними для успіху проекту.
	\item Офіс управління проектами (PMO): У деяких організаціях Офіс управління проектами контролює портфель проектів і забезпечує управління, методології, інструменти та підтримку для забезпечення ефективного виконання проектів. PMO може мати представників, які діють як зацікавлені сторони в проекті.
\end{enumerate}

\newpage
\section*{Теоретичне питання 2}
\addcontentsline{toc}{section}{Теоретичне питання 2}

Стратегії розвитку бізнесу. Етапи розроблення бізнес-стратегії.

Стратегії розвитку бізнесу включають визначення та реалізацію планів і тактик для стимулювання зростання та успіху бізнесу. Ці стратегії зазвичай передбачають аналіз ринкових тенденцій, оцінку конкурентних ландшафтів, виявлення нових можливостей і розробку планів їх використання. Етапи розробки бізнес-стратегії можуть відрізнятися залежно від організації та її конкретних потреб, але ось загальні етапи:

\begin{enumerate}
	\item Аналіз і оцінка: на цьому етапі підприємства оцінюють свою поточну ситуацію, проводячи ретельний аналіз своїх внутрішніх сильних і слабких сторін, а також зовнішніх можливостей і загроз. Це включає в себе аналіз ринкових тенденцій, потреб клієнтів, стратегії конкурентів, а також власних ресурсів і можливостей компанії.
	\item Постановка цілей: після завершення аналізу підприємства встановлюють чіткі та конкретні цілі для своєї стратегії розвитку. Ці цілі мають узгоджуватися з баченням і місією компанії, бути вимірними та обмеженими за часом. Приклади цілей можуть включати збільшення частки ринку, розширення на нові ринки або запуск нових продуктів чи послуг.
	\item Формулювання стратегії: на цьому етапі підприємства розробляють стратегії для досягнення своїх цілей. Це передбачає визначення цільових ринків, визначення ціннісних пропозицій і визначення конкурентного позиціонування. Підприємства також можуть розглянути різні стратегії зростання, такі як проникнення на ринок, розвиток ринку, розробка продукту або диверсифікація.
	\item Планування реалізації: після формулювання стратегій підприємствам необхідно створити детальні плани їх реалізації. Це включає визначення кроків дій, розподіл ресурсів, встановлення часових рамок і розподіл обов’язків. Важливо мати чітку дорожню карту, яка визначає, як стратегії будуть виконуватися ефективно.
	\item Виконання та моніторинг: цей етап передбачає впровадження планів у дію та моніторинг прогресу. Підприємствам слід відстежувати ключові показники ефективності (KPI), щоб оцінити ефективність своїх стратегій і за потреби вносити корективи. Регулярний моніторинг дозволяє підприємствам на ранній стадії виявляти будь-які проблеми чи проблеми та вживати заходів щодо їх усунення.
	\item Оцінка та адаптація: періодичний перегляд та оцінка впроваджених стратегій має вирішальне значення для забезпечення бажаних результатів. Цей етап передбачає аналіз результатів, визначення областей для покращення та внесення необхідних коригувань до стратегій. Важливо бути гнучким і адаптуватися до змін у бізнес-середовищі.
	\item Постійне вдосконалення. Стратегії розвитку бізнесу слід розглядати як постійний процес. Постійне вдосконалення передбачає вивчення минулого досвіду, збір відгуків від клієнтів і зацікавлених сторін і внесення повторюваних змін для підвищення продуктивності та досягнення сталого зростання.
\end{enumerate}

\newpage
\section*{Практичне завдання 1}
\addcontentsline{toc}{section}{Практичне завдання 1}

Бюджет проекту 6 млн. грн. (без урахування фінансових витрат). Підприємство планує фінансувати проект за рахунок кредиту. Банк встановив процентну ставку 30\% річних. Кредит повинен погашатись рівними частинами впродовж 1,5 року. Визначити розміри періодичних сплат, суму сплачених процентів та загальну суму погашення, якщо такі платежі здійснюються один раз у півроку, а кредит надається на початку проекту (01.01.2017 р.). Скласти звіт про рух грошових коштів на фінансування витрат за проектом.

Розв'язання:

Для розрахунку розміру періодичних сплат та суми сплачених процентів використовуємо формулу ануїтетного платежу:

$A = (P * i) / (1 - (1 + i)^{-n})$

Розрахуємо розмір періодичного платежу:
\begin{gather}
	P = 6,000,000 \text{ грн.}\nonumber\\
	i = 30\% / 100 / 2 = 0.15\nonumber\\
	n = 1.5 * 2 = 3\nonumber\\
	A = (6,000,000 * 0.15) / (1 - (1 + 0.15)^{-3})\nonumber\\
	A = 2,254,115.56 \text{ грн.}\nonumber
\end{gather}

Тепер розрахуємо суму сплачених процентів:

\begin{gather}
	\text{Сума процентів } = (A * n) - P\nonumber\\
	\text{Сума процентів } = (2,254,115.56 * 3) - 6,000,000\nonumber\\
	\text{Сума процентів }  6,762,346.68 \text{ грн.}\nonumber
\end{gather}

Загальна сума погашення буде складатись з суми кредиту та суми сплачених процентів:

\begin{gather}
	\text{Загальна сума погашення }= P + \text{ Сума процентів}\nonumber\\
	\text{Загальна сума погашення }= 6,000,000 + 6,762,346.68\nonumber\\
	\text{Загальна сума погашення }= 12,762,346.68\text{ грн.}\nonumber
\end{gather}

Тепер складемо звіт про рух грошових коштів на фінансування витрат за проектом. З представлених даних зробимо наступні припущення:

Періодичні платежі здійснюються один раз у півріччя (раз у 6 місяців).
Кредит надається на початку проекту (01.01.2017 р.).

Звіт про рух грошових коштів на фінансування витрат за проектом:

\begin{table}[H]
	\centering
	\begin{tabular}{|c|c|c|c|}
		\hline
		Дата & Отримано & Виплачено & Залишок \\
		\hline
		01.01.2017 & 6,000,000 грн. & - & 6,000,000 грн. \\
		\hline
		01.07.2017 & - & 2,254,115.56 грн. & 3,745,884.44 грн. \\
		\hline
		01.01.2018 & - & 2,254,115.56 грн. & 1,491,768.88 грн. \\
		\hline
		01.07.2018 & - & 2,254,115.56 грн. & 237,653.32 грн. \\
		\hline
		01.01.2019 & - & 2,254,115.56 грн. & -2,016,462.24 грн. \\
		\hline
	\end{tabular}
\end{table}

Отже, рух грошових коштів на фінансування витрат за проектом показує, що на початку проекту підприємство отримало 6,000,000 грн. кредиту. Протягом наступних трьох півріччів було виплачено по 2,254,115.56 грн., а на початку 2019 року борг становив -2,016,462.24 грн., що означає, що борг був повністю погашений.

\newpage
\section*{Практичне завдання 2}
\addcontentsline{toc}{section}{Практичне завдання 2}

Результати оцінювання трьома експертами (Е1, Е2, Е3) проектних альтернатив (А1, А2, А3, А4) за критерієм пріоритетності реалізації за десятибальною шкалою наведено у табл. 4.1. Необхідно встановити пріоритетність проектних альтернатив за їхніми коефіцієнтами вагомості.

	\begin{table}[H]
		\centering
		\caption{Експертні оцінки альтернатив}
		\begin{tabular}{|p{0.13\linewidth}|p{0.13\linewidth}|p{0.13\linewidth}|p{0.13\linewidth}|p{0.13\linewidth}|p{0.13\linewidth}|}
			\hline
			\multirow{2}{*}{Експерти} & \multicolumn{4}{c|}{Експертні оцінки альтернатив} & \multirow{2}{*}{$\sum$} \\
			\cline{2-5}
			& А1 & А2 & А3 & А4 & \\
			\hline
			Е1 & 2 & 5 & 1 & 2 & 10 \\
			\hline
			Е2 & 2 & 2 & 3 & 3 & 10 \\
			\hline
			Е3 & 1 & 3 & 3 & 3 & 10 \\
			\hline
		\end{tabular}
	\end{table}
	
	Розв'язання:
	Запишемо суму оцінок альтернатив і загальну суму балів, використаних при оцінці даних альтернатив.

	\begin{table}[H]
		\centering
		\caption{Експертні оцінки альтернатив}
		\begin{tabular}{|p{0.13\linewidth}|p{0.13\linewidth}|p{0.13\linewidth}|p{0.13\linewidth}|p{0.13\linewidth}|p{0.13\linewidth}|}
			\hline
			\multirow{2}{*}{Експерти} & \multicolumn{4}{c|}{Експертні оцінки альтернатив} & \multirow{2}{*}{$\sum$} \\
			\cline{2-5}
			& А1 & А2 & А3 & А4 & \\
			\hline
			Е1 & 2 & 5 & 1 & 2 & 10 \\
			\hline
			Е2 & 2 & 2 & 3 & 3 & 10 \\
			\hline
			Е3 & 1 & 3 & 3 & 3 & 10 \\
			\hline
			Сума & 5 & 10 & 7 & 8 & 30 \\
			\hline
		\end{tabular}
	\end{table}
	
	Значення коефіцієнтів вагомості визначають за такою формулою:
	
	\begin{gather}
		K_{B_i}=\frac{A_{\text{ср}_i}}{\sum A_{\text{ср}_i}}\nonumber
	\end{gather}
	
	Де $K_B$ - коефіцієнт вагомості.
	
	\begin{gather}
		K_{B_1}=\frac{5}{30}=0.166\nonumber\\
		K_{B_1}=\frac{10}{30}=0.333\nonumber\\
		K_{B_1}=\frac{7}{30}=0.233\nonumber\\
		K_{B_1}=\frac{8}{30}=0.266\nonumber
	\end{gather}
	
	Отже, приорітети альтернатив необхідно виставити в такому порядку: 
	
	\begin{gather}
		A_2\rightarrow A_4\rightarrow A_3\rightarrow A_1	\nonumber
	\end{gather}

\newpage
\section*{СПИСОК ВИКОРИСТАНИХ ДЖЕРЕЛ}
\addcontentsline{toc}{section}{Список використаної джерел}

\begin{enumerate}
	\item ISO 3100:2018 Risk management. Guidelines. –Second edition. – 2018-02. – 24 pp.
	\item Чумаченко І. В. Управління проектами: процеси планування проектних дій / І. В. Чумаченко, В. В. Морозов, І. В. Доценко та ін. – К.: Університет економіки та права «Крок», 2014. – 673 с.
	\item Фещур Р. В. Прийняття проектних рішень: навч. посіб. / Р. В. Фещур, В. П. Кічор, А. І. Якимів [та ін.]; за ред. Р. В. Фещура. – Львів : Вид-во Львівської політехніки, 2013. – 216 с.
	\item Крикавський Є. Промисловий маркетинг: підручник. 2-ге вид. / Є.Крикавський, Н. Чухрай. – Львів: Видавництво Національного університету «Львівська політехніка», 2004. – 472 с.
	\item Маркетинг: підручник /А.О. Старостіна, Н.П. Гончарова, Є.В. Крикавський [та ін.]; за ред. А.О. Старостіної. – К.: Знання, 2009. – 1070с.
\end{enumerate}

\end{document}
