\documentclass[oneside,14pt]{extarticle}
\usepackage[utf8]{inputenc}
\usepackage[english,ukrainian]{babel}
\usepackage{amssymb,amsfonts,amsmath,amsthm,mathtext,textcomp}

\usepackage[includehead, headsep=0pt, footskip=0pt, top=2cm, bottom=2cm, left=2cm, right=15mm]{geometry}
\usepackage{indentfirst}
\usepackage[onehalfspacing]{setspace}
\usepackage[headings]{fancyhdr}
\usepackage{etoolbox}
\usepackage{flafter}
\usepackage{hyperref}
\usepackage{multirow}
\usepackage{listings}
\usepackage{graphicx}
\usepackage{float}
\usepackage[center]{titlesec}
\titlelabel{\thetitle.\quad}
\usepackage{array}
\fancyhf{}
\renewcommand{\headrulewidth}{0pt}
\renewcommand{\baselinestretch}{1.5}
\pagestyle{fancy}
\fancyfoot[R]{\thepage}
\lstset{breaklines=true,}
\graphicspath{ {./pictures} }
\counterwithin{figure}{section}

\begin{document}
\begin{titlepage}
	\begin{center}
МІНІСТЕРСТВО ОСВІТИ І НАУКИ УКРАЇНИ\\
НАЦІОНАЛЬНИЙ УНІВЕРСИТЕТ <<ЛЬВІВСЬКА ПОЛІТЕХНІКА>>\\
ННІ АДМІНІСТРУВАННЯ ТА ПІСЛЯДИПЛОМНОЇ ОСВІТИ\\
КАФЕДРА ТЕХНОЛОГІЙ УПРАВЛІННЯ

		
		\vspace{80pt}
Контрольна робота\\
з дисципліни <<Бізнес-планування та управління проектами>>\\
Варіант №16
\begin{flushleft}
	1 теор. пит. №16\\
	2 теор. пит. №6\\
	1 прак. зав. №3\\
	2 прак. зав. №9\\
\end{flushleft}
		\vspace*{20pt}
		
		\begin{flushright}
			\textbf{Виконав}:\\
			
			студент групи БПУ-ТУ-213\\
			Коваленко Д.М.\\
			\vspace{10pt}
			\textbf{Прийняла}:\\
			ст викл. Лебідь Т.В.
			
			\vspace{28pt}
			«\rule{1cm}{0.15mm}» \rule{1.5cm}{0.15mm} 2023 р.\\
			$\sum$ = \rule{1cm}{0.15mm}……………\\
			
		\end{flushright}
		\vspace{\fill}
		Львів — 2023
	\end{center}
\end{titlepage}
\setcounter{page}{2}
\tableofcontents
\newpage

\section*{ВСТУП}
\addcontentsline{toc}{section}{ВСТУП}

\newpage
\section*{Теоретичне питання 1}
\addcontentsline{toc}{section}{Теоретичне питання 1}

\newpage
\section*{Теоретичне питання 2}
\addcontentsline{toc}{section}{Теоретичне питання 2}

\newpage
\section*{Практичне завдання 1}
\addcontentsline{toc}{section}{Практичне завдання 1}

Бюджет проекту 6 млн. грн. (без урахування фінансових витрат). Підприємство планує фінансувати проект за рахунок кредиту. Банк встановив процентну ставку 30\% річних. Кредит повинен погашатись рівними частинами впродовж 1,5 року. Визначити розміри періодичних сплат, суму сплачених процентів та загальну суму погашення, якщо такі платежі здійснюються один раз у півроку, а кредит надається на початку проекту (01.01.2017 р.). Скласти звіт про рух грошових коштів на фінансування витрат за проектом.

Розв'язання:

Для розрахунку розміру періодичних сплат та суми сплачених процентів використовуємо формулу ануїтетного платежу:

$A = (P * i) / (1 - (1 + i)^{-n})$

Розрахуємо розмір періодичного платежу:
\begin{gather}
	P = 6,000,000 \text{ грн.}\nonumber\\
	i = 30\% / 100 / 2 = 0.15\nonumber\\
	n = 1.5 * 2 = 3\nonumber\\
	A = (6,000,000 * 0.15) / (1 - (1 + 0.15)^{-3})\nonumber\\
	A = 2,254,115.56 \text{ грн.}\nonumber
\end{gather}

Тепер розрахуємо суму сплачених процентів:

\begin{gather}
	\text{Сума процентів } = (A * n) - P\nonumber\\
	\text{Сума процентів } = (2,254,115.56 * 3) - 6,000,000\nonumber\\
	\text{Сума процентів }  6,762,346.68 \text{ грн.}\nonumber
\end{gather}

Загальна сума погашення буде складатись з суми кредиту та суми сплачених процентів:

\begin{gather}
	\text{Загальна сума погашення }= P + \text{ Сума процентів}\nonumber\\
	\text{Загальна сума погашення }= 6,000,000 + 6,762,346.68\nonumber\\
	\text{Загальна сума погашення }= 12,762,346.68\text{ грн.}\nonumber
\end{gather}

Тепер складемо звіт про рух грошових коштів на фінансування витрат за проектом. З представлених даних зробимо наступні припущення:

Періодичні платежі здійснюються один раз у півріччя (раз у 6 місяців).
Кредит надається на початку проекту (01.01.2017 р.).

Звіт про рух грошових коштів на фінансування витрат за проектом:

\begin{table}[H]
	\centering
	\begin{tabular}{|c|c|c|c|}
		\hline
		Дата & Отримано & Виплачено & Залишок \\
		\hline
		01.01.2017 & 6,000,000 грн. & - & 6,000,000 грн. \\
		\hline
		01.07.2017 & - & 2,254,115.56 грн. & 3,745,884.44 грн. \\
		\hline
		01.01.2018 & - & 2,254,115.56 грн. & 1,491,768.88 грн. \\
		\hline
		01.07.2018 & - & 2,254,115.56 грн. & 237,653.32 грн. \\
		\hline
		01.01.2019 & - & 2,254,115.56 грн. & -2,016,462.24 грн. \\
		\hline
	\end{tabular}
\end{table}

Отже, рух грошових коштів на фінансування витрат за проектом показує, що на початку проекту підприємство отримало 6,000,000 грн. кредиту. Протягом наступних трьох півріччів було виплачено по 2,254,115.56 грн., а на початку 2019 року борг становив -2,016,462.24 грн., що означає, що борг був повністю погашений.

\newpage
\section*{Практичне завдання 2}
\addcontentsline{toc}{section}{Практичне завдання 2}

Результати оцінювання трьома експертами (Е1, Е2, Е3) проектних альтернатив (А1, А2, А3, А4) за критерієм пріоритетності реалізації за десятибальною шкалою наведено у табл. 4.1. Необхідно встановити пріоритетність проектних альтернатив за їхніми коефіцієнтами вагомості.

	\begin{table}[H]
		\centering
		\caption{Експертні оцінки альтернатив}
		\begin{tabular}{|p{0.13\linewidth}|p{0.13\linewidth}|p{0.13\linewidth}|p{0.13\linewidth}|p{0.13\linewidth}|p{0.13\linewidth}|}
			\hline
			\multirow{2}{*}{Експерти} & \multicolumn{4}{c|}{Експертні оцінки альтернатив} & \multirow{2}{*}{$\sum$} \\
			\cline{2-5}
			& А1 & А2 & А3 & А4 & \\
			\hline
			Е1 & 2 & 5 & 1 & 2 & 10 \\
			\hline
			Е2 & 2 & 2 & 3 & 3 & 10 \\
			\hline
			Е3 & 1 & 3 & 3 & 3 & 10 \\
			\hline
		\end{tabular}
	\end{table}
	
	Розв'язання:
	Запишемо суму оцінок альтернатив і загальну суму балів, використаних при оцінці даних альтернатив.

	\begin{table}[H]
		\centering
		\caption{Експертні оцінки альтернатив}
		\begin{tabular}{|p{0.13\linewidth}|p{0.13\linewidth}|p{0.13\linewidth}|p{0.13\linewidth}|p{0.13\linewidth}|p{0.13\linewidth}|}
			\hline
			\multirow{2}{*}{Експерти} & \multicolumn{4}{c|}{Експертні оцінки альтернатив} & \multirow{2}{*}{$\sum$} \\
			\cline{2-5}
			& А1 & А2 & А3 & А4 & \\
			\hline
			Е1 & 2 & 5 & 1 & 2 & 10 \\
			\hline
			Е2 & 2 & 2 & 3 & 3 & 10 \\
			\hline
			Е3 & 1 & 3 & 3 & 3 & 10 \\
			\hline
			Сума & 5 & 10 & 7 & 8 & 30 \\
			\hline
		\end{tabular}
	\end{table}
	
	Значення коефіцієнтів вагомості визначають за такою формулою:
	
	\begin{gather}
		K_{B_i}=\frac{A_{\text{ср}_i}}{\sum A_{\text{ср}_i}}
	\end{gather}
	
	Де $K_B$ - коефіцієнт вагомості.
	
	\begin{gather}
		K_{B_1}=\frac{5}{30}=0.166\nonumber\\
		K_{B_1}=\frac{10}{30}=0.333\nonumber\\
		K_{B_1}=\frac{7}{30}=0.233\nonumber\\
		K_{B_1}=\frac{8}{30}=0.266\nonumber
	\end{gather}
	
	Отже, приорітети альтернатив необхідно виставити в такому порядку: 
	
	\begin{gather}
		A_2\rightarrow A_4\rightarrow A_3\rightarrow A_1	\nonumber
	\end{gather}

\newpage
\section*{СПИСОК ВИКОРИСТАНИХ ДЖЕРЕЛ}
\addcontentsline{toc}{section}{Список використаної джерел}

\end{document}
