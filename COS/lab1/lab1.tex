\documentclass[oneside,14pt]{extarticle}
\usepackage{cmap}
\usepackage[utf8]{inputenc}
\usepackage[english,ukrainian]{babel}
\usepackage{graphicx}
\usepackage{geometry}
\usepackage{listings}
\usepackage{float}
\usepackage{amsmath}
\usepackage{subfig}
\usepackage{enumitem}
\geometry{
	a4paper,
	left=20mm,
	right=20mm,
	top=15mm,
	bottom=15mm,
}
\lstset{
	language=c,
	tabsize=4,
	keepspaces,
	showstringspaces=false,
	frame=single,
	language=python,
}
\graphicspath{ {./pictures} }
\setlength{\parindent}{4em}

\newcommand\subject{Основи програмування вбудованих систем}
\newcommand\lecturer{професор кафедри ПЗ\\Гавриш В.І.}
\newcommand\teacher{доцент кафедри ПЗ\\Крук О.Г.}
\newcommand\mygroup{ПЗ-32}
\newcommand\lab{1}
\newcommand\theme{Моделювання аналогових $2\pi$- періодичних сигналів рядом Фур’є}
\newcommand\purpose{Наблизити рядом Фур’є аналоговий $2\pi$- періодичний
	сигнал, виконати геометричне зображення функції, якою описано цей сигнал, та
	кривої, яку подано рядом Фур’є. Визначити середню абсолютну похибку
	наближення}

\begin{document}
\begin{normalsize}
	\begin{titlepage}
		\thispagestyle{empty}
		\begin{center}
			\textbf{МІНІСТЕРСТВО ОСВІТИ І НАУКИ УКРАЇНИ\\
				НАЦІОНАЛЬНИЙ УНІВЕРСИТЕТ "ЛЬВІВСЬКА ПОЛІТЕХНІКА"}
		\end{center}
		\begin{flushright}
			\textbf{ІКНІ}\\
			Кафедра \textbf{ПЗ}
		\end{flushright}
		\vspace{80pt}
		\begin{center}
			\textbf{ЗВІТ}\\
			\vspace{10pt}
			до лабораторної роботи № \lab\\
			\textbf{на тему}: <<\textit{\theme}>>\\
			\textbf{з дисципліни}: <<\subject>>
		\end{center}
		\vspace{80pt}
		\begin{flushright}
			
			\textbf{Лектор}:\\
			\lecturer\\
			\vspace{28pt}
			\textbf{Виконав}:\\
			
			студент групи \mygroup\\
			Коваленко Д.М.\\
			\vspace{28pt}
			\textbf{Прийняв}:\\
			
			\teacher\\
			
			\vspace{28pt}
			«\rule{1cm}{0.15mm}» \rule{1.5cm}{0.15mm} 2024 р.\\
			$\sum$ = \rule{1cm}{0.15mm}……………\\
			
		\end{flushright}
		\vspace{\fill}
		\begin{center}
			\textbf{Львів — 2024}
		\end{center}
	\end{titlepage}
		
	\begin{description}
		\item[Тема.] \theme.
		\item[Мета.] \purpose.
	\end{description}

	\section*{Індивідуальне завдання}
	Реалізувати довільною мовою програмування:
	\begin{enumerate}[label=\arabic*)]
		\item підпрограму (процедуру чи функцію), яка дає змогу визначати значення
		потужності сигналу за аналітичним виразом функції та за
		співвідношенням, отриманим у результатом розкладу її в ряд Фур’є;
		\item підпрограми (процедури чи функції), які дають змогу визначити значення
		коефіцієнтів ряду Фур’є;
		\item підпрограму (процедуру чи функцію), яка дає змогу наблизити $2\pi$-
		періодичний сигнал рядом Фур’є з певною точністю, яка пов’язана з
		кількістю доданків N ряду;
		\item підпрограму (процедуру чи функцію) для обчислення середньої
		абсолютної похибки отриманого наближення;
		\item підпрограму (процедуру чи функцію) для зберігання у файлі:
		\begin{enumerate}[label=\alph*)]
			\item параметра N;
			\item визначених коефіцієнтів тригонометричного ряду Фур’є;
			\item середню абсолютну похибку наближення;
		\end{enumerate}
		\item основну програму для виконання наближення заданого $2\pi$ - періодичного
		сигналу тригонометричним рядом Фур'є, яка дає змогу виконати
		геометричне відображення аналогового $2\pi$- періодичного сигналу та його
		наближення кривою, описаною рядом Фур'є.
	\end{enumerate}
	
	\subsection*{Варіант №6}
	
	\begin{equation}
		f(t) = 
		\begin{cases}
			t, & \hspace{-12pt} -\pi < t \leq 0,\\
			2t, & 0 < t < \pi
		\end{cases}\nonumber
	\end{equation}

	\section*{Теоретичні відомості}
	
	\section*{Хід роботи}	

	\subsection*{Код програми}
	Файл \textit{main.py}:
	{\small
		\begin{lstlisting}
#
		\end{lstlisting}
	}
	
	\section*{Висновки}
	Під час виконання лабораторної роботи я наблизив рядом Фур’є аналоговий $2\pi$- періодичний
	сигнал, виконав геометричне зображення функції, якою описано цей сигнал, та
	кривої, яку подано рядом Фур’є. Визначив середню абсолютну похибку
	наближення.
	    
\end{normalsize}
\end{document}
