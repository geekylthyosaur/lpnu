\documentclass[14pt]{extreport}
\usepackage{cmap}
\usepackage[utf8]{inputenc}
\usepackage[english,ukrainian]{babel}
\usepackage{graphicx}
\usepackage{geometry}
\usepackage{listings}
\usepackage{amsmath}
\usepackage{float}
\geometry{
	a4paper,
	left=20mm,
	right=20mm,
	top=20mm,
	bottom=20mm
}
\lstset{
	columns=fullflexible, 
	frame=single, 
	breaklines=true, 
	tabsize=4,
	postbreak=\raisebox{0ex}[0ex][0ex]{\ensuremath{\hookrightarrow\space}},
	showstringspaces=false,% no symbol for spaces in strings
}
\graphicspath{ {./pictures} }
\setlength{\parindent}{4em}

\newcommand\subject{Бази даних}
\newcommand\lecturer{асистент кафедри ПЗ\\Білоіваненко М.В.}
\newcommand\teacher{асистент кафедри ПЗ\\Білоіваненко М.В.}
\newcommand\mygroup{ПЗ-32}
\newcommand\lab{7}
\newcommand\theme{Індекси у плані виконання запитів}
\newcommand\purpose{Обрати типові запити на вибірку. Проаналізувати їхні плани виконання із різними операторами порівняння значень. Проаналізувати структуру плану виконання запиту, що містить декілька з’єднань таблиць. Встановити таблиці, які будуть мати постійний приріст даних при експлуатації}

\begin{document}
\begin{normalsize}
	\begin{titlepage}
		\thispagestyle{empty}
		\begin{center}
			\textbf{МІНІСТЕРСТВО ОСВІТИ І НАУКИ УКРАЇНИ\\
				НАЦІОНАЛЬНИЙ УНІВЕРСИТЕТ "ЛЬВІВСЬКА ПОЛІТЕХНІКА"}
		\end{center}
		\begin{flushright}
			Інститут \textbf{КНІТ}\\
			Кафедра \textbf{ПЗ}
		\end{flushright}
		\vspace{200pt}
		\begin{center}
			\textbf{ЗВІТ}\\
			\vspace{10pt}
			До лабораторної роботи № \lab\\
			\textbf{На тему}: “\textit{\theme}”\\
			\textbf{З дисципліни}: “\subject”
		\end{center}
		\vspace{40pt}
		\begin{flushright}
			
			\textbf{Лектор}:\\
			\lecturer\\
			\vspace{10pt}
			\textbf{Виконав}:\\
			
			студент групи \mygroup\\
			Коваленко Д.М.\\
			\vspace{10pt}
			\textbf{Прийняв}:\\
			
			\teacher\\
			
			\vspace{28pt}
			«\rule{1cm}{0.15mm}» \rule{1.5cm}{0.15mm} 2023 р.\\
			$\sum$ = \rule{1cm}{0.15mm}……………\\
			
		\end{flushright}
		\vspace{\fill}
		\begin{center}
			\textbf{Львів — 2023}
		\end{center}
	\end{titlepage}
		
	\begin{description}
		\item[Тема.] \theme.
		\item[Мета.] \purpose.
	\end{description}

	\section*{Лабораторне завдання}
	Обрати типові запити на вибірку. Проаналізувати їхні плани виконання із різними операторами порівняння значень. Проаналізувати структуру плану виконання запиту, що містить декілька з’єднань таблиць. Встановити таблиці, які будуть мати постійний приріст даних при експлуатації
	
	\begin{enumerate}
\item Додати індекси до колонок із зовнішніми ключами, якщо це не зроблено СУБД автоматично.
\item Додати комплексний індекс та індекс на функцію, встановити факт їх використання за планом виконання.
\item Додати 3 індекси на важливі колонки для підвищення ефективності запитів.
	\end{enumerate}
	
	\section*{Хід роботи}
	
	Створення індексів до колонок із зовнішніми ключами.
	\begin{small}
		\begin{lstlisting}[language=sql]
create index idx_driver_route_driver_id on public.driver_route (driver_id);
create index idx_driver_route_vehicle_id on public.driver_route (vehicle_id);
create index idx_driver_route_route_id on public.driver_route (route_id);
create index idx_route_first_stop_id on public.route (first_stop_id);
create index idx_route_last_stop_id on public.route (last_stop_id);
create index idx_route_stop_route_id on public.route_stop (route_id);
create index idx_route_stop_stop_id on public.route_stop (stop_id);
create index idx_ticket_route_id on public.ticket (route_id);
create index idx_ticket_vehicle_id on public.ticket (vehicle_id);
create index idx_schedule_vehicle_id on public.schedule (vehicle_id);
create index idx_schedule_route_id on public.schedule (route_id);
create index idx_schedule_stop_id on public.schedule (stop_id);
create index idx_vehicle_maintenance_vehicle_id on public.vehicle_maintenance (vehicle_id);
		\end{lstlisting}
	\end{small}
	
	
	Створення комплексного індексу та індексу на функцію.
	\begin{small}
		\begin{lstlisting}[language=sql]
create index idx_route_vehicle_route_next_stop on public.route_vehicle (route_id, next_stop);
create index idx_average_rating on public.feedback (calculate_average_rating(driver_id));
		\end{lstlisting}
	\end{small}
	
	Створення 3х індексів на важливі колонки для підвищення ефективності запитів.
	\begin{small}
		\begin{lstlisting}[language=sql]
create index idx_driver_first_name on public.driver (first_name);
create index idx_route_name on public.route (name);
create index idx_vehicle_license_plate on public.vehicle (license_plate);
		\end{lstlisting}
	\end{small}
	
	\section*{Висновок}
	Під час виконання лабораторної роботи я обрав типові запити на вибірку. Проаналізував їхні плани виконання із різними операторами порівняння значень. Проаналізував структуру плану виконання запиту, що містить декілька з’єднань таблиць. Встановив таблиці, які будуть мати постійний приріст даних при експлуатації.
	 
\end{normalsize}
\end{document}
