\documentclass[oneside,14pt]{extarticle}
\usepackage[utf8]{inputenc}
\usepackage[english,ukrainian]{babel}
\usepackage{amssymb,amsfonts,amsmath,amsthm,mathtext,textcomp}

\usepackage[includehead, headsep=0pt, footskip=0pt, top=2cm, bottom=2cm, left=2cm, right=1cm]{geometry}
\usepackage{indentfirst}
\usepackage[onehalfspacing]{setspace}
\usepackage[headings]{fancyhdr}
\usepackage{etoolbox}
\usepackage{flafter}
\usepackage{listings}
\usepackage[center]{titlesec}
\titlelabel{\thetitle.\quad}
\usepackage{array}
\fancyhf{}
\renewcommand{\headrulewidth}{0pt}
\pagestyle{fancy}
\fancyfoot[R]{\thepage}
\lstset{breaklines=true,}

\begin{document}
\begin{titlepage}
	\begin{center}
		Національний університет “Львівська політехніка”\\
		Кафедра програмного забезпечення
		
		\vspace{170pt}
		\textbf{КУРСОВА РОБОТА}\\
		\textbf{з дисципліни <<Об’єктно-орієнтоване програмування>>}\\
		\textbf{На тему:}\\
		<<>>
		\vspace*{40pt}
		
		\begin{flushright}
			Стедента групи ПЗ-22\\
			спеціальності 6.121\\
			“Програмна інженерія”\\
			Коваленко Д.М.
			\bigbreak
			
			Керівник: доцент кафедри ПЗ,\\
			к.т.н., доцент Коротєєва Т. О.
			\bigbreak
			
			Національна шкала \rule{4cm}{0.15mm}\\			
			Кількість \rule{1cm}{0.15mm} балів  Оцінка ECTS \rule{1cm}{0.15mm}
			\bigbreak
			
			Члени комісії \rule{1cm}{0.15mm} \rule{4cm}{0.15mm}\\
			\rule{1cm}{0.15mm} \rule{4cm}{0.15mm}\\
			\rule{1cm}{0.15mm} \rule{4cm}{0.15mm}
		\end{flushright}
		\vspace{\fill}
		Львів — 2022
	\end{center}
\end{titlepage}
\setcounter{page}{2}
\tableofcontents
\newpage
\section*{Завдання}
\addcontentsline{toc}{section}{Завдання}
\begin{center}
	на курсову роботу з дисципліни «Об’єктно-орієнтоване програмування»\\
	студента групи ПЗ-22 Коваленка Дмитра
	\bigbreak
	\textbf{Тема: <<>>}
\end{center}
Створити таблицю у візуальному середовищі

\begin{tabular}{|c|c|c|c|c|c|c|}
	\hline
	№ & Прізвище & Вік & Група крові & Резус-фактор & Артеріальний тиск & Пульс\\
	\hline
\end{tabular}

\begin{enumerate}
	\item Швидким алгоритмом відсортувати записи за показником артеріального
	тиску.
	\item Згрупувати людей за однаковими групами крові та однаковими резус-
	факторами.
	\item Згрупувати людей за однаковими резус-факторами та відсортувати кожну
	групу за показником Пульсу.
	\item Визначити людей, які є універсальними донорами, а які є універсальними
	реципієнтами та сформувати загальну таблицю донорів та реципієнтів.
	\item Для вказаного показника Вік визначити пацієнтів з підвищеними показниками
	артеріального тиску та пульсу.
	\item Всім пацієнтам з нормальними артеріальним тиском вивести повідомлення
	<<Прізвище --- Здоровий!>>
\end{enumerate}
Для класу створити: 1) Конструктор за замовчуванням; 2) Конструктор з
параметрами; 3) конструктор копій; 4) перевизначити операції $>>$, $<<$ для
зчитування та запису у файл.
\newpage
\begin{center}
	\textbf{Зміст завдання та календарний план його виконання}
	
	\begin{tabular}{ | m{0.7cm} | m{14cm}| m{1.1cm} | }
		\hline
		№ з/п & Зміст завдання & Дата \\ 
		\hline
		1 & Здійснити аналiтичний огляд лiтератури за заданою темою та обгрунтувати вибір інструментальних засобів реалізації. & 09.10 \\
		\hline
		2 & Побудова UML діаграм & 10.10 \\
		\hline
		3 & Розробка алгоритмів реалізації & 13.10 \\
		\hline
		4 & Реалізація завдання (кодування) & 15.10 \\
		\hline
		5 & Формування інструкції користувача & 17.10 \\
		\hline
		6 & Оформлення звіту до курсової роботи згідно з вимогами Міжнародних стандартів, дотримуючись такої структури:
		
		·       зміст;
		
		·       алгоритм розв‘язку задачі у покроковому представленні;
		
		·       діаграми UML класів, прецедентів, послідовності виконання;
		
		·       код розробленої програми з коментарями;
		
		·       протокол роботи програми для кожного пункту завдання
		
		·       інструкція користувача та системні вимоги;
		
		·       опис виняткових ситуацій;
		
		·       структура файлу вхідних даних;
		
		·       висновки;
		
		·       список використаних джерел. & 18.10 \\
		\hline
	\end{tabular}
\end{center}
Завдання прийнято до виконання: \rule{4cm}{0.15mm} Коваленко Д.М.\\
Керівник роботи: \rule{4cm}{0.15mm} Коротєєва Т. О.

\section{Покроковий алгоритм розв‘язку задачі}
\subsection{Задача сортування}
\begin{list}{}{Алгоритм A.}
	\item [A1] 
	\item [A2] 
\end{list}

\subsection{Задача групування}
\begin{list}{}{Алгоритм A.}
	\item [A1] 
	\item [A2] 
\end{list}

\subsection{Згрупувати людей та відсортувати кожну групу}
\begin{list}{}{Алгоритм A.}
	\item [A1] 
	\item [A2] 
\end{list}

\subsection{Задача визначити групи людей зі списку за умовою}
\begin{list}{}{Алгоритм A.}
	\item [A1] 
	\item [A2] 
\end{list}

\subsection{Задача визначети людей за вказаними даними}
\begin{list}{}{Алгоритм A.}
	\item [A1] 
	\item [A2] 
\end{list}

\subsection{Задача виведення повідомлення}
\begin{list}{}{Алгоритм A.}
	\item [A1] 
	\item [A2] 
\end{list}

\section{Діаграми}
\subsection{UML діаграма класів}
\subsection{Діаграма прецедентів}
\subsection{Діаграма послідовності виконання}
\section{Код розробленої програми}
\begin{small}
	

\textbf{файл \textit{app.h}}
\begin{lstlisting}[language=c++]
#ifndef APP_H
#define APP_H

#include "list.h"
#include "mainwindow.h"
#include "ui_mainwindow.h"

class App
{
	private:
	List* list;
	Ui::MainWindow* ui;
	
	public:
	App() = default;
	App(Ui::MainWindow* ui);
	
	void addPerson();
	void removePerson();
	void updateTable();
	void healthyPeople();
	void highPressureAndRate(int age);
	void bestDonors();
	void bestRecipients();
	void donorsAndRecipients();
	void showDonorsTo(int i);
	void showRecipientsFrom(int i);
	void clearTable();
	void clearList();
	void readFromFile(QString fileName);
	void writeToFile(QString fileName);
	void sort(int columnIndex);
	
};

#endif // APP_H
\end{lstlisting}

\textbf{файл \textit{blood.h}}
\begin{lstlisting}[language=c++]
#ifndef BLOOD_H
#define BLOOD_H

#include "QString"

class Blood
{
	private:
	int mPressureHigh;
	int mPressureLow;
	bool mRhD;
	int mType;
	
	public:
	const QString BEST_DONOR = "O";
	const QString BEST_RECIPIENT = "AB";
	
	Blood() = default;
	Blood(QString s);
	Blood(int pressureH, int pressureL, bool rhd, int type);
	
	int     getPressureHigh()   const { return this->mPressureHigh; }
	int     getPressureLow()    const { return this->mPressureLow; }
	bool    getRhD()            const { return this->mRhD; }
	int     getType()           const { return this->mType; }
	
	QString getPressureStr();
	QString getRhDStr();
	QString getTypeStr();
	
	bool operator > (const Blood& other) const { return this->mPressureLow + this->mPressureHigh > other.mPressureHigh + other.mPressureLow; }
	bool operator < (const Blood& other) const { return this->mPressureLow + this->mPressureHigh < other.mPressureHigh + other.mPressureLow; }
};

#endif // BLOOD_H
\end{lstlisting}

\textbf{файл \textit{list.h}}
\begin{lstlisting}[language=c++]
#ifndef LIST_H
#define LIST_H

#include "QVector"
#include "QFile"

#include "person.h"

class List
{
	private:
	QVector<Person> mVec;
	int      partition(int columnIndex, int start, int end);
	
	public:
	List() = default;
	
	void     quickSort(int columnIndex, int start, int end);
	void     push(Person p);
	void     clear();
	Person * get(int i);
	int      len() const;
	
	friend void operator << (QFile &output, const List* l);
	friend void operator >> (QFile &input, List* l);
};

#endif // LIST_H
\end{lstlisting}

\textbf{файл \textit{mainwindow.h}}
\begin{lstlisting}[language=c++]
#ifndef MAINWINDOW_H
#define MAINWINDOW_H

#include <QMainWindow>

QT_BEGIN_NAMESPACE
namespace Ui { class MainWindow; }
QT_END_NAMESPACE

class MainWindow : public QMainWindow
{
	Q_OBJECT
	
	public:
	MainWindow(QWidget *parent = nullptr);
	~MainWindow();
	
	private slots:
	void on_actionOpen_triggered();
	void on_actionSave_triggered();
	void on_addPersonBtn_clicked();
	void on_actionby_Blood_Pressure_triggered();
	void on_actionType_and_RhD_triggered();
	void on_actionRhD_and_Heart_Rate_triggered();
	void on_healthyPeople_triggered();
	void on_highPressureAndRate_triggered();
	void on_actionDefault_triggered();
	void on_bestDonors_triggered();
	void on_bestRecipients_triggered();
	void on_donorsRecepients_triggered();
	void on_tableWidget_cellDoubleClicked(int row, int column);
	void on_actionClose_triggered();
	void on_actionRhD_triggered();
	
	private:
	Ui::MainWindow *ui;
};
#endif // MAINWINDOW_H

\end{lstlisting}

\textbf{файл \textit{person.h}}
\begin{lstlisting}[language=c++]
#ifndef PERSON_H
#define PERSON_H

#include "QString"
#include "QTextStream"
#include "blood.h"
#include "QDebug"

class Person
{
	private:
	int     mN;
	QString mSurname;
	int     mAge;
	Blood * mBlood;
	int     mHeartRate;
	
	public:
	Person() = default;
	Person(QString person);
	Person(int n, QString surname, int age, Blood* blood, int hr);
	
	Person(const Person &other);
	
	int     getN()          const { return this->mN;         }
	QString getSurname()    const { return this->mSurname;   }
	int     getAge()        const { return this->mAge;       }
	Blood * getBlood()      const { return this->mBlood;     }
	int     getHeartRate()  const { return this->mHeartRate; }
	
	bool    compare(const Person& other, const int flag) const;
	
	friend void operator << (QTextStream &output, const Person* p);
	friend void operator >> (QTextStream &input, Person* p);
};

#endif // PERSON_H

\end{lstlisting}

\textbf{файл \textit{app.cpp}}
\begin{lstlisting}[language=c++]
#include "app.h"
#include "QDebug"

App::App(Ui::MainWindow* ui)
{
	this->list = new List();
	this->ui = ui;
}

void App::addPerson()
{
	bool ok;
	if (ui->nLE->text().toInt(&ok) == 0)
	if (!ok)
	throw 1;
	if (ui->ageLE->text().toInt(&ok) == 0)
	if (!ok)
	throw 3;
	if (ui->bloodtypeLE->text() != "O" && ui->bloodtypeLE->text() != "A" && ui->bloodtypeLE->text() != "B" && ui->bloodtypeLE->text() != "AB")
	throw 4;
	if (!ui->bloodpressureLE->text().contains("/"))
	throw 5;
	if (ui->rhdLE->text() != "+" && ui->rhdLE->text() != "-")
	throw 6;
	if (ui->heartrateLE->text().toInt(&ok) == 0)
	if (!ok)
	throw 7;
	QString s = ui->nLE->text() + "," +
	ui->surnameLE->text() + "," +
	ui->ageLE->text() + "," +
	ui->bloodpressureLE->text() + " " +
	ui->bloodtypeLE->text() +
	ui->rhdLE->text() + "," +
	ui->heartrateLE->text();
	Person p = Person(s);
	this->list->push(p);
}

void App::updateTable()
{
	ui->tableHealthy->setVisible(false);
	ui->tableWidget->setVisible(true);
	ui->tableDonorsAndRecipients->setVisible(false);
	this->clearTable();
	for (int i = 0; i < list->len(); i++)
	{
		ui->tableWidget->insertRow(i);
		Person * p = this->list->get(i);
		for (int j = 0; j < 7; j++)
		{
			QTableWidgetItem * item = new QTableWidgetItem();
			item->setTextAlignment(Qt::AlignCenter);
			switch (j)
			{
				case 0:
				item->setText(QString::number(p->getN()));
				break;
				case 1:
				item->setText(p->getSurname());
				break;
				case 2:
				item->setText(QString::number(p->getAge()));
				break;
				case 3:
				item->setText(p->getBlood()->getTypeStr());
				break;
				case 4:
				item->setText(p->getBlood()->getRhDStr());
				break;
				case 5:
				item->setText(p->getBlood()->getPressureStr());
				break;
				case 6:
				item->setText(QString::number(p->getHeartRate()));
				break;
			}
			ui->tableWidget->setItem(i, j, item);
		}
	}
}

void App::healthyPeople()
{
	ui->tableHealthy->setVisible(true);
	ui->tableWidget->setVisible(false);
	ui->tableDonorsAndRecipients->setVisible(false);
	ui->tableHealthy->setColumnCount(2);
	ui->tableHealthy->setRowCount(0);
	ui->tableHealthy->setColumnWidth( 0, 360 );
	ui->tableHealthy->setColumnWidth( 1, 360 );
	QStringList labels;
	labels << "Surname" << "Message";
	ui->tableHealthy->setHorizontalHeaderLabels(labels);
	int r = 0;
	for (int i = 0; i < list->len(); i++)
	{
		Person * p = this->list->get(i);
		bool healthy = p->getBlood()->getPressureHigh() <= 140 &&
		p->getBlood()->getPressureLow() <= 100 &&
		p->getBlood()->getPressureHigh() >= 100 &&
		p->getBlood()->getPressureLow() >= 60;
		
		if (healthy)
		ui->tableHealthy->insertRow(r);
		else
		continue;
		for (int j = 0; j < 2; j++)
		{
			QTableWidgetItem * item = new QTableWidgetItem();
			item->setTextAlignment(Qt::AlignCenter);
			switch (j)
			{
				case 0:
				item->setText(p->getSurname());
				break;
				case 1:
				item->setText(QString::fromStdString("Healthy"));
				break;
			}
			ui->tableHealthy->setItem(r, j, item);
		}
		r++;
	}
}

void App::highPressureAndRate(int age)
{
	ui->tableHealthy->setVisible(false);
	ui->tableWidget->setVisible(true);
	ui->tableDonorsAndRecipients->setVisible(false);
	this->clearTable();
	int r = 0;
	for (int i = 0; i < list->len(); i++)
	{
		Person * p = this->list->get(i);
		bool highPressureAndRate = p->getHeartRate() >= 100 &&
		p->getBlood()->getPressureHigh() >= 140 &&
		p->getBlood()->getPressureLow() >= 100 &&
		p->getAge() == age;
		
		if (highPressureAndRate)
		ui->tableWidget->insertRow(r);
		else
		continue;
		for (int j = 0; j < 7; j++)
		{
			QTableWidgetItem * item = new QTableWidgetItem();
			item->setTextAlignment(Qt::AlignCenter);
			switch (j)
			{
				case 0:
				item->setText(QString::number(p->getN()));
				break;
				case 1:
				item->setText(p->getSurname());
				break;
				case 2:
				item->setText(QString::number(p->getAge()));
				break;
				case 3:
				item->setText(p->getBlood()->getTypeStr());
				break;
				case 4:
				item->setText(p->getBlood()->getRhDStr());
				break;
				case 5:
				item->setText(p->getBlood()->getPressureStr());
				break;
				case 6:
				item->setText(QString::number(p->getHeartRate()));
				break;
			}
			ui->tableWidget->setItem(r, j, item);
		}
		r++;
	}
}

void App::bestDonors()
{
	ui->tableHealthy->setVisible(false);
	ui->tableWidget->setVisible(true);
	ui->tableDonorsAndRecipients->setVisible(false);
	this->clearTable();
	int r = 0;
	for (int i = 0; i < list->len(); i++)
	{
		Person * p = this->list->get(i);
		bool bestDonor = p->getBlood()->getTypeStr() == p->getBlood()->BEST_DONOR;
		
		if (bestDonor)
		ui->tableWidget->insertRow(r);
		else
		continue;
		for (int j = 0; j < 7; j++)
		{
			QTableWidgetItem * item = new QTableWidgetItem();
			item->setTextAlignment(Qt::AlignCenter);
			switch (j)
			{
				case 0:
				item->setText(QString::number(p->getN()));
				break;
				case 1:
				item->setText(p->getSurname());
				break;
				case 2:
				item->setText(QString::number(p->getAge()));
				break;
				case 3:
				item->setText(p->getBlood()->getTypeStr());
				break;
				case 4:
				item->setText(p->getBlood()->getRhDStr());
				break;
				case 5:
				item->setText(p->getBlood()->getPressureStr());
				break;
				case 6:
				item->setText(QString::number(p->getHeartRate()));
				break;
			}
			ui->tableWidget->setItem(r, j, item);
		}
		r++;
	}
}

void App::bestRecipients()
{
	ui->tableHealthy->setVisible(false);
	ui->tableWidget->setVisible(true);
	ui->tableDonorsAndRecipients->setVisible(false);
	this->clearTable();
	int r = 0;
	for (int i = 0; i < list->len(); i++)
	{
		Person * p = this->list->get(i);
		bool bestRecipient = p->getBlood()->getTypeStr() == p->getBlood()->BEST_RECIPIENT;
		
		if (bestRecipient)
		ui->tableWidget->insertRow(r);
		else
		continue;
		for (int j = 0; j < 7; j++)
		{
			QTableWidgetItem * item = new QTableWidgetItem();
			item->setTextAlignment(Qt::AlignCenter);
			switch (j)
			{
				case 0:
				item->setText(QString::number(p->getN()));
				break;
				case 1:
				item->setText(p->getSurname());
				break;
				case 2:
				item->setText(QString::number(p->getAge()));
				break;
				case 3:
				item->setText(p->getBlood()->getTypeStr());
				break;
				case 4:
				item->setText(p->getBlood()->getRhDStr());
				break;
				case 5:
				item->setText(p->getBlood()->getPressureStr());
				break;
				case 6:
				item->setText(QString::number(p->getHeartRate()));
				break;
			}
			ui->tableWidget->setItem(r, j, item);
		}
		r++;
	}
}

void App::donorsAndRecipients()
{
	ui->tableHealthy->setVisible(false);
	ui->tableWidget->setVisible(true);
	ui->tableDonorsAndRecipients->setVisible(false);
	ui->tableWidget->setRowCount(0);
	ui->tableWidget->setColumnCount(5);
	QStringList labels;
	labels << "N" << "Surname" << "Age" << "Donor to" << "Recipient from";
	ui->tableWidget->setHorizontalHeaderLabels(labels);
	ui->tableWidget->setColumnWidth( 0, 40 );
	ui->tableWidget->setColumnWidth( 1, 400 );
	ui->tableWidget->setColumnWidth( 2, 40 );
	ui->tableWidget->setColumnWidth( 3, 100 );
	ui->tableWidget->setColumnWidth( 4, 100 );
	
	for (int i = 0; i < list->len(); i++)
	{
		ui->tableWidget->insertRow(i);
		Person * p = this->list->get(i);
		for (int j = 0; j < 5; j++)
		{
			QTableWidgetItem * item = new QTableWidgetItem();
			item->setTextAlignment(Qt::AlignCenter);
			switch (j)
			{
				case 0:
				item->setText(QString::number(p->getN()));
				break;
				case 1:
				item->setText(p->getSurname());
				break;
				case 2:
				item->setText(QString::number(p->getAge()));
				break;
				case 3:
				item->setText(QString::fromStdString("..."));
				break;
				case 4:
				item->setText(QString::fromStdString("..."));
				break;
			}
			ui->tableWidget->setItem(i, j, item);
		}
	}
}

void App::showDonorsTo(int i)
{
	this->clearTable();
	int personBloodType = list->get(i)->getBlood()->getType();
	int r = 0;
	for (int i = 0; i < list->len(); i++)
	{
		Person * p = this->list->get(i);
		bool donorTo = p->getBlood()->getType() >= personBloodType;
		if (personBloodType == 2 && p->getBlood()->getType() == 3)
		continue;
		if (donorTo)
		ui->tableWidget->insertRow(r);
		else
		continue;
		for (int j = 0; j < 7; j++)
		{
			QTableWidgetItem * item = new QTableWidgetItem();
			item->setTextAlignment(Qt::AlignCenter);
			switch (j)
			{
				case 0:
				item->setText(QString::number(p->getN()));
				break;
				case 1:
				item->setText(p->getSurname());
				break;
				case 2:
				item->setText(QString::number(p->getAge()));
				break;
				case 3:
				item->setText(p->getBlood()->getTypeStr());
				break;
				case 4:
				item->setText(p->getBlood()->getRhDStr());
				break;
				case 5:
				item->setText(p->getBlood()->getPressureStr());
				break;
				case 6:
				item->setText(QString::number(p->getHeartRate()));
				break;
			}
			ui->tableWidget->setItem(r, j, item);
		}
		r++;
	}
}

void App::showRecipientsFrom(int i)
{
	this->clearTable();
	int personBloodType = list->get(i)->getBlood()->getType();
	int r = 0;
	for (int i = 0; i < list->len(); i++)
	{
		Person * p = this->list->get(i);
		bool recipientFrom = p->getBlood()->getType() <= personBloodType;
		if (personBloodType == 3 && p->getBlood()->getType() == 2)
		continue;
		if (recipientFrom)
		ui->tableWidget->insertRow(r);
		else
		continue;
		for (int j = 0; j < 7; j++)
		{
			QTableWidgetItem * item = new QTableWidgetItem();
			item->setTextAlignment(Qt::AlignCenter);
			switch (j)
			{
				case 0:
				item->setText(QString::number(p->getN()));
				break;
				case 1:
				item->setText(p->getSurname());
				break;
				case 2:
				item->setText(QString::number(p->getAge()));
				break;
				case 3:
				item->setText(p->getBlood()->getTypeStr());
				break;
				case 4:
				item->setText(p->getBlood()->getRhDStr());
				break;
				case 5:
				item->setText(p->getBlood()->getPressureStr());
				break;
				case 6:
				item->setText(QString::number(p->getHeartRate()));
				break;
			}
			ui->tableWidget->setItem(r, j, item);
		}
		r++;
	}
}

void App::clearTable()
{
	ui->tableWidget->setColumnCount(7);
	ui->tableWidget->setRowCount(0);
	QStringList labels;
	labels << "N" << "Surname" << "Age" << "Type" << "RhD" << "Pressure" << "Rate";
	ui->tableWidget->setHorizontalHeaderLabels(labels);
	ui->tableWidget->setColumnWidth( 0, 40 );
	ui->tableWidget->setColumnWidth( 1, 400 );
	ui->tableWidget->setColumnWidth( 2, 40 );
	ui->tableWidget->setColumnWidth( 3, 40 );
	ui->tableWidget->setColumnWidth( 4, 20 );
	ui->tableWidget->setColumnWidth( 5, 80 );
	ui->tableWidget->setColumnWidth( 6, 40 );
}

void App::readFromFile(QString fileName)
{
	QFile file(fileName);
	this->list->clear();
	file >> this->list;
	ui->actionClose->setEnabled(true);
}

void App::writeToFile(QString fileName)
{
	if (this->list->len() == 0)
	throw 1;
	QFile file(fileName);
	file << this->list;
	ui->actionClose->setEnabled(true);
}

void App::sort(int columnIndex)
{
	this->list->quickSort(columnIndex, 0, this->list->len() - 1);
}

void App::clearList()
{
	this->list->clear();
	ui->actionClose->setEnabled(false);
}

\end{lstlisting}

\textbf{файл \textit{blood.cpp}}
\begin{lstlisting}[language=c++]
#include "blood.h"

#include "QStringList"
#include "QDebug"

Blood::Blood(int pressureH, int pressureL, bool rhd, int type)
{
	this->mPressureLow = pressureL;
	this->mPressureHigh = pressureH;
	this->mRhD = rhd;
	this->mType = type;
}

Blood::Blood(QString s)
{
	QStringList tokens = s.split(" ");
	auto pressure_tokens = tokens[0].split("/");
	this->mPressureHigh = pressure_tokens[0].toInt();
	this->mPressureLow = pressure_tokens[1].toInt();
	if (tokens[1].right(1) == "+")
	this->mRhD = true;
	else if (tokens[1].right(1) == "-")
	this->mRhD = false;
	tokens[1].chop(1);
	if (tokens[1] == "O")
	this->mType = 1;
	else if (tokens[1] == "A")
	this->mType = 2;
	else if (tokens[1] == "B")
	this->mType = 3;
	else if (tokens[1] == "AB")
	this->mType = 4;
}

QString Blood::getPressureStr()
{
	return QString::number(this->mPressureHigh) + "/" + QString::number(this->mPressureLow);
}

QString Blood::getRhDStr()
{
	if (this->mRhD)
	return QString::fromStdString("+");
	else
	return QString::fromStdString("-");
}

QString Blood::getTypeStr()
{
	switch (this->mType)
	{
		case 1:
		return QString::fromStdString("O");
		case 2:
		return QString::fromStdString("A");
		case 3:
		return QString::fromStdString("B");
		case 4:
		return QString::fromStdString("AB");
		default:
		return QString::fromStdString("ERROR");
	}
}

\end{lstlisting}

\textbf{файл \textit{list.cpp}}
\begin{lstlisting}[language=c++]
#include "list.h"
#include "QTextStream"
#include "QDebug"

using namespace std;

void List::push(Person p)
{
	this->mVec.append(p);
}

Person * List::get(int i)
{
	return &this->mVec[i];
}

int List::len() const
{
	return this->mVec.length();
}

int List::partition(int columnIndex, int start, int end)
{
	int pivotIndex = end;
	
	int i = (start - 1);
	
	for (int j = start; j < end; j++) {
		if (this->mVec[pivotIndex].compare(this->mVec[j], columnIndex)) {
			//if (this->mVec[j] < this->mVec[pivotIndex]) {
				i++;
				this->mVec.swapItemsAt(i, j);
			}
		}
		
		this->mVec.swapItemsAt(i + 1, end);
		
		return (i + 1);
	}
	
	void List::quickSort(int columnIndex, int start, int end)
	{
		if (start < end)
		{
			int pivot = this->partition(columnIndex, start, end);
			
			this->quickSort(columnIndex, start, pivot-1);
			this->quickSort(columnIndex, pivot+1, end);
		}
	}
	
	void List::clear()
	{
		this->mVec.clear();
	}
	
	// file << list
	void operator << (QFile &output, const List* l)
	{
		if (output.open(QIODevice::ReadWrite))
		{
			QTextStream stream(&output);
			for (Person p : l->mVec)
			{
				stream << &p;
			}
		}
	}
	
	// file >> list
	void operator >> (QFile &input, List* l)
	{
		if (input.open(QIODevice::ReadOnly))
		{
			QTextStream in(&input);
			while (!in.atEnd())
			{
				QString line = input.readLine();
				Person p = Person(line);
				l->push(p);
			}
			input.close();
		}
	}

\end{lstlisting}

\textbf{файл \textit{main.cpp}}
\begin{lstlisting}[language=c++]
#include "mainwindow.h"

#include <QApplication>

int main(int argc, char *argv[])
{
	QApplication a(argc, argv);
	MainWindow w;
	w.show();
	return a.exec();
}

\end{lstlisting}

\textbf{файл \textit{mainwindow.cpp}}
\begin{lstlisting}[language=c++]
#include "mainwindow.h"
#include "ui_mainwindow.h"
#include "person.h"
#include "list.h"
#include "app.h"
#include "QFileDialog"
#include "QInputDialog"
#include "QDebug"
#include "QMessageBox"

App* app;
bool fileChanged = false;

MainWindow::MainWindow(QWidget *parent)
: QMainWindow(parent)
, ui(new Ui::MainWindow)
{
	ui->setupUi(this);
	ui->tableHealthy->setVisible(false);
	ui->tableDonorsAndRecipients->setVisible(false);
	ui->tableWidget->setColumnWidth( 0, 40 );
	ui->tableWidget->setColumnWidth( 1, 400 );
	ui->tableWidget->setColumnWidth( 2, 40 );
	ui->tableWidget->setColumnWidth( 3, 40 );
	ui->tableWidget->setColumnWidth( 4, 20 );
	ui->tableWidget->setColumnWidth( 5, 80 );
	ui->tableWidget->setColumnWidth( 6, 40 );
	ui->actionClose->setEnabled(false);
	QStringList labels;
	labels << "N" << "Surname" << "Age" << "Type" << "RhD" << "Pressure" << "Rate";
	ui->tableWidget->setHorizontalHeaderLabels(labels);
	ui->tableWidget->horizontalHeader()->setSectionResizeMode (QHeaderView::Fixed);
	app = new App(ui);
	auto header = ui->tableWidget->horizontalHeader();
	connect(header, &QHeaderView::sectionClicked, [this](int columnIndex) {
		app->sort(columnIndex);
		app->updateTable();
	});
}

MainWindow::~MainWindow()
{
	delete ui;
}

void MainWindow::on_actionOpen_triggered()
{
	QString fileName = QFileDialog::getOpenFileName(this,
	tr("Open File"), "/home/dmytro/", tr("Data file (*.csv)"));
	if (!fileName.isEmpty())
	{
		try
		{
			app->readFromFile(fileName);
			app->updateTable();
		}
		catch (int err)
		{
			QMessageBox msgBox;
			msgBox.setIcon(QMessageBox::Critical);
			msgBox.setWindowTitle("Error");
			if (err == 1) msgBox.setText("This file is corrupted!");
			msgBox.exec();
			return;
		}
	}
}

void MainWindow::on_actionSave_triggered()
{
	QString fileName = QFileDialog::getSaveFileName(this,
	tr("Save File"), "/home/dmytro/", tr("Data file (*.csv)"));
	if (!fileName.isEmpty())
	{
		try
		{
			app->writeToFile(fileName);
			fileChanged = false;
		}
		catch (int err)
		{
			QMessageBox msgBox;
			msgBox.setIcon(QMessageBox::Warning);
			msgBox.setWindowTitle("Warning");
			if (err == 1) msgBox.setText("Nothing to save!");
			msgBox.exec();
			return;
		}
	}
}

void MainWindow::on_addPersonBtn_clicked()
{
	try
	{
		app->addPerson();
		app->updateTable();
		fileChanged = true;
	}
	catch (int err)
	{
		QMessageBox msgBox;
		msgBox.setIcon(QMessageBox::Critical);
		msgBox.setWindowTitle("Error");
		if (err == 1) msgBox.setText("N field has invalid value!");
		else if (err == 2) msgBox.setText("Surname field has invalid value!");
		else if (err == 3) msgBox.setText("Age field has invalid value!");
		else if (err == 4) msgBox.setText("Blood Type field has invalid value!");
		else if (err == 5) msgBox.setText("Blood Pressure field has invalid value!");
		else if (err == 6) msgBox.setText("RhD field has invalid value!");
		else if (err == 7) msgBox.setText("Heart Rate field has invalid value!");
		msgBox.exec();
		return;
	}
}

void MainWindow::on_actionby_Blood_Pressure_triggered()
{
	app->sort(0);
	app->updateTable();
}

void MainWindow::on_actionType_and_RhD_triggered()
{
	app->sort(1);
	app->updateTable();
}

void MainWindow::on_actionRhD_triggered()
{
	app->sort(3);
	app->updateTable();
}

void MainWindow::on_actionRhD_and_Heart_Rate_triggered()
{
	app->sort(2);
	app->sort(2);
	app->sort(2);
	app->sort(2);
	app->updateTable();
}

void MainWindow::on_healthyPeople_triggered()
{
	app->healthyPeople();
}

void MainWindow::on_highPressureAndRate_triggered()
{
	int age = QInputDialog::getInt(this, "Enter", "Enter Age:");
	if (age)
	app->highPressureAndRate(age);
}

void MainWindow::on_actionDefault_triggered()
{
	app->updateTable();
}

void MainWindow::on_bestDonors_triggered()
{
	app->bestDonors();
}

void MainWindow::on_bestRecipients_triggered()
{
	app->bestRecipients();
}

void MainWindow::on_donorsRecepients_triggered()
{
	app->donorsAndRecipients();
}

void MainWindow::on_tableWidget_cellDoubleClicked(int row, int column)
{
	if (ui->tableWidget->columnCount() == 5) {
		if (column == 3) app->showDonorsTo(row);
		else if (column == 4) app->showRecipientsFrom(row);
	}
}

void MainWindow::on_actionClose_triggered()
{
	if (!fileChanged)
	{
		app->clearTable();
		app->clearList();
		return;
	}
	QMessageBox msgBox;
	msgBox.setText("The file has been modified.");
	msgBox.setInformativeText("Exit without saving?");
	msgBox.setStandardButtons(QMessageBox::Discard | QMessageBox::Cancel);
	msgBox.setDefaultButton(QMessageBox::Save);
	int ret = msgBox.exec();
	
	switch (ret) {
		case QMessageBox::Discard:
		app->clearTable();
		app->clearList();
		break;
		case QMessageBox::Cancel:
		break;
		default:
		break;
	}
}

\end{lstlisting}

\textbf{файл \textit{person.cpp}}
\begin{lstlisting}[language=c++]
#include "person.h"
#include "QDebug"

Person::Person(int n, QString surname, int age, Blood* blood, int heartRate)
{
	this->mN = n;
	this->mSurname = surname;
	this->mAge = age;
	this->mBlood = blood;
	this->mHeartRate = heartRate;
}

Person::Person(QString person)
{
	QStringList tokens = person.split(",");
	if (tokens.length() != 5)
	throw 1;
	this->mN = tokens[0].toInt();
	this->mSurname = tokens[1];
	this->mAge = tokens[2].toInt();
	this->mBlood = new Blood(tokens[3]);
	this->mHeartRate = tokens[4].toInt();
}

Person::Person(const Person &other)
{
	this->mN = other.getN();
	this->mSurname = other.getSurname();
	this->mAge = other.getAge();
	this->mBlood = other.getBlood();
	this->mHeartRate = other.getHeartRate();
}

bool Person::compare(const Person& other, const int flag) const
{
	int thisPressure;
	int otherPressure;
	bool thisRhD;
	bool otherRhD;
	int thisType;
	int otherType;
	int thisHeartRate;
	int otherHeartRate;
	switch (flag)
	{
		case 0: // pressure
		thisPressure = this->getBlood()->getPressureHigh() + this->getBlood()->getPressureLow();
		otherPressure = other.getBlood()->getPressureHigh() + other.getBlood()->getPressureLow();
		return thisPressure > otherPressure;
		case 1: // rhd
		thisRhD = this->getBlood()->getRhD();
		otherRhD = other.getBlood()->getRhD();
		return thisRhD > otherRhD;
		case 2: //rhd + rate
		thisRhD = this->getBlood()->getRhD();
		otherRhD = other.getBlood()->getRhD();
		thisHeartRate = this->getHeartRate();
		otherHeartRate = other.getHeartRate();
		return thisRhD == otherRhD && thisHeartRate > otherHeartRate;
		case 3: // Type
		thisType = this->getBlood()->getType();
		otherType = other.getBlood()->getType();
		return thisType > otherType;
	}
	return false;
}

void operator << (QTextStream &output, const Person* p)
{
	output << p->getN()
	<< ","
	<< p->getSurname()
	<< ","
	<< p->getAge()
	<< ","
	<< p->getBlood()->getPressureStr()
	<< " "
	<< p->getBlood()->getTypeStr()
	<< p->getBlood()->getRhDStr()
	<< ","
	<< p->getHeartRate()
	<< Qt::endl;
}

\end{lstlisting}
\end{small}

\section{Протокол роботи програми}
\subsection{Пункт 1}
\subsection{Пункт 2}
\subsection{Пункт 3}
\subsection{Пункт 4}
\subsection{Пункт 5}
\section{Інструкція користувача та системні вимоги}
\subsection{Інструкція користувача}
\subsection{Системні вимоги}
\section{Опис виняткових ситуацій}
\section{Структура файлу вхідних даних}
\section*{Висновки}
\addcontentsline{toc}{section}{Висновки}
\section*{Список використаної літератури}
\addcontentsline{toc}{section}{Список використаної літератури}
\end{document}
