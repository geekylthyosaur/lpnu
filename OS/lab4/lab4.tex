\documentclass{article}
\usepackage{cmap}
\usepackage[utf8]{inputenc}
\usepackage[english,ukrainian]{babel}
\usepackage{graphicx}
\usepackage{geometry}
\usepackage{listings}
\usepackage{amsmath}
\usepackage{float}
\geometry{
	a4paper,
	left=20mm,
	right=20mm,
	top=20mm,
	bottom=20mm
}
\lstset{
	language=c,
	tabsize=4,
	keepspaces,
	showstringspaces=false,
}
\graphicspath{ {./pictures} }
\setlength{\parindent}{4em}

\newcommand\subject{Операційні системи}
\newcommand\lecturer{старший викладач кафедри ПЗ\\Грицай О.Д.}
\newcommand\teacher{старший викладач кафедри ПЗ\\Грицай О.Д.}
\newcommand\mygroup{ПЗ-22}
\newcommand\lab{4}
\newcommand\theme{Програмне створення та керування процесами в операційній системі
	LINUX}
\newcommand\purpose{Ознайомитися з багатопоточністю в ОС Linux. Навчитися працювати з
	процесами, в ОС Linux}

\begin{document}
\begin{normalsize}
	\begin{titlepage}
		\thispagestyle{empty}
		\begin{center}
			\textbf{МІНІСТЕРСТВО ОСВІТИ І НАУКИ УКРАЇНИ\\
				НАЦІОНАЛЬНИЙ УНІВЕРСИТЕТ "ЛЬВІВСЬКА ПОЛІТЕХНІКА"}
		\end{center}
		\begin{flushright}
			Інститут \textbf{КНІТ}\\
			Кафедра \textbf{ПЗ}
		\end{flushright}
		\vspace{200pt}
		\begin{center}
			\textbf{ЗВІТ}\\
			\vspace{10pt}
			До лабораторної роботи № \lab\\
			\textbf{На тему}: “\textit{\theme}”\\
			\textbf{З дисципліни}: “\subject”
		\end{center}
		\vspace{112pt}
		\begin{flushright}
			
			\textbf{Лектор}:\\
			\lecturer\\
			\vspace{28pt}
			\textbf{Виконав}:\\
			
			студент групи \mygroup\\
			Коваленко Д.М.\\
			\vspace{28pt}
			\textbf{Прийняла}:\\
			
			\teacher\\
			
			\vspace{28pt}
			«\rule{1cm}{0.15mm}» \rule{1.5cm}{0.15mm} 2022 р.\\
			$\sum$ = \rule{1cm}{0.15mm}……………\\
			
		\end{flushright}
		\vspace{\fill}
		\begin{center}
			\textbf{Львів — 2022}
		\end{center}
	\end{titlepage}
		
	\begin{description}
		\item[Тема.] \theme.
		\item[Мета.] \purpose.
	\end{description}

	\section*{Лабораторне завдання}
	\begin{enumerate}
		\item Створити окремий процес, і здійснити в ньому розв’язок задачі
		згідно варіанту у відповідності до порядкового номера у
		журнальному списку (підгрупи).
		\item Реалізувати розв’язок задачі у 2-ох, 4-ох, 8-ох процесах. Виміряти
		час роботи процесів за допомогою функцій WinAPI. Порівняти
		результати роботи в одному і в багатьох процесах
		\item Для кожного процесу реалізувати можливість його запуску,
		зупинення, завершення та примусове завершення («вбиття»).
		\item Реалізувати можливість зміни пріоритету виконання процесу
		\item Продемонструвати результати виконання роботи, а також кількість
		створених процесів у “Диспетчері задач”, або подібних утилітах (н-д,
		ProcessExplorer)
	\end{enumerate}
	\begin{center}
		2. Вивести посортовані по зростанню методом «бульбашки» рядки
		матриці матриці $N\cdot N$ ($N>1000$ задається користувачем, матриця
		визначається випадково).
	\end{center}

	\section*{Хід роботи}	
	\begin{lstlisting}                 
#include <stdio.h>
#include "stdlib.h"
#include <unistd.h>
#include <signal.h>
#include <sys/wait.h>
#include <iostream>
#include <string>
#include <fstream>
#include <sys/times.h>

using namespace std;

void gettimes(int pid);
void suspend(int pid);
void resume(int pid);
void kill(int pid);
void renice(int pid);

time_t time_begin;
int main(void) {
	int* pids;
	int pid;
	int status, PC = 0, N = 0;
	char op;
	
	while (N < 1) {
		cout << "Enter value of N (> 1000): ";
		cin >> N;
	}
	
	while (PC < 1) {
		cout << "Enter number of processes: ";
		cin >> PC;
	}
	
	pids = new int[PC];
	cout << "Generating random array " << N << "x" << N << " ... ";
	
	int** array = new int* [N];
	for (int i = 0; i < N; i++) array[i] = new int[N];
	
	srand(static_cast<unsigned int>(time(nullptr)));
	for (int i = 0; i < N; i++)
	{
		for (int j = 0; j < N; j++)
		{
			array[i][j] = rand();
		}
	}
	cout << "Done!" << endl;
	
	cout << "Writing array to file ... ";
	
	ofstream file;
	file.open("array.txt");
	for (int i = 0; i < N; i++)
	{
		for (int j = 0; j < N; j++)
		{
			if (j == N-1)
			file << array[i][j];
			else 
			file << array[i][j] << ",";
		}
		file << "\n";
	}
	file.close();
	
	cout << "Done!" << endl;
	
	time_begin = clock();
	
	for (int i = 0; i < PC; i++) {
		pid = fork();
		switch(pid) {
			case -1:
			// Error
			break;
			case 0:
			printf("Started child process (pid: %d)\n", getpid());
			char s1[10];
			char s2[10];
			char s3[10];
			sprintf(s1, "%d", N);
			sprintf(s2, "%d",(N/PC)*i);
			sprintf(s3, "%d", (N/PC)*(i+1));
			execl("/bin/footclient", "--", "./process", s1, s2, s3, NULL);
			return 0;
			default: 
			usleep(100000);
			string line;
			ifstream file("pid");
			getline(file, line);
			file.close();
			
			pids[i] = atoi(line.c_str());
			break;
		}
	}
	
	int i;
	while (true) {
		cout << "Suspend [s], Resume [r], Exit [e], Kill [k], Times [t], Priority [p]: ";
		cin >> op;
		if (op == 'e') break;
		cout << "Process index: ";
		cin >> i;
		if (op == 's') suspend(pids[i]);
		if (op == 'r') resume(pids[i]);
		if (op == 'k') kill(pids[i]);
		if (op == 't') gettimes(pids[i]);
		if (op == 'p') renice(pids[i]);
	}
	
	for (int i = 0; i < PC; i++) {
		kill(pids[i]);
	}
	
	return 0;
}

void gettimes(int pid) {
	char s[100];
	sprintf(s, "ps -o time %d | grep 0", pid);
	system(s);
}

void kill(int pid) {
	kill(pid, SIGKILL);
}

void resume(int pid) {
	kill(pid, SIGCONT);
}

void suspend(int pid) {
	cout << pid << endl;
	kill(pid, SIGSTOP);
}

void renice(int pid) {
	int nice;
	char cmd[100];
	cout << "New priority: ";
	cin >> nice;
	sprintf(cmd, "/bin/doas /bin/renice -n %d -p %d", nice, pid);
	system(cmd);
}


	\end{lstlisting}

	\begin{lstlisting}
#include <iostream>
#include <chrono>
#include <fstream>
#include <string>
#include <vector>
#include <unistd.h>

using namespace std;
using namespace std::chrono;

void bubble_sort(int* array, int N);
vector<string> split(string s, string delimiter);

int main(int argc, char** argv)
{
	ofstream f;
	f.open("pid");
	f << getpid();
	f.close();
	
	if (argc < 4) return -1;
	
	int Ncol = atoi(argv[1]);
	int startRow = atoi(argv[2]);
	int endRow = atoi(argv[3]);
	int Nrow = endRow - startRow;
	
	cout << "Reading array from file " << Ncol << "x" << Nrow << " ... ";
	
	string line;
	ifstream file("/home/dmytro/lpnu/OS/lab4/array.txt");
	int** array = new int* [Nrow];
	for (int i = 0; i < startRow; i++)
	{
		getline(file, line);
	}
	for (int i = 0; i < Nrow; i++)
	{
		getline(file, line);
		array[i] = new int[Ncol];
		
		auto vec = split(line, ",");
		for (int j = 0; j < Ncol; j++)
		{
			array[i][j] = atoi(vec[j].c_str());
		}
	}
	file.close();
	
	cout << "Done!" << endl;
	
	cout << "Sorting array ... " << endl;
	
	auto start = high_resolution_clock::now();
	for (int i = 0; i < Nrow; i++)
	{
		cout << "Sorting row " << i+startRow << endl;
		bubble_sort(array[i], Ncol);
	}
	auto stop = high_resolution_clock::now();
	auto duration = duration_cast<milliseconds>(stop - start);
	cout << "Sorting took: " << duration.count() << "ms" << endl;
	
	cout << "Sorted array: " << endl;
	
	for (int i = 0; i < Nrow; i++)
	{
		for (int j = 0; j < Ncol; j++)
		{
			cout << array[i][j] << ", ";
		}
		cout << endl;
	}
	
	return 0;}

void bubble_sort(int* array, int N) 
{
	for (int i = 0; i < N; i++)
	{
		for (int j = 0; j < N - i - 1; j++)
		{
			if (array[j] > array[j + 1])
			{
				swap(array[j], array[j + 1]);
			}
		}
	}
}

vector<string> split(string s, string delimiter) {
	size_t pos_start = 0, pos_end, delim_len = delimiter.length();
	string token;
	vector<string> res;
	
	while ((pos_end = s.find(delimiter, pos_start)) != string::npos) {
		token = s.substr(pos_start, pos_end - pos_start);
		pos_start = pos_end + delim_len;
		res.push_back(token);
	}
	
	res.push_back(s.substr(pos_start));
	return res;
}



	\end{lstlisting}
	
	\begin{figure}[H]
		\centering
		\includegraphics[scale=0.6]{v}
		\caption{Виконання програми}
	\end{figure}
	
	\section*{Висновок}
	Під час виконання лабораторної роботи я Ознайомився з багатопоточністю в ОС Linux. Навчився працювати з процесами, в ОС Linux. 	
	Навчився створювати нові процеси, призупиняти, завершувати та продовжувати їх роботу.
	Навчився отримувати та встановлювати пріоритет процесу.
	Навчився отримувати час виконання процесу.
	
	 
\end{normalsize}
\end{document}
