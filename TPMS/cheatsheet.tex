\documentclass{article}
\usepackage{cmap}
\usepackage[utf8]{inputenc}
\usepackage[english,ukrainian]{babel}
\usepackage{graphicx}
\usepackage{geometry}
\usepackage{listings}
\usepackage{indentfirst}
\usepackage{caption}
\usepackage{amsmath}
\geometry{
	a4paper,
	left=20mm,
	right=20mm,
	top=20mm,
	bottom=20mm
}
\lstset{
	extendedchars=\true,
	tabsize=4,
	language=python,
	showstringspaces=false,
	showtabs=false,
	frame=lrtb,
	columns=fixed,
	keepspaces,
	breaklines=true
}
\graphicspath{ {pictures} }
\setlength{\parindent}{4em}

\begin{document}
\begin{Large}
\section*{Комбінації, перестановки, розміщення}
\begin{list}{-}{}
	\item Комбінації $C_n^k=\frac{n!}{k!(n-k)!}$ - \textit{Порядок не важливий}
	\item Комбінації з повтореннями $\overline{C_n^k}=C_{n+k-1}^k$
	\item Розміщення $A_n^k=k!C_n^k=\frac{n!}{(n-k)!}$ - \textit{Порядок важливий}
	\item Розміщення з повтореннями $\overline{A_n^k}=n^k$
	\item Перестановки $P_n=n!$
	\item Перестановки з повтореннями $\overline{P_n}=\frac{n!}{n_1!n_2!...n_k!}$
\end{list}

\section*{Математичне сподівання}
$M(x)=np)$
\section*{Дисперсія}
$D(x)=npq$

\section*{Функція розподілу}
\begin{enumerate}
	\item Неспадна функція
	\item Значення лежить в межах [0;1]
	\item Неперервна зліва
	\item \{0 .. 1
\end{enumerate}

\section*{Щільність розподілу}
\begin{enumerate}
	\item $> 0$
	\item $\int_{-\infty}^{+\infty}f(x)dx = 1$
\end{enumerate}

\section*{Найімовірніше число появи випадкової події}
$np-q\le m_0 \le np+p$

\section*{Схема Бернуллі}
Ймовірність того, що незалежна подія настане рівно $m$ разів з $n$ випробувань
$P_n(m)=C_n^m\cdot p^m\cdot q^{n-m}$

\section*{Локальна теорема Мавра-Лапласа}
Яка ймовірність настання незалежної події рівно $m$ разів з $n$ випробувань з ймовірністю успіху $p$ < ймовірності невдачі $q$

$P_n(m)=\frac{1}{\sqrt{npq}}\phi(x)$, $x=\frac{m - np}{\sqrt{npq}}$

\section*{Інтегральна теорема Мавра-Лапласа}
Яка ймовірність настання незалежної події від $m_1$ до $m_2$ разів з $n$ випробувань з ймовірністю успіху $p$ < ймовірності невдачі $q$

$P_n(m)=\Phi(x_2)-\Phi(x_1)$, $x_1=\frac{m_1 - np}{\sqrt{npq}}$, $x_2=\frac{m_2 - np}{\sqrt{npq}}$

$\Phi(-x) = -\Phi(x)$

\section*{Розподіли}
\subsection*{Дискретні розподіли}
Дано $n$ - кількість випробувань
\begin{list}{-}{}
	\item Пуассона ($p$ $\downarrow\downarrow$ $n$ $\uparrow\uparrow$, $np < 10$)
	\item Геометричний (до першого успіху)
	\item Біномний - незалежні спроби
\end{list}
\subsection*{Неперервні розподіли}
\begin{list}{-}{}
	\item Рівномірний - $\text{щільність розподілу} = const$
	\item Показниковий - $\lambda$
	\item Нормальний - $\sigma$ $a$
\end{list}
\end{Large}
\end{document}